\documentclass{report}

\usepackage{hyperref}
\usepackage{amsfonts}
\usepackage{amssymb}
\usepackage{amsmath}
\usepackage{amsthm}
\usepackage[utf8]{inputenc}
\usepackage[german]{babel}
\usepackage{tikz}
\usepackage{thmtools}
\usepackage{titlesec}
\usepackage{enumitem}
\usepackage{pgfplots}
\usepackage{relsize}
\usepackage{imakeidx}
\usepackage{framed}
\usepackage{etoolbox}
\usepackage{float}

% configure tikz circles & nodes
\tikzset{
  node style sp/.style={draw,circle,minimum size=1cm},
  node style ge/.style={circle,minimum size=1cm},
  arrow style mul/.style={draw,sloped,midway,fill=white},
  arrow style plus/.style={midway,sloped,fill=white},
}
% tikz dependencies:
\usetikzlibrary{matrix,arrows,decorations.pathmorphing, decorations.markings, positioning, tikzmark}

% page size & margins:
\usepackage[a4paper, top = 2cm, left = 2.25cm, right = 2.25cm, bottom = 2cm]{geometry}

% defines index commands with hyperref
\makeindex[columns=3, title=Stichwortverzeichnis, intoc=true,options={-s index_style.ist}]
\newcommand{\BH}[1]{\textbf{\hyperpage{#1}}}
\newcommand{\IN}[1]{\index{#1|BH}}

\title{Lineare Algebra II\\Mitschrieb}

% title formatting of sections
\titleformat{\chapter}[block]
{\normalfont\huge\bfseries}{\Huge \thechapter. }{0em}{\Huge}
\titlespacing*{\chapter}{0pt}{-15pt}{20pt}
\titlespacing*{\section}{0pt}{0pt}{10pt}
\titlespacing*{\subsection}{0pt}{0pt}{10pt}


% ease of use commands
\newcommand{\lb}{\lambda}
\newcommand{\al}{\alpha}
\newcommand{\be}{\beta}

\newcommand{\mlb}{\(\lb\)}
\newcommand{\ii}{\mathrm{i}}
\newcommand{\ee}{\mathrm{e}}
\newcommand{\R}{\mathbb{R}}
\newcommand{\N}{\mathbb{N}}
\newcommand{\Z}{\mathbb{Z}}
\newcommand{\Q}{\mathbb{Q}}
\newcommand{\C}{\mathbb{C}}
\newcommand{\mR}{\(\mathbb{R}\)}
\newcommand{\mN}{\(\mathbb{N}\)}
\newcommand{\mZ}{\(\mathbb{Z}\)}
\newcommand{\mQ}{\(\mathbb{Q}\)}
\newcommand{\mC}{\(\mathbb{C}\)}
\newcommand{\Rn}{\mathbb{R}^n}
\newcommand{\mRn}{\(\mathbb{R}^n\)}
\newcommand{\ve}[1]{{\begin{pmatrix}#1 \end{pmatrix}}}
\renewcommand{\v}{\ve}
\newcommand{\baseb}{\mathcal{B}}
\newcommand{\basea}{\mathcal{A}}
\newcommand{\En}{\mathrm{E}_n}

\DeclareMathOperator{\abb}{Abb}
\DeclareMathOperator{\Span}{span}
\DeclareMathOperator{\Hom}{Hom}
\DeclareMathOperator{\End}{End}
\DeclareMathOperator{\Aut}{Aut}
\DeclareMathOperator{\Ima}{Im}
\DeclareMathOperator{\Ker}{Ker}
\DeclareMathOperator{\rg}{rg}
\DeclareMathOperator{\Mat}{Mat}
\DeclareMathOperator{\Id}{Id}
\DeclareMathOperator{\GL}{GL}
\DeclareMathOperator{\M}{M}
\DeclareMathOperator{\Sym}{Sym}
\DeclareMathOperator{\sign}{sign}
\DeclareMathOperator{\SL}{SL}
\DeclareMathOperator{\adj}{adj}
\DeclareMathOperator{\vol}{vol}
\DeclareMathOperator{\PE}{PE}
\DeclareMathOperator{\MM}{M}
\DeclareMathOperator{\TR}{Tr}
\DeclareMathOperator{\Grad}{Grad}


% increase line height
\renewcommand{\baselinestretch}{1.3}

% redefines description label for alternative enumeration in description environement
\renewcommand*\descriptionlabel[1]{\hspace\labelsep\emph{#1}}

% define main environment used for lemmas, definitions, corollaries, ...
\makeatletter
\newtheoremstyle{customdef} % name of the style to be used
{12pt}        % measure of space to leave above the theorem. E.g.: 3pt
{12pt}        % measure of space to leave below the theorem. E.g.: 3pt
{\normalfont\addtolength{\@totalleftmargin}{7pt}\addtolength{\linewidth}{-7pt}\parshape 1 7pt\linewidth} % name of font to use in the body of the theorem
{-7pt}        % measure of space to indent
{\bfseries}   % name of head font
{ \\[.125cm] }% punctuation between head and body
{ }           % space after theorem head; " " = normal interword space
{\thmname{#1}\thmnumber{ #2} {\normalfont\thmnote{ -- \hspace{1pt}  \defemph{#3}}}}

\newtheoremstyle{customenv} % name of the style to be used
{12pt}        % measure of space to leave above the theorem. E.g.: 3pt
{12pt}        % measure of space to leave below the theorem. E.g.: 3pt
{\normalfont\addtolength{\@totalleftmargin}{7pt}\addtolength{\linewidth}{-7pt}\parshape 1 7pt\linewidth} % name of font to use in the body of the theorem
{-7pt}        % measure of space to indent
{\bfseries}   % name of head font
{ \\[.125cm] }% punctuation between head and body
{ }           % space after theorem head; " " = normal interword space
{#3}
\makeatother

% define side environment used for remarks
\newtheoremstyle{customrem} % name of the style to be used
{0pt}         % measure of space to leave above the theorem. E.g.: 3pt
{0pt}         % measure of space to leave below the theorem. E.g.: 3pt
{}            % name of font to use in the body of the theorem
{}            % measure of space to indent
{\itshape}    % name of head font
{}            % punctuation between head and body
{.5em}        % space after theorem head; " " = normal interword space
{}

% define remarks:
\theoremstyle{customrem}
\newtheorem*{bemerkung}{Bemerkung\textnormal:}
\newtheorem*{bemerkung2}{Bemerkungen\textnormal:}
\newenvironment{bemerkungen}[1][]{\begin{bemerkung2}[#1]\leavevmode}{\end{bemerkung2}}

% define definition environment names
\theoremstyle{customdef}
\newtheorem{definition}{Definition}[chapter]
\newtheorem*{definitionn}{Definition} % without numbering
\newtheorem{altdefinition}[definition]{Alternative Definition}
\newtheorem{proposition}[definition]{Proposition}
\newtheorem{lemma}[definition]{Lemma}
\newtheorem{korrolar}[definition]{Korollar}
\newtheorem{satz}[definition]{Satz}
\newtheorem*{satz*}{Satz} % without numbering
\renewenvironment{proof}{\paragraph{Beweis: }}{\qed}
\theoremstyle{customenv}
\newtheorem*{customenv}{customenv} % weird

% define definition emphasis
\newcommand{\defemph}[1]{\textsc{#1}}

% disables autoindent
\setlength{\parindent}{0em}

% defines lists of theorems:

\let\Chaptermark\chaptermark
\def\chaptermark#1{\def\Chaptername{#1}\Chaptermark{#1}}
\makeatletter
% \AtBeginDocument{\addtocontents{loe}{\protect\thmlopatch@chapter{0 - Lineare Gleichungssysteme und der n-dimensionale reelle Raum}}} % fixes chapter 0 not showing by patching this shit to the beginning of the .loe
\patchcmd\thmtlo@chaptervspacehack
{\addtocontents{loe}{\protect\addvspace{10\p@}}}
{\addtocontents{loe}{\protect\thmlopatch@endchapter\protect\thmlopatch@chapter{\thechapter\space-\space\Chaptername}}}
{}{}
\AtEndDocument{\addtocontents{loe}{\protect\thmlopatch@endchapter}}
\long\def\thmlopatch@chapter#1#2\thmlopatch@endchapter{%
  \setbox\z@=\vbox{#2}%
  \ifdim\ht\z@>\z@
  \addvspace{10\p@}
  \hbox{\bfseries\chaptername\ #1}\nobreak
  #2
  \addvspace{10\p@}
  \fi
}
\def\thmlopatch@endchapter{}

\def\ll@definition{%
  \protect\thmtopatch@numbernametext
  \ifx\@empty\thmt@shortoptarg\else[\thmt@shortoptarg]\fi
  {\csname the\thmt@envname\endcsname}%
  {\thmt@thmname}%
}

\def\ll@definitionn{%
  \protect\thmtopatch@numbernametext
  \ifx\@empty\thmt@shortoptarg\else[\thmt@shortoptarg]\fi
  {\csname the\thmt@envname\endcsname}%
  {\thmt@thmname}%
}

\def\ll@altdefinition{%
  \protect\thmtopatch@numbernametext
  \ifx\@empty\thmt@shortoptarg\else[\thmt@shortoptarg]\fi
  {\csname the\thmt@envname\endcsname}%
  {\thmt@thmname}%
}

\def\ll@satz{%
  \protect\thmtopatch@numbernametext
  \ifx\@empty\thmt@shortoptarg\else[\thmt@shortoptarg]\fi
  {\csname the\thmt@envname\endcsname}%
  {\thmt@thmname}%
}

\def\ll@lemma{%
  \protect\thmtopatch@numbernametext
  \ifx\@empty\thmt@shortoptarg\else[\thmt@shortoptarg]\fi
  {\csname the\thmt@envname\endcsname}%
  {\thmt@thmname}%
}

\def\ll@korrolar{%
  \protect\thmtopatch@numbernametext
  \ifx\@empty\thmt@shortoptarg\else[\thmt@shortoptarg]\fi
  {\csname the\thmt@envname\endcsname}%
  {\thmt@thmname}%
}

\def\ll@customenv{%
  \protect\thmtopatch@numbernametext
  \ifx\@empty\thmt@shortoptarg\else[\thmt@shortoptarg]\fi
  {\csname the\thmt@envname\endcsname}%
  {\thmt@thmname}%
}

\newcommand\thmtopatch@numbernametext[3][]{%
  #3 #2%
  \if\relax\detokenize{#1}\relax\else:\space\space #1\fi
}

\makeatother


% replace ugly hyperref boxes with colored text
\usepackage{xcolor}
\hypersetup{
  colorlinks,
  linkcolor = {red!50!black},
  urlcolor  = {blue!80!black}
}
\setcounter{chapter}{-1}

% define coefficient matrix environment
\makeatletter
\renewcommand*\env@matrix[1][*\c@MaxMatrixCols c]{%
  \hskip -\arraycolsep
  \let\@ifnextchar\new@ifnextchar
  \array{#1}}
\makeatother

\makeatletter
\newsavebox\myboxA
\newsavebox\myboxB
\newlength\mylenA

% define new overline command
\newcommand*\xoverline[2][.8]{%
  \sbox{\myboxA}{$\m@th#2$}%
  \setbox\myboxB\null% Phantom box
  \ht\myboxB=\ht\myboxA%
  \dp\myboxB=\dp\myboxA%
  \wd\myboxB=#1\wd\myboxA% Scale phantom
  \sbox\myboxB{$\m@th\overline{\copy\myboxB}$}%  Overlined phantom
  \setlength\mylenA{\the\wd\myboxA}%   calc width diff
  \addtolength\mylenA{-\the\wd\myboxB}%
  \ifdim\wd\myboxB<\wd\myboxA%
  \rlap{\hskip 0.5\mylenA\usebox\myboxB}{\usebox\myboxA}%
  \else
  \hskip -0.5\mylenA\rlap{\usebox\myboxA}{\hskip 0.5\mylenA\usebox\myboxB}%
  \fi}
\makeatother

% defines bigslant for quotient (and bigbigslant for single nested quotients)
\newcommand{\bigslant}[2]{{\raisebox{.1em}{\(#1\)}\hspace{-.2em}\left/\raisebox{-.1em}{\(#2\)}\right.}}
\newcommand{\bigbigslant}[2]{{\raisebox{.4em}{\(#1\)}\hspace{-.2em}\left/\raisebox{-.4em}{\(#2\)}\right.}}

\begin{document}
\begin{titlepage}
  \newcommand{\HRule}{\rule{\linewidth}{0.5mm}}
  \centering
  \vspace{6cm}
  \textsc{\large \thinspace}\\[0.5cm]
  \vspace{4cm}
  \HRule \\[0.8cm]
  { \Huge  \textbf{Lineare Algebra \(\mathbf{II}\)}}\\[0.4cm]
  \HRule \\[.5cm]
  {\Large Inoffizieller Mitschrieb}\\[1.0cm]
  Stand: \today
  \\[11.5cm]
  \begin{minipage}{0.65\textwidth}
    \begin{center} \large
      \textsl{Vorlesung gehalten von:}\\[1cm]
      Prof. Dr. Amador Martín-Pizarro\\
      Abteilung für Angewandte Mathematik\\
      \textsc{\large Albert-Ludwigs-Universität Freiburg}
    \end{center}
  \end{minipage}\\[2.5cm]
  \thispagestyle{empty}
\end{titlepage}


\chapter{Recap}

\begin{definition}[Ring]
  Ein (kommutativer) Ring (mit Einselement) ist eine Menge zusammen mit zwei
  bin\"aren Operationen \(+, \cdot\), derart, dass:\\
  \begin{itemize}
    \item{\((R, +)\) ist eine abelsche Gruppe}
    \item{\((R, \cdot)\) ist eine kommutative Halbgruppe}
    \item{die Dsitributivgesetze:\\
    \(a(x+y) = ax + ay\)\\
    \((x + y) z = xz + yz\))}
  \end{itemize}
\end{definition}

\begin{definition}[Integrit\"atsbereich]
  Ein Integrit\"atsbereich ist ein Ring ohne Nullteiler. Also
  \(\forall x, y \in R : x \cdot y = 0 \Rightarrow x = 0 \lor y = 0\)
\end{definition}

\begin{definition}[K\"orper]
  Ein K\"orper ist ein Ring der Art, dass
  \begin{enumerate}
    \item{\(1 \neq 0\)}
    \item{
      \(
        \forall x \in K : x \neq 0
        \Rightarrow \exists x^{-1} : xx^{-1} = x^{-1}x = 1
      \)
    }
  \end{enumerate}
\end{definition}

\begin{bemerkung}
  K\"orper sind Integrit\"atsbereiche.
\end{bemerkung}

\begin{definition}[Charakteristik]
  Sei R ein nicht trivialer Ring \((0 \neq 1)\).
  \(
    \varphi : \Z \rightarrow R, z \mapsto
    \begin{cases}
      \sum_{i=1}^n 1 & n >= 0\\
      -\sum_{i=1}^n 1 & \text{ansonsten}
    \end{cases}
  \)\\
  Dann ist \(\varphi\) ein Ringhomomorphismus.\\
  F\"ur den Kern von \(\varphi\) (\(\Ker(\varphi)\)) gibt es zwei
  M\"oglichkeiten.
  \begin{enumerate}
    \item{\(\Ker(\varphi) = \{0\}, p = 0\)}
    \item{
      \(\Ker(\varphi) \neq \{0\}\). Dann gibt es ein kleinstes echt positives
      Element \(p \in \Ker(\varphi)\).
    }
  \end{enumerate}
  R hat dann Charakteristik p. Falls R ein Integritaetsbereich ist, dann ist p
  eine Primzahl.
  \textbf{Beispiele:}\\
  \(\Z / n\Z = \{\bar{0}, \dots, \bar{n}\}\) hat Charakteristik n.\\
  Insbesondere enth\"alt jeder K\"orper mit Charakteristik p eine ''Kopie'' von
  \(\Z / m\Z\):\\
  k hat Charakteristik p \(\Rightarrow \Z /p\Z
  \overset{injectiv}{\leftrightarrow} K\).\\
  Hier ist \(\Z/p\Z\) ein K\"orper:\\
  \(a \in \Z/p\Z \setminus \{0\} \Rightarrow\) es ist a mit p teilerfremd.
  \(1 = a \cdot b + p \cdot m \Rightarrow \bar{1} = \bar{a} \cdot \bar{b}\).
\end{definition}

\begin{definition}[Polynomring]
  Sei K ein K\"orper. Der Polynomring \(K[T]\) in einer Variable R \"uber K ist
  die Menge formeller Summen der Form:\\
  \(f = \sum_{i = 0}^n a_i \cdot T^i, n \in \N\)\\
  Der Grad von \(f \in K[T]\) ist definiert als:\\
  \(\Grad(f) := max(m | m < n \land a_m \neq 0)\)\\
  \(\Grad(0) := -1\)
  Die Summe und das Produkt von Polynomen sind definiert als:\\
  \(
    \sum_{i=0}^n a_i T^i + \sum_{j=0}^m b_j T^j 
    := \sum_{k=0}^{max(m,n)} (a_k b_k) T^k
  \)\\
  \(
    \sum_{i=0}^n a_i T^i \cdot \sum_{j=0}^m b_j T^j 
    := \sum_{k=0}^{m+j} = c_K T^k, c_k = \sum_i+j=k a_i b_j
  \)\\
\end{definition}

\begin{bemerkung}
  \(K[T]\) ist ein Integrit\"atsbereich.\\
\end{bemerkung}

\begin{korrolar}
  Es seien \(f, g\) beide \(\neq 0\)\\
  \(
    \Rightarrow \Grad(f \cdot g)
    = \Grad(f) + \Grad(g) \Rightarrow f \cdot g \neq 0
  \)\\
  \(\Grad(f+g) \le max(\Grad(f), \Grad(g))\)\\
\end{korrolar}

\begin{satz}[Division mit Rest]
  Gegeben \(f, g \in K[T], \Grad(g) > 0\). Dann existieren eindeutige 
  Polynome q, r, so dass \(f = g \dot q + r\), wobei \(\Grad(r) < \Grad(g)\).
  \begin{proof}
    Eindeutigkeit: Angenommen
    \(f = g \cdot q + r = g \cdot q' + r', q \neq q' \lor r \neq r'\).\\
    \(
      \Rightarrow g(q - q') = r' - r
      \Rightarrow \Grad(r' - r) = max(\Grad(r'), \Grad(r)) < \Grad(g)
                  = \Grad(g(q - q'))
      \Rightarrow\) Widerspruch \(
      \Rightarrow q = q' \Rightarrow r = r'
    \)
    Existenz: Induktion auf \(\Grad(f)\)\\
    \(\Grad(f) = 0 \Rightarrow f = g \cdot 0 + f\)\\
    \(\Grad(f) = n + 1\)\\
    \(\Grad(f) < \Grad(g) = m \Rightarrow f = g \cdot 0 + f\)\\
    OBdA. \(n + 1 = \Grad(f) \ge \Grad(g) = m > 0\)\\
    \(f = a_{n + 1} \cdot T^{n+1} + \hat{f},
    \Grad(\hat{f}) \le n, a_{n + 1} \neq 0\)\\
    Sei \(
      f' = f - b_m^{-1} a_{n+1}T^{n+1-m} \cdot g \Rightarrow \Grad(f') \le n
    \)
    Ia: \(f' = g \cdot q' + r', \Grad(r') < \Grad(g)\)\\
    \(
      f' = f - b-b^{-1}a_{n+1}T{n + 1 - m} \cdot g
      \Rightarrow f = g(b_n^{-1}a_{n+1}T^{n+1-m} +q')+r'
      \Rightarrow \Grad(r') < \Grad(g)
    \)
  \end{proof}
\end{satz}

\begin{definition}[Polynom Teilt]
  \(f, g, q \in K[T]\), \(\Grad(g) > 0\)\\
  \(g \text{ teilt } f  = g|_f\Leftrightarrow f = g \cdot q\)\\
\end{definition}

\begin{definition}[Nullstellen von Polynomen]
  \(f \in K[T]\) besizt eine Nullstelle \(\lb \in K\) gdw.
  \((T - \lb) |_f \Leftrightarrow f(\lb) = 0\).\\
  f l\"asst sich dann schreiben als \(f = (T - \lb)q + r\).
\end{definition}

\begin{lemma}
  \(f \in K[t], f\neq 0, \Grad(f) = n
  \Rightarrow \) f besitzt h\"ochstens n Nullstellen in k.
  \begin{proof}
    \begin{itemize}
      \item[\(n=0\)]{
        \(\Rightarrow f = a_0, a_0 \neq 0\)
      }
      \item[\(n>0\)]{
        Falls f keine Nullstellen in K besitzt \(\Rightarrow\) ok!\\
        Sonst, sei \(\lb \in K\) eine Nullstelle von f.
        \(f = (T -\lb) \cdot g, \Grad(g) = n-1 < n\)\\
        I.A besitzt g h\"ochstens n - 1 Nullstellen. Jede Nullstelle von f ist
        entweder \(\lb\) oder eine Nullstelle von g. \(\Rightarrow\) f hat
        h\"ochstens n Nullstellen.\\
      }
    \end{itemize}
  \end{proof}
\end{lemma}

\begin{definition}[Vielfachheit einer Nulstelle]
  \(f \in K[T], f \neq 0, \lb \in K\) Nullstelle von f
  \(\Rightarrow f = (T-\lb)^{K_\lb}\cdot g, g(\lb \neq 0\).
  \(K_\lb\) ist die Vielfacheit der Nullstelle \(\lb\) in f.\\
\end{definition}

\begin{definition}
  Ein K\"orper hei\ss{}t algebraisch abgeschlossen, falls jedes Polynom \"uber 
  K positiven Grades eine Nullstelle besitzt.
\end{definition}

\textbf{Beispiele}
Ist \(\R\) algebraisch abgeschlossen? Nein: \(T^2 + 1\).\\
Bem.: \(\C\) ist algebraisch abgeschlossen.

\begin{bemerkung}
  Jeder algebraisch abgeschlossene K\"orper muss unendlich sein.
  Sei \(K = \{\lb_1, \dots \lb_n\}, f = (T - \lb), \dots, (T - \lb_n) + 1\).\\
\end{bemerkung}

\begin{lemma}
  K ist genau dann algebraisch abgeschlossen, wenn jedes Polynom positiven
  Grades in lineare Faktoren zerf\"allt.\\
  \(f = T(\lb_1) \dots (T - \lb_n)\).
  \begin{proof}
    \begin{itemize}
      \item[\(\Leftarrow\)]{trivial}
      \item[\(\Rightarrow\)]{
        \(
          \Grad(f) = n > 0
          \Rightarrow f = (T - \lb_1) \cdot g, \Grad(g) \le n - 1 < n
          \overset{I.A.}{\Rightarrow} f = c(T-\lb_1) \dots (T-\lb_n)
        \)
      }
    \end{itemize}
  \end{proof}
\end{lemma}

\newpage
\renewcommand{\listtheoremname}{Satz- und Definitionsverzeichnis}
% \listoftheorems[ignoreall, show={definition}, show={satz}, show={lemma}, show={definitionn}, show={korrolar}, show={altdefinition}]
\addcontentsline{toc}{chapter}{Satz- und Definitionsverzeichnis}
\newpage
\printindex
\end{document}
