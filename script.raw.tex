\documentclass{report}

\usepackage{hyperref}
\usepackage{amsfonts}
\usepackage{amssymb}
\usepackage{amsmath}
\usepackage{amsthm}
\usepackage[utf8]{inputenc}
\usepackage[german]{babel}
\usepackage{tikz}
\usepackage{thmtools}
\usepackage{titlesec}
\usepackage{enumitem}
\usepackage{pgfplots}
\usepackage{relsize}
\usepackage{imakeidx}
\usepackage{framed}
\usepackage{etoolbox}
\usepackage{float}

% configure tikz circles & nodes
\tikzset{
  node style sp/.style={draw,circle,minimum size=1cm},
  node style ge/.style={circle,minimum size=1cm},
  arrow style mul/.style={draw,sloped,midway,fill=white},
  arrow style plus/.style={midway,sloped,fill=white},
}
% tikz dependencies:
\usetikzlibrary{matrix,arrows,decorations.pathmorphing, decorations.markings, positioning, tikzmark}

% page size & margins:
\usepackage[a4paper, top = 2cm, left = 2.25cm, right = 2.25cm, bottom = 2cm]{geometry}

% defines index commands with hyperref
\makeindex[columns=3, title=Stichwortverzeichnis, intoc=true,options={-s index_style.ist}]
\newcommand{\BH}[1]{\textbf{\hyperpage{#1}}}
\newcommand{\IN}[1]{\index{#1|BH}}

\title{Lineare Algebra II\\Mitschrieb}

% title formatting of sections
\titleformat{\chapter}[block]
{\normalfont\huge\bfseries}{\Huge \thechapter. }{0em}{\Huge}
\titlespacing*{\chapter}{0pt}{-15pt}{20pt}
\titlespacing*{\section}{0pt}{0pt}{10pt}
\titlespacing*{\subsection}{0pt}{0pt}{10pt}


% ease of use commands
\newcommand{\lb}{\lambda}
\newcommand{\al}{\alpha}
\newcommand{\be}{\beta}

\newcommand{\mlb}{\(\lb\)}
\newcommand{\ii}{\mathrm{i}}
\newcommand{\ee}{\mathrm{e}}
\newcommand{\R}{\mathbb{R}}
\newcommand{\N}{\mathbb{N}}
\newcommand{\Z}{\mathbb{Z}}
\newcommand{\Q}{\mathbb{Q}}
\newcommand{\C}{\mathbb{C}}
\newcommand{\mR}{\(\mathbb{R}\)}
\newcommand{\mN}{\(\mathbb{N}\)}
\newcommand{\mZ}{\(\mathbb{Z}\)}
\newcommand{\mQ}{\(\mathbb{Q}\)}
\newcommand{\mC}{\(\mathbb{C}\)}
\newcommand{\Rn}{\mathbb{R}^n}
\newcommand{\mRn}{\(\mathbb{R}^n\)}
\newcommand{\ve}[1]{{\begin{pmatrix}#1 \end{pmatrix}}}
\renewcommand{\v}{\ve}
\newcommand{\baseb}{\mathcal{B}}
\newcommand{\basea}{\mathcal{A}}
\newcommand{\En}{\mathrm{E}_n}

\DeclareMathOperator{\abb}{Abb}
\DeclareMathOperator{\Span}{span}
\DeclareMathOperator{\Hom}{Hom}
\DeclareMathOperator{\End}{End}
\DeclareMathOperator{\Aut}{Aut}
\DeclareMathOperator{\Ima}{Im}
\DeclareMathOperator{\Ker}{Ker}
\DeclareMathOperator{\rg}{rg}
\DeclareMathOperator{\Mat}{Mat}
\DeclareMathOperator{\Id}{Id}
\DeclareMathOperator{\GL}{GL}
\DeclareMathOperator{\M}{M}
\DeclareMathOperator{\Sym}{Sym}
\DeclareMathOperator{\sign}{sign}
\DeclareMathOperator{\SL}{SL}
\DeclareMathOperator{\adj}{adj}
\DeclareMathOperator{\vol}{vol}
\DeclareMathOperator{\PE}{PE}
\DeclareMathOperator{\MM}{M}
\DeclareMathOperator{\TR}{Tr}
\DeclareMathOperator{\Grad}{Grad}
\DeclareMathOperator{\Char}{Char}
\DeclareMathOperator{\ord}{ord}


% increase line height
\renewcommand{\baselinestretch}{1.3}

% redefines description label for alternative enumeration in description environement
\renewcommand*\descriptionlabel[1]{\hspace\labelsep\emph{#1}}

% define main environment used for lemmas, definitions, corollaries, ...
\makeatletter
\newtheoremstyle{customdef} % name of the style to be used
{12pt}        % measure of space to leave above the theorem. E.g.: 3pt
{12pt}        % measure of space to leave below the theorem. E.g.: 3pt
{\normalfont\addtolength{\@totalleftmargin}{7pt}\addtolength{\linewidth}{-7pt}\parshape 1 7pt\linewidth} % name of font to use in the body of the theorem
{-7pt}        % measure of space to indent
{\bfseries}   % name of head font
{ \\[.125cm] }% punctuation between head and body
{ }           % space after theorem head; " " = normal interword space
{\thmname{#1}\thmnumber{ #2} {\normalfont\thmnote{ -- \hspace{1pt}  \defemph{#3}}}}

\newtheoremstyle{customenv} % name of the style to be used
{12pt}        % measure of space to leave above the theorem. E.g.: 3pt
{12pt}        % measure of space to leave below the theorem. E.g.: 3pt
{\normalfont\addtolength{\@totalleftmargin}{7pt}\addtolength{\linewidth}{-7pt}\parshape 1 7pt\linewidth} % name of font to use in the body of the theorem
{-7pt}        % measure of space to indent
{\bfseries}   % name of head font
{ \\[.125cm] }% punctuation between head and body
{ }           % space after theorem head; " " = normal interword space
{#3}
\makeatother

% define side environment used for remarks
\newtheoremstyle{customrem} % name of the style to be used
{0pt}         % measure of space to leave above the theorem. E.g.: 3pt
{0pt}         % measure of space to leave below the theorem. E.g.: 3pt
{}            % name of font to use in the body of the theorem
{}            % measure of space to indent
{\itshape}    % name of head font
{}            % punctuation between head and body
{.5em}        % space after theorem head; " " = normal interword space
{}

% define remarks:
\theoremstyle{customrem}
\newtheorem*{bemerkung}{Bemerkung\textnormal:}
\newtheorem*{bemerkung2}{Bemerkungen\textnormal:}
\newenvironment{bemerkungen}[1][]{\begin{bemerkung2}[#1]\leavevmode}{\end{bemerkung2}}

% define definition environment names
\theoremstyle{customdef}
\newtheorem{definition}{Definition}[chapter]
\newtheorem*{definitionn}{Definition} % without numbering
\newtheorem{altdefinition}[definition]{Alternative Definition}
\newtheorem{proposition}[definition]{Proposition}
\newtheorem{lemma}[definition]{Lemma}
\newtheorem{korrolar}[definition]{Korollar}
\newtheorem{satz}[definition]{Satz}
\newtheorem*{satz*}{Satz} % without numbering
\renewenvironment{proof}{\paragraph{Beweis: }}{\qed}
\theoremstyle{customenv}
\newtheorem*{customenv}{customenv} % weird

% define definition emphasis
\newcommand{\defemph}[1]{\textsc{#1}}

% disables autoindent
\setlength{\parindent}{0em}

% defines lists of theorems:

\let\Chaptermark\chaptermark
\def\chaptermark#1{\def\Chaptername{#1}\Chaptermark{#1}}
\makeatletter
% \AtBeginDocument{\addtocontents{loe}{\protect\thmlopatch@chapter{0 - Lineare Gleichungssysteme und der n-dimensionale reelle Raum}}} % fixes chapter 0 not showing by patching this shit to the beginning of the .loe
\patchcmd\thmtlo@chaptervspacehack
{\addtocontents{loe}{\protect\addvspace{10\p@}}}
{\addtocontents{loe}{\protect\thmlopatch@endchapter\protect\thmlopatch@chapter{\thechapter\space-\space\Chaptername}}}
{}{}
\AtEndDocument{\addtocontents{loe}{\protect\thmlopatch@endchapter}}
\long\def\thmlopatch@chapter#1#2\thmlopatch@endchapter{%
  \setbox\z@=\vbox{#2}%
  \ifdim\ht\z@>\z@
  \addvspace{10\p@}
  \hbox{\bfseries\chaptername\ #1}\nobreak
  #2
  \addvspace{10\p@}
  \fi
}
\def\thmlopatch@endchapter{}

\def\ll@definition{%
  \protect\thmtopatch@numbernametext
  \ifx\@empty\thmt@shortoptarg\else[\thmt@shortoptarg]\fi
  {\csname the\thmt@envname\endcsname}%
  {\thmt@thmname}%
}

\def\ll@definitionn{%
  \protect\thmtopatch@numbernametext
  \ifx\@empty\thmt@shortoptarg\else[\thmt@shortoptarg]\fi
  {\csname the\thmt@envname\endcsname}%
  {\thmt@thmname}%
}

\def\ll@altdefinition{%
  \protect\thmtopatch@numbernametext
  \ifx\@empty\thmt@shortoptarg\else[\thmt@shortoptarg]\fi
  {\csname the\thmt@envname\endcsname}%
  {\thmt@thmname}%
}

\def\ll@satz{%
  \protect\thmtopatch@numbernametext
  \ifx\@empty\thmt@shortoptarg\else[\thmt@shortoptarg]\fi
  {\csname the\thmt@envname\endcsname}%
  {\thmt@thmname}%
}

\def\ll@lemma{%
  \protect\thmtopatch@numbernametext
  \ifx\@empty\thmt@shortoptarg\else[\thmt@shortoptarg]\fi
  {\csname the\thmt@envname\endcsname}%
  {\thmt@thmname}%
}

\def\ll@korrolar{%
  \protect\thmtopatch@numbernametext
  \ifx\@empty\thmt@shortoptarg\else[\thmt@shortoptarg]\fi
  {\csname the\thmt@envname\endcsname}%
  {\thmt@thmname}%
}

\def\ll@customenv{%
  \protect\thmtopatch@numbernametext
  \ifx\@empty\thmt@shortoptarg\else[\thmt@shortoptarg]\fi
  {\csname the\thmt@envname\endcsname}%
  {\thmt@thmname}%
}

\newcommand\thmtopatch@numbernametext[3][]{%
  #3 #2%
  \if\relax\detokenize{#1}\relax\else:\space\space #1\fi
}

\makeatother


% replace ugly hyperref boxes with colored text
\usepackage{xcolor}
\hypersetup{
  colorlinks,
  linkcolor = {red!50!black},
  urlcolor  = {blue!80!black}
}
\setcounter{chapter}{-1}

% define coefficient matrix environment
\makeatletter
\renewcommand*\env@matrix[1][*\c@MaxMatrixCols c]{%
  \hskip -\arraycolsep
  \let\@ifnextchar\new@ifnextchar
  \array{#1}}
\makeatother

\makeatletter
\newsavebox\myboxA
\newsavebox\myboxB
\newlength\mylenA

% define new overline command
\newcommand*\xoverline[2][.8]{%
  \sbox{\myboxA}{$\m@th#2$}%
  \setbox\myboxB\null% Phantom box
  \ht\myboxB=\ht\myboxA%
  \dp\myboxB=\dp\myboxA%
  \wd\myboxB=#1\wd\myboxA% Scale phantom
  \sbox\myboxB{$\m@th\overline{\copy\myboxB}$}%  Overlined phantom
  \setlength\mylenA{\the\wd\myboxA}%   calc width diff
  \addtolength\mylenA{-\the\wd\myboxB}%
  \ifdim\wd\myboxB<\wd\myboxA%
  \rlap{\hskip 0.5\mylenA\usebox\myboxB}{\usebox\myboxA}%
  \else
  \hskip -0.5\mylenA\rlap{\usebox\myboxA}{\hskip 0.5\mylenA\usebox\myboxB}%
  \fi}
\makeatother

% defines bigslant for quotient (and bigbigslant for single nested quotients)
\newcommand{\bigslant}[2]{{\raisebox{.1em}{\(#1\)}\hspace{-.2em}\left/\raisebox{-.1em}{\(#2\)}\right.}}
\newcommand{\bigbigslant}[2]{{\raisebox{.4em}{\(#1\)}\hspace{-.2em}\left/\raisebox{-.4em}{\(#2\)}\right.}}

\begin{document}
\begin{titlepage}
  \newcommand{\HRule}{\rule{\linewidth}{0.5mm}}
  \centering
  \vspace{6cm}
  \textsc{\large \thinspace}\\[0.5cm]
  \vspace{4cm}
  \HRule \\[0.8cm]
  { \Huge  \textbf{Lineare Algebra \(\mathbf{II}\)}}\\[0.4cm]
  \HRule \\[.5cm]
  {\Large Inoffizieller Mitschrieb}\\[1.0cm]
  Stand: \today
  \\[11.5cm]
  \begin{minipage}{0.65\textwidth}
    \begin{center} \large
      \textsl{Vorlesung gehalten von:}\\[1cm]
      Prof. Dr. Amador Martín-Pizarro\\
      Abteilung für Angewandte Mathematik\\
      \textsc{\large Albert-Ludwigs-Universität Freiburg}
    \end{center}
  \end{minipage}\\[2.5cm]
  \thispagestyle{empty}
\end{titlepage}


\chapter{Recap}

\begin{definition}[Ring]
  Ein (kommutativer) Ring (mit Einselement) ist eine Menge zusammen mit zwei
  bin\"aren Operationen \(+, \cdot\), derart, dass:\\
  \begin{itemize}
    \item{\((R, +)\) ist eine abelsche Gruppe}
    \item{\((R, \cdot)\) ist eine kommutative Halbgruppe}
    \item{die Dsitributivgesetze:\\
    \(a(x+y) = ax + ay\)\\
    \((x + y) z = xz + yz\))}
  \end{itemize}
\end{definition}

\begin{definition}[Integrit\"atsbereich]
  Ein Integrit\"atsbereich ist ein Ring ohne Nullteiler. Also
  \(\forall x, y \in R : x \cdot y = 0 \Rightarrow x = 0 \lor y = 0\)
\end{definition}

\begin{definition}[K\"orper]
  Ein K\"orper ist ein Ring der Art, dass
  \begin{enumerate}
    \item{\(1 \neq 0\)}
    \item{
      \(
        \forall x \in K : x \neq 0
        \Rightarrow \exists x^{-1} : xx^{-1} = x^{-1}x = 1
      \)
    }
  \end{enumerate}
\end{definition}

\begin{bemerkung}
  K\"orper sind Integrit\"atsbereiche.
\end{bemerkung}

\begin{definition}[Charakteristik]
  Sei R ein nicht trivialer Ring \((0 \neq 1)\).
  \(
    \varphi : \Z \rightarrow R, z \mapsto
    \begin{cases}
      \sum_{i=1}^n 1 & n >= 0\\
      -\sum_{i=1}^n 1 & \text{ansonsten}
    \end{cases}
  \)\\
  Dann ist \(\varphi\) ein Ringhomomorphismus.\\
  F\"ur den Kern von \(\varphi\) (\(\Ker(\varphi)\)) gibt es zwei
  M\"oglichkeiten.
  \begin{enumerate}
    \item{\(\Ker(\varphi) = \{0\}, p = 0\)}
    \item{
      \(\Ker(\varphi) \neq \{0\}\). Dann gibt es ein kleinstes echt positives
      Element \(p \in \Ker(\varphi)\).
    }
  \end{enumerate}
  R hat dann Charakteristik p \((\Char(R) = p)\). Falls R ein
  Integritaetsbereich ist, dann ist p eine Primzahl.
  \textbf{Beispiele:}\\
  \(\Z / n\Z = \{\bar{0}, \dots, \bar{n}\}\) hat Charakteristik n.\\
  Insbesondere enth\"alt jeder K\"orper mit Charakteristik p eine ''Kopie'' von
  \(\Z / m\Z\):\\
  k hat Charakteristik p \(\Rightarrow \Z /p\Z
  \overset{injectiv}{\leftrightarrow} K\).\\
  Hier ist \(\Z/p\Z\) ein K\"orper:\\
  \(a \in \Z/p\Z \setminus \{0\} \Rightarrow\) es ist a mit p teilerfremd.
  \(1 = a \cdot b + p \cdot m \Rightarrow \bar{1} = \bar{a} \cdot \bar{b}\).
\end{definition}

\begin{definition}[Polynomring]
  Sei K ein K\"orper. Der Polynomring \(K[T]\) in einer Variable R \"uber K ist
  die Menge formeller Summen der Form:\\
  \(f = \sum_{i = 0}^n a_i \cdot T^i, n \in \N\)\\
  Der Grad von \(f \in K[T]\) ist definiert als:\\
  \(\Grad(f) := max(m | m < n \land a_m \neq 0)\)\\
  \(\Grad(0) := -1\)\\
  Falls \(Grad(f) = n\) und \(n = 1\) hei\ss{}t das Polynom normiert.\\
  Die Summe und das Produkt von Polynomen sind definiert als:\\
  \(
    \sum_{i=0}^n a_i T^i + \sum_{j=0}^m b_j T^j 
    := \sum_{k=0}^{max(m,n)} (a_k b_k) T^k
  \)\\
  \(
    \sum_{i=0}^n a_i T^i \cdot \sum_{j=0}^m b_j T^j 
    := \sum_{k=0}^{m+j} = c_K T^k, c_k = \sum_i+j=k a_i b_j
  \)\\
\end{definition}

\begin{bemerkung}
  \(K[T]\) ist ein Integrit\"atsbereich.\\
\end{bemerkung}

\begin{korrolar}
  Es seien \(f, g\) beide \(\neq 0\)\\
  \(
    \Rightarrow \Grad(f \cdot g)
    = \Grad(f) + \Grad(g) \Rightarrow f \cdot g \neq 0
  \)\\
  \(\Grad(f+g) \le max(\Grad(f), \Grad(g))\)\\
\end{korrolar}

\begin{satz}[Division mit Rest]
  Gegeben \(f, g \in K[T], \Grad(g) > 0\). Dann existieren eindeutige 
  Polynome q, r, so dass \(f = g \dot q + r\), wobei \(\Grad(r) < \Grad(g)\).
  \begin{proof}
    Eindeutigkeit: Angenommen
    \(f = g \cdot q + r = g \cdot q' + r', q \neq q' \lor r \neq r'\).\\
    \(
      \Rightarrow g(q - q') = r' - r
      \Rightarrow \Grad(r' - r) = max(\Grad(r'), \Grad(r)) < \Grad(g)
                  = \Grad(g(q - q'))
      \Rightarrow\) Widerspruch \(
      \Rightarrow q = q' \Rightarrow r = r'
    \)
    Existenz: Induktion auf \(\Grad(f)\)\\
    \(\Grad(f) = 0 \Rightarrow f = g \cdot 0 + f\)\\
    \(\Grad(f) = n + 1\)\\
    \(\Grad(f) < \Grad(g) = m \Rightarrow f = g \cdot 0 + f\)\\
    OBdA. \(n + 1 = \Grad(f) \ge \Grad(g) = m > 0\)\\
    \(f = a_{n + 1} \cdot T^{n+1} + \hat{f},
    \Grad(\hat{f}) \le n, a_{n + 1} \neq 0\)\\
    Sei \(
      f' = f - b_m^{-1} a_{n+1}T^{n+1-m} \cdot g \Rightarrow \Grad(f') \le n
    \)
    Ia: \(f' = g \cdot q' + r', \Grad(r') < \Grad(g)\)\\
    \(
      f' = f - b-b^{-1}a_{n+1}T{n + 1 - m} \cdot g
      \Rightarrow f = g(b_n^{-1}a_{n+1}T^{n+1-m} +q')+r'
      \Rightarrow \Grad(r') < \Grad(g)
    \)
  \end{proof}
\end{satz}

\begin{definition}[Polynom Teilt]
  \(f, g, q \in K[T]\), \(\Grad(g) > 0\)\\
  \(g \text{ teilt } f  = g|_f\Leftrightarrow f = g \cdot q\)\\
\end{definition}

\begin{definition}[Nullstellen von Polynomen]
  \(f \in K[T]\) besizt eine Nullstelle \(\lb \in K\) gdw.
  \((T - \lb) |_f \Leftrightarrow f(\lb) = 0\).\\
  f l\"asst sich dann schreiben als \(f = (T - \lb)q + r\).
\end{definition}

\begin{lemma}
  \(f \in K[t], f\neq 0, \Grad(f) = n
  \Rightarrow \) f besitzt h\"ochstens n Nullstellen in k.
  \begin{proof}
    \begin{itemize}
      \item[\(n=0\)]{
        \(\Rightarrow f = a_0, a_0 \neq 0\)
      }
      \item[\(n>0\)]{
        Falls f keine Nullstellen in K besitzt \(\Rightarrow\) ok!\\
        Sonst, sei \(\lb \in K\) eine Nullstelle von f.
        \(f = (T -\lb) \cdot g, \Grad(g) = n-1 < n\)\\
        I.A besitzt g h\"ochstens n - 1 Nullstellen. Jede Nullstelle von f ist
        entweder \(\lb\) oder eine Nullstelle von g. \(\Rightarrow\) f hat
        h\"ochstens n Nullstellen.\\
      }
    \end{itemize}
  \end{proof}
\end{lemma}

\begin{definition}[Vielfachheit einer Nullstelle]
  \(f \in K[T], f \neq 0, \lb \in K\) Nullstelle von f
  \(\Rightarrow f = (T-\lb)^{K_\lb}\cdot g, g(\lb \neq 0\).
  \(K_\lb\) ist die Vielfacheit der Nullstelle \(\lb\) in f.\\
\end{definition}

\begin{definition}
  Ein K\"orper hei\ss{}t algebraisch abgeschlossen, falls jedes Polynom \"uber 
  K positiven Grades eine Nullstelle besitzt.
\end{definition}

\textbf{Beispiele}
Ist \(\R\) algebraisch abgeschlossen? Nein: \(T^2 + 1\).\\
Bem.: \(\C\) ist algebraisch abgeschlossen.

\begin{bemerkung}
  Jeder algebraisch abgeschlossene K\"orper muss unendlich sein.
  Sei \(K = \{\lb_1, \dots \lb_n\}, f = (T - \lb), \dots, (T - \lb_n) + 1\).\\
\end{bemerkung}

\begin{lemma}
  K ist genau dann algebraisch abgeschlossen, wenn jedes Polynom positiven
  Grades in lineare Faktoren zerf\"allt.\\
  \(f = T(\lb_1) \dots (T - \lb_n)\).
  \begin{proof}
    \begin{itemize}
      \item[\(\Leftarrow\)]{trivial}
      \item[\(\Rightarrow\)]{
        \(
          \Grad(f) = n > 0
          \Rightarrow f = (T - \lb_1) \cdot g, \Grad(g) \le n - 1 < n
          \overset{I.A.}{\Rightarrow} f = c(T-\lb_1) \dots (T-\lb_n)
        \)
      }
    \end{itemize}
  \end{proof}
\end{lemma}

\begin{definition}[Vektorraum]
  Vektorraum V \"uber K ist eine abelsche Gruppe \((V, +, 0_V)\) zusammen mit
  einer Verkn\"upfung \(K \times V \to V\) \((\lb, v) \mapsto \lb v\) die die
  folgenden Bedingungen erf\"ullt:\\
  \begin{enumerate}
    \item{\(\lb (v+w) = \lb v + \lb w\)}
    \item{\(\lb(\mu \v) = (\lb \mu) v\)} 
    \item{\((\lb + \mu) v = \lb v + \mu v\)}
    \item{\(1_{k} v = v\)}
  \end{enumerate}
\end{definition}

\begin{definition}[Untervektorraum]
  Ein Untervektorraum \(U \subset V\) ist eine Untergruppe, welche unter der
  Skalarmultiplikation abgeschlossen ist. 
\end{definition}

\begin{bemerkung}
  \(\{U_i\}_{i \in I}\) Untervektorr\"aume von V
  \(\Rightarrow \bigcap_{i\in I} U_{i}\) ist Untervektorraum.
  Insb. gebenen \(M \subset V\) existiert \(\Span(M) = <M> = \) der kleinste
  Unterraum von V, der M enth\"alt.\\
  \(\Span(M) = \sum_{i=1}^{n} \lb_{i} m_{i}\),
  \(m_{i} \in M, \lb_{i} \in K, n \in \N\)\\
  M ist ein Erzeugendensystem f\"ur \(\Span(M)\)\\
  Au\ss{}erdem gilt:\\
  \(\sum_{i\in I}U_{i} = \Span(\bigcup_{i \in I} U_{i})\)\\
  \(M_{1} \subset M_{2} \Rightarrow \Span(M_{1}) \subset \Span(M_{2})\)
\end{bemerkung}

\begin{definition}[Lineare Unabh\"angigkeit]
  Sei \(V\) ein Vektorraum \"uber K. Dann gilt \(v_{1}, \dots v_{n}\) sind
  linear unabh\"angig falls
  \(\forall \lb_{1}, \dots, \lb_{n} \in K : \sum \lb_{i} v_{i} 
  \Rightarrow \lb_{1} = \dots = \lb_{n} = 0\) 
  \(M \subset V\) ist linear unabh\"angig, falls jede endliche Teilmenge von M
  linear unabh\"angig ist. \"Aquivalent dazu ist: M ist linear unabh\"angig, 
  falls kein Element von M sich als Linearkombination der anderen schreiben
  l\"asst.
\end{definition}

\begin{definition}[Basis]
  Sei \(B = \{v_1, \dots, v_n\}, v_i \in V\).
  Die folgenden Aussagen sind \"aquivalent und definieren eine Basis:
  \begin{enumerate}
    \item{B ist ein lineare unabh\"angiges Erzeugendensystem von V}
    \item{
      Jedes Element von V l\"asst sich eindeutig als Linearkombination der
      Elemente in B schreiben.
    }
    \item{B ist ein minimales Erzeugendensystem.}
    \item{B ist maximal lineare unabh\"angig.}
  \end{enumerate}
\end{definition}

\begin{satz}[Basiserg\"anzungssatz]
  Sei \(M \subset V\) lineare unabh\"angig, dann gilt \( \exists B \subset V\),
  und B ist eine Basis welche M ent\"alt. Insbesondere hat jeder Vektorraum eine
  Basis. ''Je zwei Basen sind in Bijektion''.
\end{satz}

\begin{definition}[Dimension]  
  V ist endlichdimensional, falls V eine endliche Basis besitzt. Sonst ist V
  unendlichdimensional. Fall V endlichdimensional ist, ist die Dimension von V
  definert durch:\\
  \(dim(V) = |B|\) mit B beliebeige Basis.
\end{definition}

\begin{satz}[Basisauswahlsatz]
  Sei \(M \subset V\) ein Erzeugendensystem von V, dann gilt
  \(\exists B \subset M\) mit B ist eine Basis von V.
\end{satz}

\begin{lemma}
  Sei \(U \subset V\) ein Unterraum, dann gilt
  \(\dim(V) < \infty \Rightarrow \dim(U) < \infty\)
\end{lemma}

\begin{lemma}
  Die Dimension ist modular:
  \(\dim(U_1 + U_2) + \dim(U_1 \cap U_2) = \dim(U_1) + \dim(U_2)\)
\end{lemma}

\begin{definition}[Direktes Produkt von Vektorr\"aumen]
  \(
  V = U_1 \oplus U_2 \Leftrightarrow V = U_1 + U_2 \land U_1 \cap U_2 = \{0\}
  \)\\
  \(V = \oplus_{i \in I} U_i \Leftrightarrow V = \sum_{i \in I} U_i \) und die
  Familie ist transversal:
  \(\{U_i\}_{i \in I} \to U_i \cap (\sum_{j \in I} U_j) = \{0\}\)
\end{definition}

\begin{definition}[Komplement\"ar]
  Sei \(U \subset V\) ein Unterverktorraum, dann gilt
  \(\exists \hat{U} \subset V : V = U \oplus \hat{U}\).\\
  \(\hat{U}\) hei\ss{}t dann Komplement\"ar zu U.
\end{definition}

\textbf{Beispiele}\\
\(K^2\) ist ein K-VR. \(\v{1\\0} \v{0\\1}\) ist eine Basis.\\
\(U = \Span(\v{1\\0})\). \(K^2 = U \oplus \Span(\v{0\\1})\).
\(K^2 = U \oplus \Span(\v{1 \\1}\).

\begin{definition}[Lineare Abbildungen]
  \(F : V \to W\) ist linear, falls gilt:
  \(F(\lb v + \mu u) = \lb F(v) + \mu F(u)\)
\end{definition}

\begin{definition}[Kern und Bild]
  \(\Ker(F) = \{v \in V \ F(v) = 0\}\)\\
  \(\Ima(F) = \{w \in W| \exists v \in V : F(V) = w\}\)\\
  \(\Ker(F)\) ist ein Untervektorraum von V, \(\Ima(F)\) ist ein Untervektorraum
  von W.
\end{definition}

\begin{lemma}
  Falls B eine Basis von V ist, ist \(F(B)\) ein Erzeugendensystem
   von \(Im(F)\).
  F ist injektiv genau dann wenn \(\Ker(F) = \{0\}\).
\end{lemma}

\begin{lemma}
  V endlichdimensional: \(dim(V) = dim(Ker(F)) + dim(Im(F))\).\\
  \(V / Ker(f) \cong \Ima(F)\).
\end{lemma}

\begin{bemerkung}
  V, W endlichdimensional, \(\{v_1, \dots, v_n\}\) Basis von V \(V \cong K^n\),
  \(v_i \mapsto e_i\).
\end{bemerkung}

\begin{definition}[Matrix]
  Sei \(F : V \to W, dim(V) = n, dim(W) = m, \{v_1, \dots, v_n\}\) Basis von V,
  \(\{w_1, \dots, w_n\}\) Basis von W.\\
  \(K^n \cong V \overset{F}{\to} W \cong K^m\). Dadurch wird durch F und die
  beiden Basen eine Abbildung von \(K^n\) nach \(K^m\) definiert. Diese
  Diese Abbildung kann durch eine Matrix A dargestellt werden.\\
  \(\v{\lb_1\\ \vdots \\ \lb_n} \mapsto A \v{\lb_1, \vdots \lb_n}\)\\
  \(F(v_j) = \sum_{i=1}^m a_{ij}w_i\)\\
  \(
  F(v_1), \dots, F(v_n)\\
  \begin{pmatrix}
    a_{11} & \dots  & a_{1m}\\
    \vdots & \ddots & \vdots\\
    a_{n1} & \dots  & a_{nm}
  \end{pmatrix}
  \)
  ist die mxn Matrix A.
\end{definition}

\begin{definition}[Rang einer Matrix]
  \(Rg(A) = \dim(\Span(\text{Spaltenvektoren}))
  = \dim(\Span(\text{Zeilenvektoren}))\)\\
  \(F : V \to W\) linear. \(Rg(F) = Rg(A) = dim(\Ima(F))\), mit A eine beliebige
  darstellende Matrix von F.
\end{definition}

\begin{satz}[Normalform]
  Es seien V, W endlichdimensional. Dann existieren Basen \(\{v_1, \dots v_n\}\)
   von V, \(\{w_1, \dots, w_n\}\) von W, so dass die darstellende Matrix von F
   der Form
  \(
    \begin{pmatrix}
      1      & \dots  & 0      & 0      & \dots  & 0\\
      \vdots & \ddots & \vdots & \vdots & \ddots & \vdots\\
      0      & \dots  & 1      & 0      & \dots  & 0\\
      0      & \dots  & 0      & 0      & \dots  & 0\\
      \vdots & \ddots & \vdots & \vdots & \ddots & \vdots\\
      0      & \dots  & 0      & 0      & \dots  & 0 & 
    \end{pmatrix}
  \) ist.
  \begin{proof}
    Sei \(U = \Ker(F)\) und \(\{v_{r+1}, \dots, v_n\}\) eine Basis von
    U. Sei \(U'\) ein Komplement von U in V \(\Rightarrow V = U \oplus U'\).
    Sei \(\{v_1, \dots, v_r\}\) eine Basis von \(U'\).
    \(B = \{v_1, \dots v_n\}\) ist eine Basis von V. \(\Ima(F)\) hat
    \(\{F(v_1), \dots, F(v_r)\}\) als Basis.\\
    \(
      \sum_{i=1}^n \lb_i F(v_i) = 0
      \Rightarrow F(\sum_{i=1}^n \lb_i v_i) = 0
      \Rightarrow \sum_{i=1}^n \lb_i v_i) \in U
                  \land \sum_{i=1}^n \lb_i v_i) \in U'
      \Rightarrow \sum_{i=1}^n \lb_i v_i) = 0
      \Rightarrow \lb_1 = \dots \lb_n = 0
    \).
    Erg\"anze \(\{F(v_1), \dots, F(v_r)\}\) zu einer Basis
    \(B' = \{w_1, \dots, w_m\}\) von W.
    \(
      F(v_1), \dots, F(v_r), F(v_{r+1}), \dots, F(v_n)\\
      \begin{pmatrix}
        1      & \dots  & 0      & 0      & \dots  & 0\\
        \vdots & \ddots & \vdots & \vdots & \ddots & \vdots\\
        0      & \dots  & 1      & 0      & \dots  & 0\\
        0      & \dots  & 0      & 0      & \dots  & 0\\
        \vdots & \ddots & \vdots & \vdots & \ddots & \vdots\\
        0      & \dots  & 0      & 0      & \dots  & 0 & 
      \end{pmatrix}
    \)
  \end{proof}
\end{satz}

\begin{definition}[Invertierbarkeit von Matrizen]
  \(A \in M_{n\times n}(K)\) ist invertierbar, fall es eine Matrix
  \(B \in M_{n\times n}(K)\) gibt, so dass \(A \cdot B = B \cdot A = Id_n\).
  B wird dann als \(A^{-1}\) bezeichnet.\\
  \(GL(n, k) = Gl_n(K) = \{A \in M_{n\times n}(K) \text{invertierbar}\}\) ist
  eine Gruppe.\\
  \(A \in GL_k(n) \Leftrightarrow \rg(A) = n\) (Eine Matrix ist genau dann
  invertierbar, wenn sie regul\"ar ist).
\end{definition}

\begin{bemerkung}
  Sei A regul\"ar. Dann besitz ein Gleichungssystem der Form
  \(A \v{\lb_1\\ \vdots \\ \lb_n} = \v{b_1 \\ \vdots \\ b_n}\) die Eindeutige
  L\"osung, \(A^{-1} \v{b_1 \\ \vdots \\ b_n}\).
\end{bemerkung}

\begin{bemerkung}
  A ist regul\"ar genau dann wenn A sich durch elementare Zeilenoperationen in
  \(Id_n\) \"uberf\"uhren l\"asst.\\
  \(E_{i,j}\) sei Die Matrix, die an der Stelle ij 1 ist, ansonsten 0.\\
  Elementare Zeilenoperationen sind:\\
  Multiplikation der Zeile i mit \(\lb\): \(\Id_n + (\lb  -1) E_{i,j}\).\\
  Addieren von \(\lb\) mal der iten Zeilten zur jten: \(Id_n + \lb E_{i,j}\).\\
  Vertauschung der i-ten und j-ten Zeile: \(Id_n - E_{ii} - E_{jj} + E_{j,i} + E_{i,j}\)\\
\end{bemerkung}

\begin{bemerkung}
Das inverse einer Matrix l\"asst sich durch nutzen dieser elementaren
Zeilenoperationen nach z.B. dem Gau\ss{}-Jordan Verfahren errechnen:\\
\(
  \left(
  \begin{array}{c|c}
    A & Id_n
  \end{array}
  \right)
  \overset{Zeilenoperationen}{\to}
  \left(
  \begin{array}{c|c}
  Id_n & A^{-1}
    \end{array}
  \right)
\)\\
Die linke H\"alfte der Ergebnis Matrix enth\"alt dann \(A^{-1}\), denn:\\
\(
B_m \dots B_2 B_1 A = Id_n \Rightarrow B_m \dots B_1 = A^{-1}
\)
\end{bemerkung}

% Basiswechselmatrizen?
\begin{definition}[\"Ubergangsmatrizen]
  Es sei \(dim(V) = n\) und \(\{v_1, \dots, v_n\}\),  \(\{v_1', \dots, v_n'\}\)
  Basen von V. Weiterhin sei \(F: V \to V, v_i \mapsto v_i'\). Dann gilt:\\
  \(v_i' = \sum_{ij} s_{ij} v_j\) und die darstellende Matrix S von F, 
  \(
  S = \begin{pmatrix}
    s_{11} & \dots  & s_{1m}\\
    \vdots & \ddots & \vdots\\
    s_{n1} & \dots  & s_{nm}
  \end{pmatrix}
  \)
  ist regul\"ar.
\end{definition}


\begin{definition}
  Zwei (mxn) Matrizen A, A' sind \"aquivalent, falls es reg\"lare matrizen \(T
  \in GL_m(K), s \in GL_n(K)\) gibt, so dass \(A' = T^{-1} \cdot A \cdot S\).\\
  \(A, A' \in M_{n \times n}(K)\)  sind \"ahnlich, fall es \(S \in GL_n(K)\)
  gibt, so dass \(A' = s^{-1} \cdot A \cdot S\).
\end{definition}

\begin{bemerkung}
  \"Ahnlichkeit ist eine \"Aquivalenzrelation auf \(M_{n \times n}(K)\).
\end{bemerkung}


\begin{definition}[Determinante]
  \(det K^n \to K\) ist eine multilineare alternierende  Abildung der Art,
  dass \(det(e_1, \dots, e_n) = 1\).\\
  \(A \in M_{n \times n}(K)\)\\
  \(A = (a_1 | a_2 | \dots | a)n) \Rightarrow det(a_1, a_2, \dots, a_n) = det(A)\).\\
  \(A = (a_ij), det(a_ij) = \sum sign(\pi) \cdot \Pi_{i=1}^n a_{\pi(i)i}\) mit
  \(sign(\pi) = -1^{\text{Anzahl der Fehlst\"ande von }\pi}\) bzw. Anzahl von
  Faktoren von \(\pi\) als Produkt von Transpositionen.
\end{definition}

Eigenschaften von Determinanten:\\
\begin{enumerate}
  \item{\(det(A \cdot B) = \det(A) \det(B)\)\\}
  \item{A ist genau dann invertierbar, wenn \(det(A) \neq 0\)\\}
  \item{\(\det(A^-1) = det(A)^{-1}\)}
  \item{\(\det(A^T) = \det(A)\)}
\end{enumerate}

\begin{bemerkung}
  \(Id_n + (-\Id_n)\) ist nicht invertierbar, also \(\exists A, B : det(A+B) \neq \det(A) + \det(B)\)
\end{bemerkung}

\begin{satz}[Laplacescher entwicklkungssats]
  Sei \(j_0\) ein Spaltenindex\\
  \(det(A) = \sum_{i=1}^n (-1)^{i+j_0} a_{ij_o} det(A_{j_0i})\) wobei
  \(A_{j_{0i}}\) die Matrix ohne Zeile \(j_0\) und Spalte i ist.
\end{satz}


\begin{satz}[Cramersche Regel]
  \((a_1 | \dots | a_n) = A, A \v{\lb_1\\ \vdots \\ \lb_n} = \v{b_1 \\ \vdots \\ b_n}\)
  Falls A regul|'ar ist, gibt es eine einzige L\'Osung zum System:
  \(\lb_j = \frac{det(a_1, \dots, a_{j-1}, b_j, a_{j+1}, \dots, a_n}{det(A)}\)
\end{satz}

\begin{definition}[Determinante eines Homomorphismus]
  Sei \(F : V \to V\). \(det(F) = det(A)\) woei \(A\) eine Darstellungmatrix von F bezgl. einer Baiss \(\{v_1, \dots, v_n\}\).
\end{definition}


\begin{definition}[Adjunte Matrix]
Sei A eine \(n\times n\) Matrix, dann ist die Adjunte von A\\
\(\adj(A) = (\gamma_{ij})\) mit \(\gamma_{ij} = (-1)^{i+ j} \det(A_{ij}\)
\end{definition}

\begin{bemerkung}
  Sei \(c_i\) die j-te Zeile von \(\adj(A)\). Sei weiterhing \(a_i\) die i-te
  Spalte von A.\\
  \(\gamma_{j1}, \dots, \gamma_{jn} \cdot \v{a_{1i}\ \\vdots \\ a_{ni}}
  = \sum_{k=1}^n \gamma_{jk} a_{ki} = \sum_{k=1}^n (-1)^{j+n} a_{ki} \det(A_{jk})
  \overset{\text{Laplacescher Entw. Satz}}{=}
  \det(a_1, \dots, a_{j-1} , a_i, a_{j+1}, \dots, a_n) =
  \begin{cases}
   \det(A) & j=i\\
   0 & j \neq i
  \end{cases}\)\\
  Angenommen A ist regul\"ar.\\
  \(adj(A) \cdot A = det(A) \cdot Id_n
    \Rightarrow \frac{\adj(A)}{\det(A) \cdot A} = \Id_n = A^{-1} \cdot A
    \Rightarrow \frac{\adj(A)}{\det(A)} = A^{-1}
    \Rightarrow A \cdot \adj(A) = det(A) Id_n
  \)
\end{bemerkung}

\section{Diagonalisiserbarkeit}
Sei V ein Vektorraum, \(\{U_i\}_{i=1}^k\) Unterr\"aume von V.\\
\(V = \oplus_{i=1}^k U_i
\Leftrightarrow V = \sum_{i=1}^n U_i \land U_i \bigcap(\sum_{j=1}^k U_i) = 0\)\\
\"Aquivalent dazu ist, dass jeder Vektor \(v \in V\) sich eindeutig als
Linearkombination von Vektoren \(\cup_{j=i}^k B_j\) schreiben l\"asst, woebi
\(B_j\) eine Basis von \(U_i\) ist.

\begin{definition}[Eigenwerte und -vektoren]
  Ein Endomorphismus \(F : V \to V\) besitzt einen Eigenvektor, falls es ein
  \(v \in V \setminus \{0\}\), so dass \(F(V)  \lb \cdot v\) f\"ur ein
  \(\lb \in K\). Falls \(F(v) = \lb v\) ist \(\lambda\) eindeutig bestimmt durch
  F und v. \(\lambda\) ist dann ein Eigenwert von F.
\end{definition}

\begin{definition}[Eigenr\"aume]
  \(\lb \in K, F V \to V\) Endomorphismus.\\
  \(V(\lb) = \{v \in V | F(v) = \lb v\}\), der Eigeneraum zu \(\lb\) is ein UVR.
\end{definition}

\begin{bemerkung}
  \(\lb\) ist ein Eigenwet von F gdw, \(dim(V(\lb)) \ge 1\).
\end{bemerkung}

\begin{bemerkung}
  Falls \(\lb_1, \dots, \lb_k\) verschiedene Eigenwerte von F \(\Rightarrow
  V(\lb_i) \cap \sum_{j=1, j \neq i}^k V(\lb_j) = \{0\}\)
\end{bemerkung}


\begin{definition}[Diagonalisiserbarkeit]
  Sei V ein endlichdimensionaler Vektorraum. \(F : V \to V\) Endomorphismus.
  Bzw. eine Matrix \(A : K^n \to K^n\). F ist diagonalisierbar, falls
  \(V = \oplus_{i=1}^k V('lb), \lb\) verschiedene Eigenwerte von F.\\
  \"Aquivalent dazu, wenn V eine basis von Eigenwerten von F besitzt.
  \"Aquivalent dazu, wenn F bez\"uglich einer Basis von V die Darstellungsmatrix
  \(
    \begin{pmatrix}
      \lb_1  &        & 0\\
             & \ddots &\\
      0      &        & \lb_n
    \end{pmatrix}
  \) hat.\\
  \"Aquivalentz dazu, f\"ur Matrizen: A ist diagonalisierbar gdw.es eine regul\"are
  Matrix S gibt, soda\ss{} \(S^{-1}AS = 
  \begin{pmatrix}
    \lb_1  &        & 0\\
           & \ddots &\\
    0      &        & \lb_n
  \end{pmatrix}\)
\end{definition}

\begin{satz}
  \(A \in M_{n\times n}(K), \lb \in K\)\\
  \(\lb\) ist ein Eigenwert von A gdw. \(\lb Id_n - A\) nicht regul\"ar ist.
  \(\Leftrightarrow det(\lb \cdot Id_n -A) = 0\)
\end{satz}

\begin{definition}[Charakteristisches Polynom]
  Das charakteristische Polynom einer Matrix \(A \in M_{nxn}(K)\) ist 
  \(\xi_{A(T)} = det(T \cdot Id_n - A)\)
\end{definition}

\begin{bemerkung}
  \(\lb\) ist ein eigenwert von A \(\Leftrightarrow \xi_A(\lb) = 0\)
\end{bemerkung}

\textbf{Beispiel}
\(
  \begin{pmatrix}
   0 & -1\\
   -1 & 0
  \end{pmatrix}\\
  \xi_{A(T)} = T^2 + 1 = \det(
  \begin{pmatrix}
    T & -1\\
    1 & T
  \end{pmatrix}
  )
\)

\begin{bemerkung}
  A und \(A'\) \"ahnlich, \(A' = s^{-1}AS \Rightarrow \xi_A(T) = \xi_{A'}(T)\).
  Insebsondere k\"onnen wir \"uber das charakteritische Polynom eines
  Endomorphismus reden.\\
  \(A \in M_{nxn}(K), \xi_A(T) = T^n + b_{n-1} T^{n-1} + \dots + b_o\) wobei
  \(b_0 = (-1)^n det(A), b_n{-1} = - Tr(A) = - \sum_{i=1}^n a_{ii}\)
\end{bemerkung}

\begin{korrolar}
  Ein Endomorphismus \(F : V \to V\) mit \(\dim(V) = n < \infty\) kann
  h\"ochstens n viele Eigenwerte besizten.
\end{korrolar}

\begin{korrolar}
  \(F : V \to V\) mit \(\dim(V) = n < \infty\) mit verschiedenen Eigenwerten
  \((\lb_1, \dots, \lb_n)\) ist diagonalisierbar,  gdw. 
  \(n = \sum_{i=1}^k d_i, d_i = \dim(V(\lb_i))\). \(d_i\) hei\ss{}t geometrische
  Vielfachheit von \(\lb_i\).
  
  \begin{proof} \hfill\break
  \(\Rightarrow\)\\
  F ist diag. gdw. V eine Basis aus Eigenvektoren besitzt, welche aus
  \(\cup_{i=1}^n B_i\) besteht, \(|B_I| = di = dim(V\lb_i)\),
  \(n = |B| = \sum_{i=1}^k |B_i|\)\\
  \(\Leftarrow\)\\
    \(n = \sum d_i \Rightarrow \dim(\sum_{i=1}^k (V(\lb_i))) = n
    \Rightarrow V = \sum_{i=1}^k(V(lb_i)\) da die Eigenr\"aume tranversal sind,
     und ein Vektorraum nur einen UVR der dimension dim(V) hat, sich selbst.
  \end{proof}
\end{korrolar}


% potentially missing part of the beginning of lecture 4
\begin{definition}[Algebraische Vielfachheit]
  Es seien \(F : V \to V\) ein Endomorphismus, \(dim(V) = n < \infty\),
  \(\lb \in K\) Eigenwert \(\Rightarrow \xi_F(\lb) = 0\).\\
  Dann gilt  \(\xi_F(T) = (T-\lb)^K G(T)\), \(G(\lb) \neq 0\). k ist die
  algebraische Vielfachheit von \(\lb\), bzw. \(\ord_\lb(F)\).
\end{definition}

\begin{bemerkung}
  \(\ord_\lb(F) \ge \dim(V(\lb))\)\\
  \begin{proof}
    Sei \({v_1, \dots, v_k}\) eine Basis von \(V(\lb)\). Wir erweitern sie zu
    einer Basis \(\{v_1, \dots, v_k, v_{K+1}, \dots, v_n\}\) von V. Die
    Darstellungsmatrix M von F bzwg. B ist dann\\
    \(\{F(v_1), \dots, F(v_k), F(v_{k+1}), F(v_n)\}\).\\
    \(
      \begin{pmatrix}
        \lb &        & 0         & \\
            & \ddots &           & \\
        0   &        &  \lambda  & C_2 \\
            & 0  
      \end{pmatrix}
    \)\\
    Wobei \(C_2 \in Mat_{n-k \times k}(K)\).\\
    \(
      \xi_F(T) = det(T Id_n - M)
      = (T - \lb)^k \cdot \det(T Id_{n-k} \cdot C_1)\\
      \Rightarrow \ord_\lb(F) \ge K
    \). Wobei \(\det(T Id_{n-k} \cdot C_1) = 0\) sein kann.
  \end{proof}
\end{bemerkung}

\begin{lemma}
  Sei V endlichdimensional, \(F : V \to V\) ein Endomorphismus,
  U ein F-Invarianter Unterraum (\(F(U) \subset U\)).\\
  \(F' : V/U \to V/U\) ist eine lineare Abbildung,
  \(\bar{V} \mapsto \bar{F(V)}\). \(F'\) ist woldefiniert, linear und es gilt
  \(\xi_F(T) = \xi_{F|_U}(T)\cdot\xi_{F'}(T)\)
  \begin{proof}\hfill\break
    \(F'\) ist wohldefiniert;\\
    \(\bar{v_1} = \bar{v} \overset{zZ}{\Rightarrow} F'(v_1) = F(v)\)
    \(
      \bar{v_1} = \bar{v} 
      \Rightarrow v_1 = v + (v_1 - v), v_1 - v \in U\\
      \Rightarrow F(v_1) = F(v) + F(v_1 - v),  F(v_1 - v) \in U
      \Rightarrow \bar{F(v_1)} = \bar{F(v)}
    \)\\
    Restklassen sind linear und F ist linear \(\Rightarrow F'\) ist linear.\\
    Sei \(\{u_1, \dots, u_k\}\) eine Basis von U. erweitert zu
    \(\{u_1, \dots, u_k, v_{k+1}, \dots, v_n\}\) sei sie eine Basis von V.
    \begin{bemerkung}
      \(\{\bar{v_{k+1}}, \dots, \bar{{v_n}}\}\) ist eine Basis von \(V/U\).
      Bew. Einfach.
    \end{bemerkung}
    Darstellungsmatrix H von F bzgl. B:\\
    \(
      \begin{array}{c}
        u_1\\ \vdots \\ u_k \\ v_{k+1} \\ \vdots \\ v_n
      \end{array}
      \begin{pmatrix}
        &           \\
        & A      &        &        &  & C_2\\
        &           \\
        & 0      & \dots  & 0   \\
        & \vdots & \ddots & \vdots &  &C_1\\
        & 0      & \dots  & 0
      \end{pmatrix}
    \) mit \(A, C_1, C_2\) Matrizen.\\
    \(\xi_F(T) = det(T Id_n - H)
    = \det(T Id_n - \begin{pmatrix}A & C_2 \\ 0 & C_1\end{pmatrix})\\
    = \det(\begin{pmatrix}T _id_k - A & -C_2 \\ 0 & T_ Id_{n-k} -C_1\end{pmatrix})
    = \det(T_id_k - A) \cot det (T_E{n-k} - C_1)
    \)\\
     A ist die Darstellungsmatrix von \(F|_U\) bez\"uglich \(\{u_1, \dots, u_k\}\)
     \(\Rightarrow det(T Id_k - A) = \xi_{F|_U}(T)\)\\
     \(C_1\) ist die Darstellungmatrixvon \(F'\) bzg.
     \(\{\bar{v_{k+1}}, \dots, \bar{v_n}\}\).\\
     \(\Rightarrow \det(T \Id_{n-k}-C_1) = \xi_{F'}(T)\)
  \end{proof}
\end{lemma}

\begin{satz}
  Sei \(K\) ein K\"orper, \(dim(V) < \infty, F : V \to V\) ein Endomorphismus so 
  gilt:\\
  F Diagonalisierbar gdw \(\xi_F(T) = (T-\lb_1)^{k_1} \dots (T- \lb_n)^{k_n}\)
  in Linearfaktoren zerf\"allt, wobei f\"ur jeden Faktor \(\lb_1, \dots, \lb_n\)
  \(T-\lb_i\) gilt \(\ord_{lb_i}(F) = \dim(V(\lb_i))\).
  \begin{proof}\hfill\break
  \(\Rightarrow\)\\
    Sei \(b = \{v_1, \dots, v_n\}\) eine Basis von Eigenvektoren. Seien
    \(\lb_1, \dots \lb_r\) die verschiedenen Eigenwerte. Ordne nun B um so dass\\
    \(
      v_1, \dots, v_{d_1} \in V(\lb_i), 
      v_{d_1 + 1}, \dots v_{d_1 + d_2} \in V(\lb_2)
      , \dots,
      v_{d_1  + \dots + d_{r-1}}, \dots v_{d_1 + \dots + d_r} \in V(\lb_r)\) mit
    \(d_i = dim(V(\lb_i))\).\\
    Die Darstellungsmatrix von F bzgl. B:\\
    \((F(v_1), \dots, F(v_{d_1}), \dots F(v_r)\)\\
    \(
      \begin{pmatrix}
        \lb_1 & \\
              & \ddots &  \\
              &        & \lb_1\\
              &        &       & \lb_r & \\
              &        &       &       & \ddots &  \\
              &        &       &       &        & \lb_r
      \end{pmatrix}
      %d_1 viele  \lb_1 auf der Diagonale
    \)\\
    Wobei \(d_i\) viele \(\lb_i\) auf der Diagonale sind\\
    \(\xi_F(T) = det(T \Id_n - A) = (T - \lb_1)^{d_1} \dots T(-\lb_r)^{d_t}\)
    \(((T - \lb_2)^{d_2} \dots T(-\lb_r)^{d_t})(\lb_1) \neq 0
    \Rightarrow d_i = \ord_{\lb_I}(F)
    \), da die \(\lb_i\) verschieden sind.\\
    \(\Leftarrow\)\\
    \(\xi_F(T) = (T - \lb_i)^{d_1} \dots (T - \lb_r)^{d_r}\)\\
    F ist diag \(\Leftrightarrow n = dim(V) = \sum d_i\)
  \end{proof}
\end{satz}


\begin{definition}
  Eine Matrix \(A \in M_{n\times n}(K)\) ist 
  trigonalisierbar, wenn sie \"ahnlich zu einer oberen  Dreiecksmatrix ist:\\
  \(
    \begin{pmatrix}
    a_{11} & \dots  & a_{1n}\\
           & \ddots & \vdots\\
    0      &        & a_{nn}\\
    \end{pmatrix}
  \)
\end{definition}

\begin{satz}
  \(F : V \to V\) ist trigonalisierbar? gdw. \(\xi_F(t)\) in
  Linearfaktoren zerf\"allt \(\xi_F(T) = (T - \lb_1) \dots (T - \lb_n)\).
  
  \begin{proof}
    \(\Rightarrow\)
    F hat Darstellungsmatrix:\\
    \(
    \begin{pmatrix}
      a_11\\
           & a){21} & X\\
                    & \ddots\\
         0                     & a_{nn}\\
         
    \end{pmatrix}\\
    \xi_F(T) = det(T \cdot Id_n - A) = \Pi_{i=1}^n (T - a_{ii})
    \)
    \(\Leftarrow\)
    Induktion \"uber \(n = dim(V)\)\\
    \(n=1 \Rightarrow \) Jede \(1 \times 1\) Matrix ist in oberer Dreiecksform.\\
    \(n \ge 2\)  \(\xi_F(t) = (T - \lb_1) \dots (T_{\lb_n})\). \(\lb_1\) ist ein
    Eigenwert \( \Rightarrow \exists v_1 \in V\{0\} : F(v_10 + \lb_1 v_1\).
    \(U = \Span(v_1) \subset V\)  ist F-invariant. Nach Lemma ... gilt
    \(T - \lb_1) \Pi_{i=2}^n(T - \lb_n) = xi_F(T) = \xi_{F\_U} \cdot \xi_{F'}\), \(\xi_{F|_U} = (T - \lb_1)\)
    \(K[T]\) ist ein Integrit\"atsbereich \(\Rightarrow
    \Xi_{F'}(T) = \pi_{i=2}^n (T-\lb_i), \dim(V/U) < \dim(V)\).\\
    Nach Ia gibt ese eine Basis\(\exists (\bar{v}_2, \dots, \bar{v}_n)\) von
    \(V/U\) derart, dass \(F'\) bzgl. dieser Basis Darstellungsmatrix\\
    \(
    \begin{pmatrix}
      \lb_2 &        & X\\
            & \ddots &\\
      0     &        & \lb_n\\
    \end{pmatrix} = (\mu_{ij})
    \)
    Behauptung: Seien \(v_i \in V, \bar{v_i} = \bar{v}_i, 2 \le i \le n, \{v_1, v_2, \dots, v_n\}\) ist eine
    Basis von V. Bew. \"Ubungsaufgabe.\\
    Frage: Wie sieht die Darstellungmatrix von F bzgl \(\{v_1, \dots v_n\}\)
    aus?\\
    \(\sum_{2\le i \le j} \mu_{ij} \bar{v_i} = F'(v_j) = \bar{F(v_j)}\)\\
    \(\sum_{2\le i \le j} \mu_{ij} \bar{v_i} = \bar{\sum_{2\le i \le j} \mu_{ij} v_i}
    \Rightarrow \exists  \mu_{ij} \in K / F(v_j) = \mu_{ij}v_1 + \sum_{2 \le i \le j} \mu_{ij}v_j = \sum_{1 \le i \le j} \mu_{ij}v_j\)\\
    \(F(v_1) F(v_2), \dots, F(v_n)\\
    \begin{pmatrix}
    \lb_1 & \mu_{12}
          & \lb_2 &       & X
          &       & \ddots
    0     &       &       & \lb_n
    \end{pmatrix}
    \)
  \end{proof}
\end{satz}

\begin{korrolar}
  Jeder Endomorphismus eines endlichdimensionalen Vektorraumes \"uber einen
  algebraisch abgeschlossenen K\"orper (z.B. \(\C\)) ist trigonalisierbar.
\end{korrolar}

\begin{lemma}
  Sei V endlichdimensional \'uber ienen K\"orper K.\\
  \(v \in V \setminus \{0\} \exists r \in \N : F^r(v) = sum_{i=0}^{r-1}a_i F^i(v)\) \(a_i\) ist eindeutig bestimmt.
  Insb. ist \(U = Span(v, F(v)m \dots, F^{r-1}(v))\) ist F inviariant, hat Basis \(v, F(V), \dots, F^{r-1}(v))\)\\
  \(F \uparrow U\) hat Darstellungsmatrix
  \(
  \begin{pmatrix}
  0     ... a_0
  1 0   ... \vdots
    1 0     a_n
  \end{pmatrix}
  \)
  \(\xi_{F \uparrow U}(T) = T^r - a_{r-1} T^{r-1} - \dots - a_0\).
  
  \begin{proof}
    \(n = \dim(V)\).
    \(\{v, F(v), \dots, F^n(v)\}\) sind lin.abh. Sei r die kleinste nat\'Urlich Zahl,
    so dass \(\{v, F(v), \dots, F^r(v)\}\) lin. abh sind. Dann sind
    \(\{v, F(v), \dots, F^{r-1}(v)\}\) linear unabh\"angig.
    \(\overset{\text{Ausstauschprinzip von r}}{\Rightarrow} F^r(v) = \sum_{i=0}^{r-1} a_i F^i(v)\)
    wobei \(a_i\) eindeutig bestimmt sind.\\
    \(U = \Span(v_1, \dots, F^{r-1}(v))\) ist F-invariant. Sei \(u \in U, u = \sum_{0}^{r-1} \mu_iF^i(v)\\
    F(u) = \sum_{i=0}^{r-1} \mu_i F^{i+1}(v)\\
    = \sum_{i=0}^{r-2} \mu_iF({i+1}(v) + \mu_{i-1}F^r(v)\)
    Wobei \(F^r(v) = \sum_{i=0}^{r-1} a_iF^i(v)\).\\
    Beide teile der Summe \(\sum_{i=0}^{r-2} \mu_iF({i+1}(v) + \mu_{i-1}F^r(v)\) liegen eindeutig in U, also liegt auch die Summe in U.\\
    \(\{v_1, \dots, F^{r-1}(v)\}\) ist eine Basis von U.\\
    \(F(v_1), \dots F(F^{3-1}(v))\\
    \begin{pmatrix}
      0 & 0 & \dots & a_0
      1 & 0 &       & \vdots
      0 & 1 &       & a_{r-1}
    \end{pmatrix}\)\\
    \(\xi_{F|U}(T) = det(T Id_r - A)
    = det(    \begin{pmatrix}
      T & 0 & \dots & -a_0
      -1 & T &       & -\vdots
      0 & -1 &       & -a_{r-1}
    \end{pmatrix}\)\\
    Laplacsche Entwicklung nach der letzten Spalte:
    \(= (-1)^{r+1}(-a_0) det(\ve{-1 & T\\0 -1 & \ddots} + (-1)^{r+2}(-a_1)\det(\ve{& T\\0 -1 & \ddots}) + \dots + (-1)^{2r} (T - a_r) \det(\ve{T\\-1 & T\\  & -1 T})\)
    \(= (-1)^r a_0 (-1)^r + (-1)^{r+1} a_1T (-1)^{r+1} + \dots + (-1)^{2r}(T - a_1)T^{r-1}
    = - a_0 - a_1 T - \dots - a_{r-1}T^{r-1} + T^r\)
  \end{proof}
\end{lemma}

Notation:\\
\(P(T) \in K[T], P(T) + \sum_{i=0}^m a_iT^i\), \(F : V \to V\) Endomorphismus.\\
\(P(f) : V \to V, v \mapsto \sum_{i=0}^m a_iF^i(v)\)\\
Mit dieser Notation haben wir, dass im vorigen Lemma
\(\xi_{F|_u}(F) = F^r(v) - \sum_{i=0}^{r-1} a_i F^i(v) = 0\)

\begin{satz}[Caloy - Hamilton]
  \(F : v \to V\) Homomorphismus\\
  \(\xi_F(F)\) ist der 0 Endomorphismus auf V.
  \begin{proof}
    z.Z. ist dass \(\forall v \in V : \xi_F(F)(v) = 0\)\\
    \(v = 0 \Rightarrow \xi_F(F)(v) = 0\)\\
    Ansosnten \(v \neq 0\):\\
    \(\exists r : U =  = \Span(v, F(v), \dots, F^{r-1}(v))\) ist F-invariant.\\
    U - F-invariant \(\Rightarrow \xi_f = \xi_{F|_U} \cdot \xi_{F'} = \xi_{F'} \cdot \xi_{F|_U}\), \(F' : v/U \to V/U, \bar{w} \mapsto \bar{F(w)}\).\\
    Aufgabe:\\
    \(R(T) = P \cdot Q\\ % als Polynome
    R(F) = P \circ (Q(F))\) % als Abbildungen\\
    \(\xi_F(F)(v) = \xi_{F'} \circ (\xi_{F|_U}(v))\), wobei \(\xi_{F|_U}(v) = 0\), also \(\xi_F{F}(v) = 0\).
  \end{proof}
\end{satz}

\begin{korrolar}
  \(A \in M_{n \times n}(K)\\
  \xi_A(T) = T^n + \sum_{i=0}^{n-1}b_i T^i \Rightarrow A^n + \sum_{i=0}^{n-1} b_i \cdot A^i = 0\)
\end{korrolar}

\begin{satz}
  V endlichdimensionaler Vr, \(F V \to V\) Endormophismus, Dann existiert genau ein normiertes
  Plynom kleinsten Grades,  \(m_f\), derart, dass \(\forall P \in K[T] : m_F |_P \Leftrightarrow P(\tau) = 0\)
  Insbesondere gilt \(m_F(F) = 0\). Das  Polynom \(m_F\) hei\ss{}t das Minimalpolynom von F. 
\end{satz}

% missing stuff

\begin{satz}
  \(F : V \to V\) Endormorphismus, Dann existiert genau ein normiertes Polynom
  derart, dass \(\forall P \in K[T]\)  \(m_F|_P \Leftrightarrow P(F) = 0\).
  Das Polynom \(m_F(T)\) hei\ss{}t das Minimalpolynom von F.
  \begin{proof}
    Sei \(\mathcal{F} = \{P[T] \ text{normmiert} | P(F) = 0 \text{als Endomorphis.}\).\\
    Caley - Hamilton:
    \(Xi_F(T) \in \mathcal{F} \neq \emptyset\) Sei \(m-F(T) \in \mathcal{F}\)
    Polynom kleinsten Grades.\\
    Zu zeigen: \(\forall P \in K[T] | m_F|_P |leftrightarrow P(F) = 0\)\\
    \(\Rightarrow\)\\
    \(m_F |_P \Rightarrow \exists Q \in K[T] : P = Q \cdot m_F\)\\
    \(P(F) = Q(F) \circ m_F(F), m_F(F) = 0\\
     P(F) = 0\)\\
    \(\Leftarrow\)\\
    Sei \(P \in K[T], P(F) = 0\). Division mit Rest
    \(\Rightarrow \exists q, r \in K[T] : P = Q m_F + r, \Grad(r) < \Grad(m_F)\)
    \(0 = P(F) = Q(F) \circ m_F(F) + r(F) = r(F) = 0\)\\
    \(\Rightarrow r(F) = 0\) als Endomorphismus, \(\Rightarrow r = 0\).
    (sonst \(\frac{1}{a_{\Grad(T)}} \cdot r(T) \in \mathcal{F}\)\\
    Eindeutigkeit:\\
    Angenommen \((m_F'\) w\"urde auch die Bedingungen erf\"ullen\\
    \(m_F'|_P \Leftrightarrow P(F) = 0 \forall P \in K[T]\)\\
    \(\Leftrightarrow M_F|_{m_F'} \land m_F'|_{m_F}\)\\
    \(m_F = Q m_F' \land m_F' = H m_F\)\\
    zu Zeigen \(Q = H = 1\). Sowohl \(m_F, m_F'\) beide normiert, \(\Rightarrow
    Q, H\) sind normiert.\\
    \(m_F = Q \cdot m_F' = Q H m_F \land K[T] \) Integrit\"atsbereich\\
    \(\Rightarrow 1 = Q H\)\\
    \(\Grad(G H) = \Grad(G) \Grad(H) = \Grad(1) = 0 \land G, H \text{ normiert}\\
    \Rightarrow G = H = 1 \Rightarrow m_F = m_F'\).
  \end{proof}
\end{satz}

Frage\\
\(A \in \ve{0 & 1 \\ 0 & 0}
m|_A = \\
\xi_A(T) = T^2
m_A|_{T^2} \Rightarrow m_A = \begin{cases} T \\ T^2 \end{cases}
m_A(A) = 0 \Rightarrow m_A \neq T \Rightarrow m_A = T^2
\)

\begin{lemma}
  gegeben \(F : V \to V\), V endlich dimensional, F Endormorphismus, dann haben
  \(\xi_F\) und \(m_F\) dieselben Nullstellen in K.
  \begin{proof}
    \(m_F|_{\xi_F} \Rightarrow \xi_F = Q m_F \Rightarrow \forall \lb \in K\),
    falls \(m_F(\lb) = 0 \Rightarrow \xi_F(\lb) = 0\)\\
    Sei \(\lb \in K " \xi_F(\lb) = 0 \Rightarrow \lb \) ist ein Eigenwert von F
    \(\Rightarrow \exists v \in V \setminus \{0\} : F(v) = \lb v\). Sei \(m_F(T)
    = T^d + \sum_{i=0}^{d-1} c_i T^i\). \(0 = m_F(F)(v) = (F^d + \sum c_iF^I)(v)
    = F^d(v) + \sum c_i F^i(v)
    = \lb^d v + \sum c_i \lb^i v
    = (\lb^d + \sum c_i \lb^i) v
    = m_F(\lb) v
    \Rightarrow m_F(\lb) = 0\), da \(v \neq 0\).
  \end{proof}
\end{lemma}

\begin{satz}
  \(F : V \to V\), \(V\) endlichdimensional, F endomorphismus ist
  diagonalisierbar gw. \(m|_F\) in lauter paarweise veschiedene Linearfaktoren
  zerf\"allt.
  \begin{proof}
    Kommt in einer sp\"ateren Vorlesung.
  \end{proof}
\end{satz}

\begin{bemerkung}
  \(A \in Mat_{n \times n}(K) : A^2 = Id_n \Rightarrow A\) ist diagonalisierbar.\\
  \(A^2 = Id_n \Rightarrow (T^ - 1)(A) = 0 \Rightarrow m_F|_{T^2 - 1}
  m_A = \begin{cases}T^2 - 1 = (T - 1)(T + 1) \\ T - 1 \\ T + 1\end{cases}
  \)
\end{bemerkung}

\section*{Jordansche Normalform}

\begin{lemma}
  \(F : V \to V\) Endomorphismus, \(U \subset V\), dann ist U F-invariant gdw.
  \(U (F - \lb Id\)-invariant ist \(\forall \lb \in K\).
  \begin{proof}
    \"Ubungsblatt 4
  \end{proof}
\end{lemma}

\textbf{Beispiel}\\
\(F : V \to V, F \neq 0\) Endomorphismus, derart, dass \(F^m = 0\) als
Endomorphismus und \(m > 0\) minimal (F ist nilpotent). \(F^{m-1} \neq 0
\Rightarrow v \in V : F^{m-1}(v) \neq 0\). \(\{v, F(v), \dots, F^{m-1}(v)\}\)
ist linear unabh\"angig. Erg\"anze sie zu einer Basis B von V.
\(B = \{v, F(v), \dots, F^{m-1}(v), \dotsm v_n\).\\
\(F^{m-1}(v), \dots F(v), v, \dots v_n\)\\
\(
  \begin{pmatrix}
    0      & 1      & 0      & \\
    \vdots & 0      & \ddots & *\\
    0      & \vdots & 0      & \\
    0      & \dots  &        & *
  \end{pmatrix}
\)

\begin{definition}[Hauptraum]
  Sei \(\lb \in K\) ein Eigenwert von \(F : V \to V\), V endlichdimens..\\
  \(V(\lb) = \Ker(F - \lb)\) Eigenraum von F bzgl \(\lb\).\\
  \(\Ker(F - \lb) \subset \ker(F - \lb)^2 \subset \dots\)\\
  \(\cup_{n \in \N} \ker(F - \lb)^n\) ist der Hauptraum von F bzgl. \(\lb\).
  \begin{bemerkung}
    \(V_\lb\) ist ein Unterraum und \(V_\lb\) ist F-invariant,
    da \(V_\lb\) \((F - \lb)\)-invariant ist.
  \end{bemerkung}
  \begin{bemerkung}
    Falls \(\Ker(F - \lb) = V(\lb) = 0\) ist
    \(\cup_{n \in \N}\Ker(F - \lb)^n = 0\).
  \end{bemerkung}
\end{definition}

\begin{lemma}
  Seien \(\lb_1 \dots \lb_k\) verschiedene Elemente aus K.
  \(V_{\lb_i} \cap \sum_{j=1 j\neq i}^k V_{\lb_j} = \{0\}\).
  \begin{proof}
    \(V_{\lb_j} = (F - \lb_j)\) invariant \(\Rightarrow V_{\lb_j}\) ist
    F-invariant \(\Rightarrow (F-\lb_i\) invariant.\\
    Wir wollen zuerst zeigen, dass  \(F - \lb_i \uparrow V_{\lb_j}\) ein
    Automorphismus ist falls \(i \neq j\).\\
    \(\Rightarrow\) es gen\"ugt zu zeigen, dass \(F - \lb_i\) injektiv auf
    \(V_{\lb_j}\) ist.\\
    Sei \(w \in V_{\lb_j} \setminus \{0\} \Rightarrow m \in \N\) kleinstm\"oglich,
    \(F - \lb_J)^m(w) = 0 \Rightarrow (F - \lb_j)^{m - 1}(w) \neq 0
    \Rightarrow ((F - \lb_j)^{m-1}((\lb_i - \lb_j)(w)) \neq 0\)\\
    Sei \(0 \neq (F - \lb)j)^{m-1}(F(w) - \lb_j w - (F(w) - \lb_i w))
    \Rightarrow (F - \lb_i)^m(w) + (F - \lb_j)^{m-1} \circ (F - \lb_i)(w)
    \Rightarrow (F- \lb_i)(w) \neq 0\).\\
    Insebsondere is jede Potez \(F - \lb_i^k\) ein Automorphismus von
    \(V_{\lb_j}\). \(V_{\lb_i} \cap \sum_{j=1, j\neq i}^k V_{\lb_j} = \{0\}\)
    \(v \in V_{\lb_i} \cap \sum_{j \neq i} V_{\lb_j}\\
    v = \sum_{j \neq i} v_j \in V_{\lb_j}
    \Rightarrow exist sm_j\) kleinstes \(F( - \lb_J)^{m_j}(v_j) = 0\)\\
    \((F - \lb_i)^{m_i} \circ \dots \circ (F - \lb_{i - 1}^{m}{i-1} \circ
    (F - \lb_{i-1}^{m_{i+1}} \dots (F - \lb_{k})^{m_k}\) ist eine Automorphismus
    von \(V_{\lb_i}\). Sie dieser automorph. H. \(H(v) = \sum_{j \neq i}(H(v_j)
    = \sum_{j \neq i} (F - \lb1)^{m_1} \circ \dots \circ (F - \lb_j)^{m_j}(v_j), F - \lb_j)^{m_j}(v_j) = 0
    \Rightarrow H(v) = 0 \Rightarrow v = 0\).
    \end{proof}
\end{lemma}




\newpage
\renewcommand{\listtheoremname}{Satz- und Definitionsverzeichnis}
% \listoftheorems[ignoreall, show={definition}, show={satz}, show={lemma}, show={definitionn}, show={korrolar}, show={altdefinition}]
\addcontentsline{toc}{chapter}{Satz- und Definitionsverzeichnis}
\newpage
\printindex
\end{document}
