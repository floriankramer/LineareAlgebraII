\documentclass{report}

\usepackage{hyperref}
\usepackage{amsfonts}
\usepackage{amssymb}
\usepackage{amsmath}
\usepackage{amsthm}
\usepackage[utf8]{inputenc}
\usepackage[german]{babel}
\usepackage{tikz}
\usepackage{thmtools}
\usepackage{titlesec}
\usepackage{enumitem}
\usepackage{pgfplots}
\usepackage{relsize}
\usepackage{imakeidx}
\usepackage{framed}
\usepackage{etoolbox}
\usepackage{float}

% configure tikz circles & nodes
\tikzset{
  node style sp/.style={draw,circle,minimum size=1cm},
  node style ge/.style={circle,minimum size=1cm},
  arrow style mul/.style={draw,sloped,midway,fill=white},
  arrow style plus/.style={midway,sloped,fill=white},
}
% tikz dependencies:
\usetikzlibrary{matrix,arrows,decorations.pathmorphing, decorations.markings, positioning, tikzmark}

% page size & margins:
\usepackage[a4paper, top = 2cm, left = 2.25cm, right = 2.25cm, bottom = 2cm]{geometry}

% defines index commands with hyperref
\makeindex[columns=3, title=Stichwortverzeichnis, intoc=true,options={-s index_style.ist}]
\newcommand{\BH}[1]{\textbf{\hyperpage{#1}}}
\newcommand{\IN}[1]{\index{#1|BH}}

\title{Lineare Algebra II\\Mitschrieb}

% title formatting of sections
\titleformat{\chapter}[block]
{\normalfont\huge\bfseries}{\Huge \thechapter. }{0em}{\Huge}
\titlespacing*{\chapter}{0pt}{-15pt}{20pt}
\titlespacing*{\section}{0pt}{0pt}{10pt}
\titlespacing*{\subsection}{0pt}{0pt}{10pt}

% ease of use commands
\newcommand{\lb}{\lambda}
\newcommand{\al}{\alpha}
\newcommand{\be}{\beta}

\newcommand{\mlb}{\(\lb\)}
\newcommand{\ii}{\mathrm{i}}
\newcommand{\ee}{\mathrm{e}}
\newcommand{\R}{\mathbb{R}}
\newcommand{\N}{\mathbb{N}}
\newcommand{\Z}{\mathbb{Z}}
\newcommand{\Q}{\mathbb{Q}}
\newcommand{\C}{\mathbb{C}}
\newcommand{\mR}{\(\mathbb{R}\)}
\newcommand{\mN}{\(\mathbb{N}\)}
\newcommand{\mZ}{\(\mathbb{Z}\)}
\newcommand{\mQ}{\(\mathbb{Q}\)}
\newcommand{\mC}{\(\mathbb{C}\)}
\newcommand{\Rn}{\mathbb{R}^n}
\newcommand{\mRn}{\(\mathbb{R}^n\)}
\newcommand{\ve}[1]{{\begin{pmatrix}#1 \end{pmatrix}}}
\renewcommand{\v}{\ve}
\newcommand{\baseb}{\mathcal{B}}
\newcommand{\basea}{\mathcal{A}}
\newcommand{\En}{\mathrm{E}_n}

\DeclareMathOperator{\abb}{Abb}
\DeclareMathOperator{\Span}{span}
\DeclareMathOperator{\Hom}{Hom}
\DeclareMathOperator{\End}{End}
\DeclareMathOperator{\Aut}{Aut}
\DeclareMathOperator{\Ima}{Im}
\DeclareMathOperator{\Ker}{Ker}
\DeclareMathOperator{\rg}{rg}
\DeclareMathOperator{\Mat}{Mat}
\DeclareMathOperator{\Id}{Id}
\DeclareMathOperator{\GL}{GL}
\DeclareMathOperator{\M}{M}
\DeclareMathOperator{\Sym}{Sym}
\DeclareMathOperator{\sign}{sign}
\DeclareMathOperator{\SL}{SL}
\DeclareMathOperator{\adj}{adj}
\DeclareMathOperator{\vol}{vol}
\DeclareMathOperator{\PE}{PE}
\DeclareMathOperator{\MM}{M}
\DeclareMathOperator{\TR}{Tr}
\DeclareMathOperator{\Grad}{Grad}
\DeclareMathOperator{\Char}{Char}
\DeclareMathOperator{\ord}{ord}


% increase line height
\renewcommand{\baselinestretch}{1.3}

% redefines description label for alternative enumeration in description environement
\renewcommand*\descriptionlabel[1]{\hspace\labelsep\emph{#1}}

% define main environment used for lemmas, definitions, corollaries, ...
\newenvironment{beispiel} {
\textbf{Beispiel:}\hfill\break
}{}

\makeatletter
\newtheoremstyle{customdef} % name of the style to be used
{12pt}        % measure of space to leave above the theorem. E.g.: 3pt
{12pt}        % measure of space to leave below the theorem. E.g.: 3pt
{\normalfont\addtolength{\@totalleftmargin}{7pt}\addtolength{\linewidth}{-7pt}\parshape 1 7pt\linewidth} % name of font to use in the body of the theorem
{-7pt}        % measure of space to indent
{\bfseries}   % name of head font
{ \\[.125cm] }% punctuation between head and body
{ }           % space after theorem head; " " = normal interword space
{\thmname{#1}\thmnumber{ #2} {\normalfont\thmnote{ -- \hspace{1pt}  \defemph{#3}}}}

\newtheoremstyle{customenv} % name of the style to be used
{12pt}        % measure of space to leave above the theorem. E.g.: 3pt
{12pt}        % measure of space to leave below the theorem. E.g.: 3pt
{\normalfont\addtolength{\@totalleftmargin}{7pt}\addtolength{\linewidth}{-7pt}\parshape 1 7pt\linewidth} % name of font to use in the body of the theorem
{-7pt}        % measure of space to indent
{\bfseries}   % name of head font
{ \\[.125cm] }% punctuation between head and body
{ }           % space after theorem head; " " = normal interword space
{#3}
\makeatother

% define side environment used for remarks
\newtheoremstyle{customrem} % name of the style to be used
{0pt}         % measure of space to leave above the theorem. E.g.: 3pt
{0pt}         % measure of space to leave below the theorem. E.g.: 3pt
{}            % name of font to use in the body of the theorem
{}            % measure of space to indent
{\itshape}    % name of head font
{}            % punctuation between head and body
{.5em}        % space after theorem head; " " = normal interword space
{}

% define remarks:
\theoremstyle{customrem}
\newtheorem*{bemerkung}{Bemerkung\textnormal:}
\newtheorem*{bemerkung2}{Bemerkungen\textnormal:}
\newenvironment{bemerkungen}[1][]{\begin{bemerkung2}[#1]\leavevmode}{\end{bemerkung2}}

% define definition environment names
\theoremstyle{customdef}
\newtheorem{definition}{Definition}[chapter]
\newtheorem*{definitionn}{Definition} % without numbering
\newtheorem{altdefinition}[definition]{Alternative Definition}
\newtheorem{proposition}[definition]{Proposition}
\newtheorem{lemma}[definition]{Lemma}
\newtheorem{korrolar}[definition]{Korollar}
\newtheorem{satz}[definition]{Satz}
\newtheorem*{satz*}{Satz} % without numbering
\renewenvironment{proof}{\paragraph{Beweis: }}{\qed}
\theoremstyle{customenv}
\newtheorem*{customenv}{customenv} % weird

% define definition emphasis
\newcommand{\defemph}[1]{\textsc{#1}}

% disables autoindent
\setlength{\parindent}{0em}

% defines lists of theorems:

\let\Chaptermark\chaptermark
\def\chaptermark#1{\def\Chaptername{#1}\Chaptermark{#1}}
\makeatletter
% \AtBeginDocument{\addtocontents{loe}{\protect\thmlopatch@chapter{0 - Lineare Gleichungssysteme und der n-dimensionale reelle Raum}}} % fixes chapter 0 not showing by patching this shit to the beginning of the .loe
\patchcmd\thmtlo@chaptervspacehack
{\addtocontents{loe}{\protect\addvspace{10\p@}}}
{\addtocontents{loe}{\protect\thmlopatch@endchapter\protect\thmlopatch@chapter{\thechapter\space-\space\Chaptername}}}
{}{}
\AtEndDocument{\addtocontents{loe}{\protect\thmlopatch@endchapter}}
\long\def\thmlopatch@chapter#1#2\thmlopatch@endchapter{%
  \setbox\z@=\vbox{#2}%
  \ifdim\ht\z@>\z@
  \addvspace{10\p@}
  \hbox{\bfseries\chaptername\ #1}\nobreak
  #2
  \addvspace{10\p@}
  \fi
}
\def\thmlopatch@endchapter{}

\def\ll@definition{%
  \protect\thmtopatch@numbernametext
  \ifx\@empty\thmt@shortoptarg\else[\thmt@shortoptarg]\fi
  {\csname the\thmt@envname\endcsname}%
  {\thmt@thmname}%
}

\def\ll@definitionn{%
  \protect\thmtopatch@numbernametext
  \ifx\@empty\thmt@shortoptarg\else[\thmt@shortoptarg]\fi
  {\csname the\thmt@envname\endcsname}%
  {\thmt@thmname}%
}

\def\ll@altdefinition{%
  \protect\thmtopatch@numbernametext
  \ifx\@empty\thmt@shortoptarg\else[\thmt@shortoptarg]\fi
  {\csname the\thmt@envname\endcsname}%
  {\thmt@thmname}%
}

\def\ll@satz{%
  \protect\thmtopatch@numbernametext
  \ifx\@empty\thmt@shortoptarg\else[\thmt@shortoptarg]\fi
  {\csname the\thmt@envname\endcsname}%
  {\thmt@thmname}%
}

\def\ll@lemma{%
  \protect\thmtopatch@numbernametext
  \ifx\@empty\thmt@shortoptarg\else[\thmt@shortoptarg]\fi
  {\csname the\thmt@envname\endcsname}%
  {\thmt@thmname}%
}

\def\ll@korrolar{%
  \protect\thmtopatch@numbernametext
  \ifx\@empty\thmt@shortoptarg\else[\thmt@shortoptarg]\fi
  {\csname the\thmt@envname\endcsname}%
  {\thmt@thmname}%
}

\def\ll@customenv{%
  \protect\thmtopatch@numbernametext
  \ifx\@empty\thmt@shortoptarg\else[\thmt@shortoptarg]\fi
  {\csname the\thmt@envname\endcsname}%
  {\thmt@thmname}%
}

\newcommand\thmtopatch@numbernametext[3][]{%
  #3 #2%
  \if\relax\detokenize{#1}\relax\else:\space\space #1\fi
}

\makeatother


% replace ugly hyperref boxes with colored text
\usepackage{xcolor}
\hypersetup{
  colorlinks,
  linkcolor = {red!50!black},
  urlcolor  = {blue!80!black}
}
\setcounter{chapter}{-1}

% define coefficient matrix environment
\makeatletter
\renewcommand*\env@matrix[1][*\c@MaxMatrixCols c]{%
  \hskip -\arraycolsep
  \let\@ifnextchar\new@ifnextchar
  \array{#1}}
\makeatother

\makeatletter
\newsavebox\myboxA
\newsavebox\myboxB
\newlength\mylenA

% define new overline command
\newcommand*\xoverline[2][.8]{%
  \sbox{\myboxA}{$\m@th#2$}%
  \setbox\myboxB\null% Phantom box
  \ht\myboxB=\ht\myboxA%
  \dp\myboxB=\dp\myboxA%
  \wd\myboxB=#1\wd\myboxA% Scale phantom
  \sbox\myboxB{$\m@th\overline{\copy\myboxB}$}%  Overlined phantom
  \setlength\mylenA{\the\wd\myboxA}%   calc width diff
  \addtolength\mylenA{-\the\wd\myboxB}%
  \ifdim\wd\myboxB<\wd\myboxA%
  \rlap{\hskip 0.5\mylenA\usebox\myboxB}{\usebox\myboxA}%
  \else
  \hskip -0.5\mylenA\rlap{\usebox\myboxA}{\hskip 0.5\mylenA\usebox\myboxB}%
  \fi}
\makeatother

% defines bigslant for quotient (and bigbigslant for single nested quotients)
\newcommand{\bigslant}[2]{{\raisebox{.1em}{\(#1\)}\hspace{-.2em}\left/\raisebox{-.1em}{\(#2\)}\right.}}
\newcommand{\bigbigslant}[2]{{\raisebox{.4em}{\(#1\)}\hspace{-.2em}\left/\raisebox{-.4em}{\(#2\)}\right.}}

\begin{document}
\begin{titlepage}
  \newcommand{\HRule}{\rule{\linewidth}{0.5mm}}
  \centering
  \vspace{6cm}
  \textsc{\large \thinspace}\\[0.5cm]
  \vspace{4cm}
  \HRule \\[0.8cm]
  { \Huge  \textbf{Lineare Algebra \(\mathbf{II}\)}}\\[0.4cm]
  \HRule \\[.5cm]
  {\Large Inoffizieller Mitschrieb}\\[1.0cm]
  Stand: \today
  \\[11.5cm]
  \begin{minipage}{0.65\textwidth}
    \begin{center} \large
      \textsl{Vorlesung gehalten von:}\\[1cm]
      Prof. Dr. Amador Martín-Pizarro\\
      Abteilung für Angewandte Mathematik\\
      \textsc{\large Albert-Ludwigs-Universität Freiburg}
    \end{center}
  \end{minipage}\\[2.5cm]
  \thispagestyle{empty}
\end{titlepage}


\chapter{Recap}

\begin{definition}[Ring]
  Ein (kommutativer) Ring (mit Einselement) ist eine Menge zusammen mit zwei
  bin\"aren Operationen \(+, \cdot\), derart, dass:\\
  \begin{itemize}
    \item{\((R, +)\) ist eine abelsche Gruppe}
    \item{\((R, \cdot)\) ist eine kommutative Halbgruppe}
    \item{die Dsitributivgesetze:\\
    \(a(x+y) = ax + ay\)\\
    \((x + y) z = xz + yz\))}
  \end{itemize}
\end{definition}

\begin{definition}[Integrit\"atsbereich]
  Ein Integrit\"atsbereich ist ein Ring ohne Nullteiler. Also
  \(\forall x, y \in R : x \cdot y = 0 \Rightarrow x = 0 \lor y = 0\)
\end{definition}

\begin{definition}[K\"orper]
  Ein K\"orper ist ein Ring der Art, dass
  \begin{enumerate}
    \item{\(1 \neq 0\)}
    \item{
      \(
        \forall x \in K : x \neq 0
        \Rightarrow \exists x^{-1} : xx^{-1} = x^{-1}x = 1
      \)
    }
  \end{enumerate}
\end{definition}

\begin{bemerkung}
  K\"orper sind Integrit\"atsbereiche.
\end{bemerkung}

\begin{definition}[Charakteristik]
  Sei R ein nicht trivialer Ring \((0 \neq 1)\).
  \(
    \varphi : \Z \rightarrow R, z \mapsto
    \begin{cases}
      \sum_{i=1}^n 1 & n >= 0\\
      -\sum_{i=1}^n 1 & \text{ansonsten}
    \end{cases}
  \)\\
  Dann ist \(\varphi\) ein Ringhomomorphismus.\\
  F\"ur den Kern von \(\varphi\) (\(\Ker(\varphi)\)) gibt es zwei
  M\"oglichkeiten.
  \begin{enumerate}
    \item{\(\Ker(\varphi) = \{0\}, p = 0\)}
    \item{
      \(\Ker(\varphi) \neq \{0\}\). Dann gibt es ein kleinstes echt positives
      Element \(p \in \Ker(\varphi)\).
    }
  \end{enumerate}
  R hat dann Charakteristik p \((\Char(R) = p)\). Falls R ein
  Integritaetsbereich ist, dann ist p eine Primzahl.
  \textbf{Beispiele:}\\
  \(\Z / n\Z = \{\bar{0}, \dots, \bar{n}\}\) hat Charakteristik n.\\
  Insbesondere enth\"alt jeder K\"orper mit Charakteristik p eine ''Kopie'' von
  \(\Z / m\Z\):\\
  k hat Charakteristik p \(\Rightarrow \Z /p\Z
  \overset{injectiv}{\leftrightarrow} K\).\\
  Hier ist \(\Z/p\Z\) ein K\"orper:\\
  \(a \in \Z/p\Z \setminus \{0\} \Rightarrow\) es ist a mit p teilerfremd.
  \(1 = a \cdot b + p \cdot m \Rightarrow \bar{1} = \bar{a} \cdot \bar{b}\).
\end{definition}

\begin{definition}[Polynomring]
  Sei K ein K\"orper. Der Polynomring \(K[T]\) in einer Variable R \"uber K ist
  die Menge formeller Summen der Form:\\
  \(f = \sum_{i = 0}^n a_i \cdot T^i, n \in \N\)\\
  Der Grad von \(f \in K[T]\) ist definiert als:\\
  \(\Grad(f) := max(m | m < n \land a_m \neq 0)\)\\
  \(\Grad(0) := -1\)\\
  Falls \(Grad(f) = n\) und \(n = 1\) hei\ss{}t das Polynom normiert.\\
  Die Summe und das Produkt von Polynomen sind definiert als:\\
  \(
    \sum_{i=0}^n a_i T^i + \sum_{j=0}^m b_j T^j 
    := \sum_{k=0}^{max(m,n)} (a_k b_k) T^k
  \)\\
  \(
    \sum_{i=0}^n a_i T^i \cdot \sum_{j=0}^m b_j T^j 
    := \sum_{k=0}^{m+j} = c_K T^k, c_k = \sum_i+j=k a_i b_j
  \)\\
\end{definition}

\begin{bemerkung}
  \(K[T]\) ist ein Integrit\"atsbereich.\\
\end{bemerkung}

\begin{korrolar}
  Es seien \(f, g\) beide \(\neq 0\)\\
  \(
    \Rightarrow \Grad(f \cdot g)
    = \Grad(f) + \Grad(g) \Rightarrow f \cdot g \neq 0
  \)\\
  \(\Grad(f+g) \le max(\Grad(f), \Grad(g))\)\\
\end{korrolar}

\begin{satz}[Division mit Rest]
  Gegeben \(f, g \in K[T], \Grad(g) > 0\). Dann existieren eindeutige 
  Polynome q, r, so dass \(f = g \dot q + r\), wobei \(\Grad(r) < \Grad(g)\).
  \begin{proof}
    Eindeutigkeit: Angenommen
    \(f = g \cdot q + r = g \cdot q' + r', q \neq q' \lor r \neq r'\).\\
    \(
      \Rightarrow g(q - q') = r' - r
      \Rightarrow \Grad(r' - r) = max(\Grad(r'), \Grad(r)) < \Grad(g)
                  = \Grad(g(q - q'))
      \Rightarrow\) Widerspruch \(
      \Rightarrow q = q' \Rightarrow r = r'
    \)
    Existenz: Induktion auf \(\Grad(f)\)\\
    \(\Grad(f) = 0 \Rightarrow f = g \cdot 0 + f\)\\
    \(\Grad(f) = n + 1\)\\
    \(\Grad(f) < \Grad(g) = m \Rightarrow f = g \cdot 0 + f\)\\
    OBdA. \(n + 1 = \Grad(f) \ge \Grad(g) = m > 0\)\\
    \(f = a_{n + 1} \cdot T^{n+1} + \hat{f},
    \Grad(\hat{f}) \le n, a_{n + 1} \neq 0\)\\
    Sei \(
      f' = f - b_m^{-1} a_{n+1}T^{n+1-m} \cdot g \Rightarrow \Grad(f') \le n
    \)
    Ia: \(f' = g \cdot q' + r', \Grad(r') < \Grad(g)\)\\
    \(
      f' = f - b-b^{-1}a_{n+1}T{n + 1 - m} \cdot g
      \Rightarrow f = g(b_n^{-1}a_{n+1}T^{n+1-m} +q')+r'
      \Rightarrow \Grad(r') < \Grad(g)
    \)
  \end{proof}
\end{satz}

\begin{definition}[Polynom Teilt]
  \(f, g, q \in K[T]\), \(\Grad(g) > 0\)\\
  \(g \text{ teilt } f  = g|_f\Leftrightarrow f = g \cdot q\)\\
\end{definition}

\begin{definition}[Nullstellen von Polynomen]
  \(f \in K[T]\) besizt eine Nullstelle \(\lb \in K\) gdw.
  \((T - \lb) |_f \Leftrightarrow f(\lb) = 0\).\\
  f l\"asst sich dann schreiben als \(f = (T - \lb)q + r\).
\end{definition}

\begin{lemma}
  \(f \in K[t], f\neq 0, \Grad(f) = n
  \Rightarrow \) f besitzt h\"ochstens n Nullstellen in k.
  \begin{proof}
    \begin{itemize}
      \item[\(n=0\)]{
        \(\Rightarrow f = a_0, a_0 \neq 0\)
      }
      \item[\(n>0\)]{
        Falls f keine Nullstellen in K besitzt \(\Rightarrow\) ok!\\
        Sonst, sei \(\lb \in K\) eine Nullstelle von f.
        \(f = (T -\lb) \cdot g, \Grad(g) = n-1 < n\)\\
        I.A besitzt g h\"ochstens n - 1 Nullstellen. Jede Nullstelle von f ist
        entweder \(\lb\) oder eine Nullstelle von g. \(\Rightarrow\) f hat
        h\"ochstens n Nullstellen.\\
      }
    \end{itemize}
  \end{proof}
\end{lemma}

\begin{definition}[Vielfachheit einer Nullstelle]
  \(f \in K[T], f \neq 0, \lb \in K\) Nullstelle von f
  \(\Rightarrow f = (T-\lb)^{K_\lb}\cdot g, g(\lb \neq 0\).
  \(K_\lb\) ist die Vielfacheit der Nullstelle \(\lb\) in f.\\
\end{definition}

\begin{definition}
  Ein K\"orper hei\ss{}t algebraisch abgeschlossen, falls jedes Polynom \"uber 
  K positiven Grades eine Nullstelle besitzt.
\end{definition}

\textbf{Beispiele}
Ist \(\R\) algebraisch abgeschlossen? Nein: \(T^2 + 1\).\\
Bem.: \(\C\) ist algebraisch abgeschlossen.

\begin{bemerkung}
  Jeder algebraisch abgeschlossene K\"orper muss unendlich sein.
  Sei \(K = \{\lb_1, \dots \lb_n\}, f = (T - \lb), \dots, (T - \lb_n) + 1\).\\
\end{bemerkung}

\begin{lemma}
  K ist genau dann algebraisch abgeschlossen, wenn jedes Polynom positiven
  Grades in lineare Faktoren zerf\"allt.\\
  \(f = T(\lb_1) \dots (T - \lb_n)\).
  \begin{proof}
    \begin{itemize}
      \item[\(\Leftarrow\)]{trivial}
      \item[\(\Rightarrow\)]{
        \(
          \Grad(f) = n > 0
          \Rightarrow f = (T - \lb_1) \cdot g, \Grad(g) \le n - 1 < n
          \overset{I.A.}{\Rightarrow} f = c(T-\lb_1) \dots (T-\lb_n)
        \)
      }
    \end{itemize}
  \end{proof}
\end{lemma}

\begin{definition}[Vektorraum]
  Vektorraum V \"uber K ist eine abelsche Gruppe \((V, +, 0_V)\) zusammen mit
  einer Verkn\"upfung \(K \times V \to V\) \((\lb, v) \mapsto \lb v\) die die
  folgenden Bedingungen erf\"ullt:\\
  \begin{enumerate}
    \item{\(\lb (v+w) = \lb v + \lb w\)}
    \item{\(\lb(\mu \v) = (\lb \mu) v\)} 
    \item{\((\lb + \mu) v = \lb v + \mu v\)}
    \item{\(1_{k} v = v\)}
  \end{enumerate}
\end{definition}

\begin{definition}[Untervektorraum]
  Ein Untervektorraum \(U \subset V\) ist eine Untergruppe, welche unter der
  Skalarmultiplikation abgeschlossen ist. 
\end{definition}

\begin{bemerkung}
  \(\{U_i\}_{i \in I}\) Untervektorr\"aume von V
  \(\Rightarrow \bigcap_{i\in I} U_{i}\) ist Untervektorraum.
  Insb. gebenen \(M \subset V\) existiert \(\Span(M) = <M> = \) der kleinste
  Unterraum von V, der M enth\"alt.\\
  \(\Span(M) = \sum_{i=1}^{n} \lb_{i} m_{i}\),
  \(m_{i} \in M, \lb_{i} \in K, n \in \N\)\\
  M ist ein Erzeugendensystem f\"ur \(\Span(M)\)\\
  Au\ss{}erdem gilt:\\
  \(\sum_{i\in I}U_{i} = \Span(\bigcup_{i \in I} U_{i})\)\\
  \(M_{1} \subset M_{2} \Rightarrow \Span(M_{1}) \subset \Span(M_{2})\)
\end{bemerkung}

\begin{definition}[Lineare Unabh\"angigkeit]
  Sei \(V\) ein Vektorraum \"uber K. Dann gilt \(v_{1}, \dots v_{n}\) sind
  linear unabh\"angig falls
  \(\forall \lb_{1}, \dots, \lb_{n} \in K : \sum \lb_{i} v_{i} 
  \Rightarrow \lb_{1} = \dots = \lb_{n} = 0\) 
  \(M \subset V\) ist linear unabh\"angig, falls jede endliche Teilmenge von M
  linear unabh\"angig ist. \"Aquivalent dazu ist: M ist linear unabh\"angig, 
  falls kein Element von M sich als Linearkombination der anderen schreiben
  l\"asst.
\end{definition}

\begin{definition}[Basis]
  Sei \(B = \{v_1, \dots, v_n\}, v_i \in V\).
  Die folgenden Aussagen sind \"aquivalent und definieren eine Basis:
  \begin{enumerate}
    \item{B ist ein lineare unabh\"angiges Erzeugendensystem von V}
    \item{
      Jedes Element von V l\"asst sich eindeutig als Linearkombination der
      Elemente in B schreiben.
    }
    \item{B ist ein minimales Erzeugendensystem.}
    \item{B ist maximal lineare unabh\"angig.}
  \end{enumerate}
\end{definition}

\begin{satz}[Basiserg\"anzungssatz]
  Sei \(M \subset V\) lineare unabh\"angig, dann gilt \( \exists B \subset V\),
  und B ist eine Basis welche M ent\"alt. Insbesondere hat jeder Vektorraum eine
  Basis. ''Je zwei Basen sind in Bijektion''.
\end{satz}

\begin{definition}[Dimension]  
  V ist endlichdimensional, falls V eine endliche Basis besitzt. Sonst ist V
  unendlichdimensional. Fall V endlichdimensional ist, ist die Dimension von V
  definert durch:\\
  \(dim(V) = |B|\) mit B beliebeige Basis.
\end{definition}

\begin{satz}[Basisauswahlsatz]
  Sei \(M \subset V\) ein Erzeugendensystem von V, dann gilt
  \(\exists B \subset M\) mit B ist eine Basis von V.
\end{satz}

\begin{lemma}
  Sei \(U \subset V\) ein Unterraum, dann gilt
  \(\dim(V) < \infty \Rightarrow \dim(U) < \infty\)
\end{lemma}

\begin{lemma}
  Die Dimension ist modular:
  \(\dim(U_1 + U_2) + \dim(U_1 \cap U_2) = \dim(U_1) + \dim(U_2)\)
\end{lemma}

\begin{definition}[Direktes Produkt von Vektorr\"aumen]
  \(
  V = U_1 \oplus U_2 \Leftrightarrow V = U_1 + U_2 \land U_1 \cap U_2 = \{0\}
  \)\\
  \(V = \oplus_{i \in I} U_i \Leftrightarrow V = \sum_{i \in I} U_i \) und die
  Familie ist transversal:
  \(\{U_i\}_{i \in I} \to U_i \cap (\sum_{j \in I} U_j) = \{0\}\)
\end{definition}

\begin{definition}[Komplement\"ar]
  Sei \(U \subset V\) ein Unterverktorraum, dann gilt
  \(\exists \hat{U} \subset V : V = U \oplus \hat{U}\).\\
  \(\hat{U}\) hei\ss{}t dann Komplement\"ar zu U.
\end{definition}

\textbf{Beispiele}\\
\(K^2\) ist ein K-VR. \(\v{1\\0} \v{0\\1}\) ist eine Basis.\\
\(U = \Span(\v{1\\0})\). \(K^2 = U \oplus \Span(\v{0\\1})\).
\(K^2 = U \oplus \Span(\v{1 \\1}\).

\begin{definition}[Lineare Abbildungen]
  \(F : V \to W\) ist linear, falls gilt:
  \(F(\lb v + \mu u) = \lb F(v) + \mu F(u)\)
\end{definition}

\begin{definition}[Kern und Bild]
  \(\Ker(F) = \{v \in V \ F(v) = 0\}\)\\
  \(\Ima(F) = \{w \in W| \exists v \in V : F(V) = w\}\)\\
  \(\Ker(F)\) ist ein Untervektorraum von V, \(\Ima(F)\) ist ein Untervektorraum
  von W.
\end{definition}

\begin{lemma}
  Falls B eine Basis von V ist, ist \(F(B)\) ein Erzeugendensystem
   von \(Im(F)\).
  F ist injektiv genau dann wenn \(\Ker(F) = \{0\}\).
\end{lemma}

\begin{lemma}
  V endlichdimensional: \(dim(V) = dim(Ker(F)) + dim(Im(F))\).\\
  \(V / Ker(f) \cong \Ima(F)\).
\end{lemma}

\begin{bemerkung}
  V, W endlichdimensional, \(\{v_1, \dots, v_n\}\) Basis von V \(V \cong K^n\),
  \(v_i \mapsto e_i\).
\end{bemerkung}

\begin{definition}[Matrix]
  Sei \(F : V \to W, dim(V) = n, dim(W) = m, \{v_1, \dots, v_n\}\) Basis von V,
  \(\{w_1, \dots, w_n\}\) Basis von W.\\
  \(K^n \cong V \overset{F}{\to} W \cong K^m\). Dadurch wird durch F und die
  beiden Basen eine Abbildung von \(K^n\) nach \(K^m\) definiert. Diese
  Diese Abbildung kann durch eine Matrix A dargestellt werden.\\
  \(\v{\lb_1\\ \vdots \\ \lb_n} \mapsto A \v{\lb_1, \vdots \lb_n}\)\\
  \(F(v_j) = \sum_{i=1}^m a_{ij}w_i\)\\
  \(
  F(v_1), \dots, F(v_n)\\
  \begin{pmatrix}
    a_{11} & \dots  & a_{1m}\\
    \vdots & \ddots & \vdots\\
    a_{n1} & \dots  & a_{nm}
  \end{pmatrix}
  \)
  ist die mxn Matrix A.
\end{definition}

\begin{definition}[Rang einer Matrix]
  \(Rg(A) = \dim(\Span(\text{Spaltenvektoren}))
  = \dim(\Span(\text{Zeilenvektoren}))\)\\
  \(F : V \to W\) linear. \(Rg(F) = Rg(A) = dim(\Ima(F))\), mit A eine beliebige
  darstellende Matrix von F.
\end{definition}

\begin{satz}[Normalform]
  Es seien V, W endlichdimensional. Dann existieren Basen \(\{v_1, \dots v_n\}\)
   von V, \(\{w_1, \dots, w_n\}\) von W, so dass die darstellende Matrix von F
   der Form
  \(
    \begin{pmatrix}
      1      & \dots  & 0      & 0      & \dots  & 0\\
      \vdots & \ddots & \vdots & \vdots & \ddots & \vdots\\
      0      & \dots  & 1      & 0      & \dots  & 0\\
      0      & \dots  & 0      & 0      & \dots  & 0\\
      \vdots & \ddots & \vdots & \vdots & \ddots & \vdots\\
      0      & \dots  & 0      & 0      & \dots  & 0 & 
    \end{pmatrix}
  \) ist.
  \begin{proof}
    Sei \(U = \Ker(F)\) und \(\{v_{r+1}, \dots, v_n\}\) eine Basis von
    U. Sei \(U'\) ein Komplement von U in V \(\Rightarrow V = U \oplus U'\).
    Sei \(\{v_1, \dots, v_r\}\) eine Basis von \(U'\).
    \(B = \{v_1, \dots v_n\}\) ist eine Basis von V. \(\Ima(F)\) hat
    \(\{F(v_1), \dots, F(v_r)\}\) als Basis.\\
    \(
      \sum_{i=1}^n \lb_i F(v_i) = 0
      \Rightarrow F(\sum_{i=1}^n \lb_i v_i) = 0
      \Rightarrow \sum_{i=1}^n \lb_i v_i) \in U
                  \land \sum_{i=1}^n \lb_i v_i) \in U'
      \Rightarrow \sum_{i=1}^n \lb_i v_i) = 0
      \Rightarrow \lb_1 = \dots \lb_n = 0
    \).
    Erg\"anze \(\{F(v_1), \dots, F(v_r)\}\) zu einer Basis
    \(B' = \{w_1, \dots, w_m\}\) von W.
    \(
      F(v_1), \dots, F(v_r), F(v_{r+1}), \dots, F(v_n)\\
      \begin{pmatrix}
        1      & \dots  & 0      & 0      & \dots  & 0\\
        \vdots & \ddots & \vdots & \vdots & \ddots & \vdots\\
        0      & \dots  & 1      & 0      & \dots  & 0\\
        0      & \dots  & 0      & 0      & \dots  & 0\\
        \vdots & \ddots & \vdots & \vdots & \ddots & \vdots\\
        0      & \dots  & 0      & 0      & \dots  & 0 & 
      \end{pmatrix}
    \)
  \end{proof}
\end{satz}

\begin{definition}[Invertierbarkeit von Matrizen]
  \(A \in M_{n\times n}(K)\) ist invertierbar, fall es eine Matrix
  \(B \in M_{n\times n}(K)\) gibt, so dass \(A \cdot B = B \cdot A = Id_n\).
  B wird dann als \(A^{-1}\) bezeichnet.\\
  \(GL(n, k) = Gl_n(K) = \{A \in M_{n\times n}(K) \text{invertierbar}\}\) ist
  eine Gruppe.\\
  \(A \in GL_k(n) \Leftrightarrow \rg(A) = n\) (Eine Matrix ist genau dann
  invertierbar, wenn sie regul\"ar ist).
\end{definition}

\begin{bemerkung}
  Sei A regul\"ar. Dann besitz ein Gleichungssystem der Form
  \(A \v{\lb_1\\ \vdots \\ \lb_n} = \v{b_1 \\ \vdots \\ b_n}\) die Eindeutige
  L\"osung, \(A^{-1} \v{b_1 \\ \vdots \\ b_n}\).
\end{bemerkung}

\begin{bemerkung}
  A ist regul\"ar genau dann wenn A sich durch elementare Zeilenoperationen in
  \(Id_n\) \"uberf\"uhren l\"asst.\\
  \(E_{i,j}\) sei Die Matrix, die an der Stelle ij 1 ist, ansonsten 0.\\
  Elementare Zeilenoperationen sind:\\
  Multiplikation der Zeile i mit \(\lb\): \(\Id_n + (\lb  -1) E_{i,j}\).\\
  Addieren von \(\lb\) mal der iten Zeilten zur jten: \(Id_n + \lb E_{i,j}\).\\
  Vertauschung der i-ten und j-ten Zeile: \(Id_n - E_{ii} - E_{jj} + E_{j,i} + E_{i,j}\)\\
\end{bemerkung}

\begin{bemerkung}
Das inverse einer Matrix l\"asst sich durch nutzen dieser elementaren
Zeilenoperationen nach z.B. dem Gau\ss{}-Jordan Verfahren errechnen:\\
\(
  \left(
  \begin{array}{c|c}
    A & Id_n
  \end{array}
  \right)
  \overset{Zeilenoperationen}{\to}
  \left(
  \begin{array}{c|c}
  Id_n & A^{-1}
    \end{array}
  \right)
\)\\
Die linke H\"alfte der Ergebnis Matrix enth\"alt dann \(A^{-1}\), denn:\\
\(
B_m \dots B_2 B_1 A = Id_n \Rightarrow B_m \dots B_1 = A^{-1}
\)
\end{bemerkung}

% Basiswechselmatrizen?
\begin{definition}[\"Ubergangsmatrizen]
  Es sei \(dim(V) = n\) und \(\{v_1, \dots, v_n\}\),  \(\{v_1', \dots, v_n'\}\)
  Basen von V. Weiterhin sei \(F: V \to V, v_i \mapsto v_i'\). Dann gilt:\\
  \(v_i' = \sum_{ij} s_{ij} v_j\) und die darstellende Matrix S von F, 
  \(
  S = \begin{pmatrix}
    s_{11} & \dots  & s_{1m}\\
    \vdots & \ddots & \vdots\\
    s_{n1} & \dots  & s_{nm}
  \end{pmatrix}
  \)
  ist regul\"ar.
\end{definition}


\begin{definition}
  Zwei (mxn) Matrizen A, A' sind \"aquivalent, falls es reg\"lare matrizen \(T
  \in GL_m(K), s \in GL_n(K)\) gibt, so dass \(A' = T^{-1} \cdot A \cdot S\).\\
  \(A, A' \in M_{n \times n}(K)\)  sind \"ahnlich, fall es \(S \in GL_n(K)\)
  gibt, so dass \(A' = s^{-1} \cdot A \cdot S\).
\end{definition}

\begin{bemerkung}
  \"Ahnlichkeit ist eine \"Aquivalenzrelation auf \(M_{n \times n}(K)\).
\end{bemerkung}


\begin{definition}[Determinante]
  \(det K^n \to K\) ist eine multilineare alternierende  Abildung der Art,
  dass \(det(e_1, \dots, e_n) = 1\).\\
  \(A \in M_{n \times n}(K)\)\\
  \(A = (a_1 | a_2 | \dots | a)n) \Rightarrow det(a_1, a_2, \dots, a_n) = det(A)\).\\
  \(A = (a_ij), det(a_ij) = \sum sign(\pi) \cdot \Pi_{i=1}^n a_{\pi(i)i}\) mit
  \(sign(\pi) = -1^{\text{Anzahl der Fehlst\"ande von }\pi}\) bzw. Anzahl von
  Faktoren von \(\pi\) als Produkt von Transpositionen.
\end{definition}

Eigenschaften von Determinanten:\\
\begin{enumerate}
  \item{\(det(A \cdot B) = \det(A) \det(B)\)\\}
  \item{A ist genau dann invertierbar, wenn \(det(A) \neq 0\)\\}
  \item{\(\det(A^-1) = det(A)^{-1}\)}
  \item{\(\det(A^T) = \det(A)\)}
\end{enumerate}

\begin{bemerkung}
  \(Id_n + (-\Id_n)\) ist nicht invertierbar, also \(\exists A, B : det(A+B) \neq \det(A) + \det(B)\)
\end{bemerkung}

\begin{satz}[Laplacescher entwicklkungssats]
  Sei \(j_0\) ein Spaltenindex\\
  \(det(A) = \sum_{i=1}^n (-1)^{i+j_0} a_{ij_o} det(A_{j_0i})\) wobei
  \(A_{j_{0i}}\) die Matrix ohne Zeile \(j_0\) und Spalte i ist.
\end{satz}


\begin{satz}[Cramersche Regel]
  \((a_1 | \dots | a_n) = A, A \v{\lb_1\\ \vdots \\ \lb_n} = \v{b_1 \\ \vdots \\ b_n}\)
  Falls A regul|'ar ist, gibt es eine einzige L\'Osung zum System:
  \(\lb_j = \frac{det(a_1, \dots, a_{j-1}, b_j, a_{j+1}, \dots, a_n}{det(A)}\)
\end{satz}

\begin{definition}[Determinante eines Homomorphismus]
  Sei \(F : V \to V\). \(det(F) = det(A)\) woei \(A\) eine Darstellungmatrix von F bezgl. einer Baiss \(\{v_1, \dots, v_n\}\).
\end{definition}


\begin{definition}[Adjunte Matrix]
Sei A eine \(n\times n\) Matrix, dann ist die Adjunte von A\\
\(\adj(A) = (\gamma_{ij})\) mit \(\gamma_{ij} = (-1)^{i+ j} \det(A_{ij}\)
\end{definition}

\begin{bemerkung}
  Sei \(c_i\) die j-te Zeile von \(\adj(A)\). Sei weiterhing \(a_i\) die i-te
  Spalte von A.\\
  \(\gamma_{j1}, \dots, \gamma_{jn} \cdot \v{a_{1i}\ \\vdots \\ a_{ni}}
  = \sum_{k=1}^n \gamma_{jk} a_{ki} = \sum_{k=1}^n (-1)^{j+n} a_{ki} \det(A_{jk})
  \overset{\text{Laplacescher Entw. Satz}}{=}
  \det(a_1, \dots, a_{j-1} , a_i, a_{j+1}, \dots, a_n) =
  \begin{cases}
   \det(A) & j=i\\
   0 & j \neq i
  \end{cases}\)\\
  Angenommen A ist regul\"ar.\\
  \(adj(A) \cdot A = det(A) \cdot Id_n
    \Rightarrow \frac{\adj(A)}{\det(A) \cdot A} = \Id_n = A^{-1} \cdot A
    \Rightarrow \frac{\adj(A)}{\det(A)} = A^{-1}
    \Rightarrow A \cdot \adj(A) = det(A) Id_n
  \)
\end{bemerkung}

\chapter{Diagonalisiserbarkeit}
Sei V ein Vektorraum, \(\{U_i\}_{i=1}^k\) Unterr\"aume von V.\\
\(V = \oplus_{i=1}^k U_i
\Leftrightarrow V = \sum_{i=1}^n U_i \land U_i \bigcap(\sum_{j=1}^k U_i) = 0\)\\
\"Aquivalent dazu ist, dass jeder Vektor \(v \in V\) sich eindeutig als
Linearkombination von Vektoren \(\cup_{j=i}^k B_j\) schreiben l\"asst, woebi
\(B_j\) eine Basis von \(U_i\) ist.

\begin{definition}[Eigenwerte und -vektoren]
  Ein Endomorphismus \(F : V \to V\) besitzt einen Eigenvektor, falls es ein
  \(v \in V \setminus \{0\}\), so dass \(F(V)  \lb \cdot v\) f\"ur ein
  \(\lb \in K\). Falls \(F(v) = \lb v\) ist \(\lambda\) eindeutig bestimmt durch
  F und v. \(\lambda\) ist dann ein Eigenwert von F.
\end{definition}

\begin{definition}[Eigenr\"aume]
  \(\lb \in K, F V \to V\) Endomorphismus.\\
  \(V(\lb) = \{v \in V | F(v) = \lb v\}\), der Eigeneraum zu \(\lb\) is ein UVR.
\end{definition}

\begin{bemerkung}
  \(\lb\) ist ein Eigenwet von F gdw, \(dim(V(\lb)) \ge 1\).
\end{bemerkung}

\begin{bemerkung}
  Falls \(\lb_1, \dots, \lb_k\) verschiedene Eigenwerte von F \(\Rightarrow
  V(\lb_i) \cap \sum_{j=1, j \neq i}^k V(\lb_j) = \{0\}\)
\end{bemerkung}


\begin{definition}[Diagonalisiserbarkeit]
  Sei V ein endlichdimensionaler Vektorraum. \(F : V \to V\) Endomorphismus.
  Bzw. eine Matrix \(A : K^n \to K^n\). F ist diagonalisierbar, falls
  \(V = \oplus_{i=1}^k V('lb), \lb\) verschiedene Eigenwerte von F.\\
  \"Aquivalent dazu, wenn V eine basis von Eigenwerten von F besitzt.
  \"Aquivalent dazu, wenn F bez\"uglich einer Basis von V die Darstellungsmatrix
  \(
    \begin{pmatrix}
      \lb_1  &        & 0\\
             & \ddots &\\
      0      &        & \lb_n
    \end{pmatrix}
  \) hat.\\
  \"Aquivalentz dazu, f\"ur Matrizen: A ist diagonalisierbar gdw.es eine regul\"are
  Matrix S gibt, soda\ss{} \(S^{-1}AS = 
  \begin{pmatrix}
    \lb_1  &        & 0\\
           & \ddots &\\
    0      &        & \lb_n
  \end{pmatrix}\)
\end{definition}

\begin{satz}
  \(A \in M_{n\times n}(K), \lb \in K\)\\
  \(\lb\) ist ein Eigenwert von A gdw. \(\lb Id_n - A\) nicht regul\"ar ist.
  \(\Leftrightarrow det(\lb \cdot Id_n -A) = 0\)
\end{satz}

\begin{definition}[Charakteristisches Polynom]
  Das charakteristische Polynom einer Matrix \(A \in M_{nxn}(K)\) ist 
  \(\chi_{A(T)} = det(T \cdot Id_n - A)\)
\end{definition}

\begin{bemerkung}
  \(\lb\) ist ein eigenwert von A \(\Leftrightarrow \chi_A(\lb) = 0\)
\end{bemerkung}

\textbf{Beispiel}
\(
  \begin{pmatrix}
   0 & -1\\
   -1 & 0
  \end{pmatrix}\\
  \chi_{A(T)} = T^2 + 1 = \det(
  \begin{pmatrix}
    T & -1\\
    1 & T
  \end{pmatrix}
  )
\)

\begin{bemerkung}
  A und \(A'\) \"ahnlich, \(A' = s^{-1}AS \Rightarrow \chi_A(T) = \xi_{A'}(T)\).
  Insebsondere k\"onnen wir \"uber das charakteritische Polynom eines
  Endomorphismus reden.\\
  \(A \in M_{nxn}(K), \chi_A(T) = T^n + b_{n-1} T^{n-1} + \dots + b_o\) wobei
  \(b_0 = (-1)^n det(A), b_n{-1} = - Tr(A) = - \sum_{i=1}^n a_{ii}\)
\end{bemerkung}

\begin{korrolar}
  Ein Endomorphismus \(F : V \to V\) mit \(\dim(V) = n < \infty\) kann
  h\"ochstens n viele Eigenwerte besizten.
\end{korrolar}

\begin{korrolar}
  \(F : V \to V\) mit \(\dim(V) = n < \infty\) mit verschiedenen Eigenwerten
  \((\lb_1, \dots, \lb_n)\) ist diagonalisierbar,  gdw. 
  \(n = \sum_{i=1}^k d_i, d_i = \dim(V(\lb_i))\). \(d_i\) hei\ss{}t geometrische
  Vielfachheit von \(\lb_i\).
  
  \begin{proof} \hfill\break
  \(\Rightarrow\)\\
  F ist diag. gdw. V eine Basis aus Eigenvektoren besitzt, welche aus
  \(\cup_{i=1}^n B_i\) besteht, \(|B_I| = di = dim(V\lb_i)\),
  \(n = |B| = \sum_{i=1}^k |B_i|\)\\
  \(\Leftarrow\)\\
    \(n = \sum d_i \Rightarrow \dim(\sum_{i=1}^k (V(\lb_i))) = n
    \Rightarrow V = \sum_{i=1}^k(V(lb_i)\) da die Eigenr\"aume tranversal sind,
     und ein Vektorraum nur einen UVR der dimension dim(V) hat, sich selbst.
  \end{proof}
\end{korrolar}


% potentially missing part of the beginning of lecture 4
\begin{definition}[Algebraische Vielfachheit]
  Es seien \(F : V \to V\) ein Endomorphismus, \(dim(V) = n < \infty\),
  \(\lb \in K\) Eigenwert \(\Rightarrow \chi_F(\lb) = 0\).\\
  Dann gilt  \(\chi_F(T) = (T-\lb)^K G(T)\), \(G(\lb) \neq 0\). k ist die
  algebraische Vielfachheit von \(\lb\), bzw. \(\ord_\lb(F)\).
\end{definition}

\begin{bemerkung}
  \(\ord_\lb(F) \ge \dim(V(\lb))\)\\
  \begin{proof}
    Sei \({v_1, \dots, v_k}\) eine Basis von \(V(\lb)\). Wir erweitern sie zu
    einer Basis \(\{v_1, \dots, v_k, v_{K+1}, \dots, v_n\}\) von V. Die
    Darstellungsmatrix M von F bzwg. B ist dann\\
    \(\{F(v_1), \dots, F(v_k), F(v_{k+1}), F(v_n)\}\).\\
    \(
      \begin{pmatrix}
        \lb &        & 0         & \\
            & \ddots &           & \\
        0   &        &  \lambda  & C_2 \\
            & 0  
      \end{pmatrix}
    \)\\
    Wobei \(C_2 \in Mat_{n-k \times k}(K)\).\\
    \(
      \chi_F(T) = det(T Id_n - M)
      = (T - \lb)^k \cdot \det(T Id_{n-k} \cdot C_1)\\
      \Rightarrow \ord_\lb(F) \ge K
    \). Wobei \(\det(T Id_{n-k} \cdot C_1) = 0\) sein kann.
  \end{proof}
\end{bemerkung}

\begin{lemma}
  Sei V endlichdimensional, \(F : V \to V\) ein Endomorphismus,
  U ein F-Invarianter Unterraum (\(F(U) \subset U\)).\\
  \(F' : V/U \to V/U\) ist eine lineare Abbildung,
  \(\bar{V} \mapsto \bar{F(V)}\). \(F'\) ist woldefiniert, linear und es gilt
  \(\chi_F(T) = \xi_{F|_U}(T)\cdot\xi_{F'}(T)\)
  \begin{proof}\hfill\break
    \(F'\) ist wohldefiniert;\\
    \(\bar{v_1} = \bar{v} \overset{zZ}{\Rightarrow} F'(v_1) = F(v)\)
    \(
      \bar{v_1} = \bar{v} 
      \Rightarrow v_1 = v + (v_1 - v), v_1 - v \in U\\
      \Rightarrow F(v_1) = F(v) + F(v_1 - v),  F(v_1 - v) \in U
      \Rightarrow \bar{F(v_1)} = \bar{F(v)}
    \)\\
    Restklassen sind linear und F ist linear \(\Rightarrow F'\) ist linear.\\
    Sei \(\{u_1, \dots, u_k\}\) eine Basis von U. erweitert zu
    \(\{u_1, \dots, u_k, v_{k+1}, \dots, v_n\}\) sei sie eine Basis von V.
    \begin{bemerkung}
      \(\{\bar{v_{k+1}}, \dots, \bar{{v_n}}\}\) ist eine Basis von \(V/U\).
      Bew. Einfach.
    \end{bemerkung}
    Darstellungsmatrix H von F bzgl. B:\\
    \(
      \begin{array}{c}
        u_1\\ \vdots \\ u_k \\ v_{k+1} \\ \vdots \\ v_n
      \end{array}
      \begin{pmatrix}
        &           \\
        & A      &        &        &  & C_2\\
        &           \\
        & 0      & \dots  & 0   \\
        & \vdots & \ddots & \vdots &  &C_1\\
        & 0      & \dots  & 0
      \end{pmatrix}
    \) mit \(A, C_1, C_2\) Matrizen.\\
    \(\chi_F(T) = det(T Id_n - H)
    = \det(T Id_n - \begin{pmatrix}A & C_2 \\ 0 & C_1\end{pmatrix})\\
    = \det(\begin{pmatrix}T _id_k - A & -C_2 \\ 0 & T_ Id_{n-k} -C_1\end{pmatrix})
    = \det(T_id_k - A) \cot det (T_E{n-k} - C_1)
    \)\\
     A ist die Darstellungsmatrix von \(F|_U\) bez\"uglich \(\{u_1, \dots, u_k\}\)
     \(\Rightarrow det(T Id_k - A) = \chi_{F|_U}(T)\)\\
     \(C_1\) ist die Darstellungmatrixvon \(F'\) bzg.
     \(\{\bar{v_{k+1}}, \dots, \bar{v_n}\}\).\\
     \(\Rightarrow \det(T \Id_{n-k}-C_1) = \chi_{F'}(T)\)
  \end{proof}
\end{lemma}

\begin{satz}
  Sei \(K\) ein K\"orper, \(dim(V) < \infty, F : V \to V\) ein Endomorphismus so 
  gilt:\\
  F Diagonalisierbar gdw \(\chi_F(T) = (T-\lb_1)^{k_1} \dots (T- \lb_n)^{k_n}\)
  in Linearfaktoren zerf\"allt, wobei f\"ur jeden Faktor \(\lb_1, \dots, \lb_n\)
  \(T-\lb_i\) gilt \(\ord_{lb_i}(F) = \dim(V(\lb_i))\).
  \begin{proof}\hfill\break
  \(\Rightarrow\)\\
    Sei \(b = \{v_1, \dots, v_n\}\) eine Basis von Eigenvektoren. Seien
    \(\lb_1, \dots \lb_r\) die verschiedenen Eigenwerte. Ordne nun B um so dass\\
    \(
      v_1, \dots, v_{d_1} \in V(\lb_i), 
      v_{d_1 + 1}, \dots v_{d_1 + d_2} \in V(\lb_2)
      , \dots,
      v_{d_1  + \dots + d_{r-1}}, \dots v_{d_1 + \dots + d_r} \in V(\lb_r)\) mit
    \(d_i = dim(V(\lb_i))\).\\
    Die Darstellungsmatrix von F bzgl. B:\\
    \((F(v_1), \dots, F(v_{d_1}), \dots F(v_r)\)\\
    \(
      \begin{pmatrix}
        \lb_1 & \\
              & \ddots &  \\
              &        & \lb_1\\
              &        &       & \lb_r & \\
              &        &       &       & \ddots &  \\
              &        &       &       &        & \lb_r
      \end{pmatrix}
      %d_1 viele  \lb_1 auf der Diagonale
    \)\\
    Wobei \(d_i\) viele \(\lb_i\) auf der Diagonale sind\\
    \(\chi_F(T) = det(T \Id_n - A) = (T - \lb_1)^{d_1} \dots T(-\lb_r)^{d_t}\)
    \(((T - \lb_2)^{d_2} \dots T(-\lb_r)^{d_t})(\lb_1) \neq 0
    \Rightarrow d_i = \ord_{\lb_I}(F)
    \), da die \(\lb_i\) verschieden sind.\\
    \(\Leftarrow\)\\
    \(\chi_F(T) = (T - \lb_i)^{d_1} \dots (T - \lb_r)^{d_r}\)\\
    F ist diag \(\Leftrightarrow n = dim(V) = \sum d_i\)
  \end{proof}
\end{satz}


\begin{definition}
  Eine Matrix \(A \in M_{n\times n}(K)\) ist 
  trigonalisierbar, wenn sie \"ahnlich zu einer oberen  Dreiecksmatrix ist:\\
  \(
    \begin{pmatrix}
    a_{11} & \dots  & a_{1n}\\
           & \ddots & \vdots\\
    0      &        & a_{nn}\\
    \end{pmatrix}
  \)
\end{definition}

\begin{satz}
  \(F : V \to V\) ist trigonalisierbar? gdw. \(\chi_F(t)\) in
  Linearfaktoren zerf\"allt \(\chi_F(T) = (T - \lb_1) \dots (T - \lb_n)\).
  
  \begin{proof}
    \(\Rightarrow\)
    F hat Darstellungsmatrix:\\
    \(
    \begin{pmatrix}
      a_11\\
           & a){21} & X\\
                    & \ddots\\
         0                     & a_{nn}\\
         
    \end{pmatrix}\\
    \chi_F(T) = det(T \cdot Id_n - A) = \Pi_{i=1}^n (T - a_{ii})
    \)
    \(\Leftarrow\)
    Induktion \"uber \(n = dim(V)\)\\
    \(n=1 \Rightarrow \) Jede \(1 \times 1\) Matrix ist in oberer Dreiecksform.\\
    \(n \ge 2\)  \(\chi_F(t) = (T - \lb_1) \dots (T_{\lb_n})\). \(\lb_1\) ist ein
    Eigenwert \( \Rightarrow \exists v_1 \in V\{0\} : F(v_10 + \lb_1 v_1\).
    \(U = \Span(v_1) \subset V\)  ist F-invariant. Nach Lemma ... gilt
    \(T - \lb_1) \Pi_{i=2}^n(T - \lb_n) = xi_F(T) = \chi_{F\_U} \cdot \xi_{F'}\), \(\xi_{F|_U} = (T - \lb_1)\)
    \(K[T]\) ist ein Integrit\"atsbereich \(\Rightarrow
    \Xi_{F'}(T) = \pi_{i=2}^n (T-\lb_i), \dim(V/U) < \dim(V)\).\\
    Nach Ia gibt ese eine Basis\(\exists (\bar{v}_2, \dots, \bar{v}_n)\) von
    \(V/U\) derart, dass \(F'\) bzgl. dieser Basis Darstellungsmatrix\\
    \(
    \begin{pmatrix}
      \lb_2 &        & X\\
            & \ddots &\\
      0     &        & \lb_n\\
    \end{pmatrix} = (\mu_{ij})
    \)
    Behauptung: Seien \(v_i \in V, \bar{v_i} = \bar{v}_i, 2 \le i \le n, \{v_1, v_2, \dots, v_n\}\) ist eine
    Basis von V. Bew. \"Ubungsaufgabe.\\
    Frage: Wie sieht die Darstellungmatrix von F bzgl \(\{v_1, \dots v_n\}\)
    aus?\\
    \(\sum_{2\le i \le j} \mu_{ij} \bar{v_i} = F'(v_j) = \bar{F(v_j)}\)\\
    \(\sum_{2\le i \le j} \mu_{ij} \bar{v_i} = \bar{\sum_{2\le i \le j} \mu_{ij} v_i}
    \Rightarrow \exists  \mu_{ij} \in K / F(v_j) = \mu_{ij}v_1 + \sum_{2 \le i \le j} \mu_{ij}v_j = \sum_{1 \le i \le j} \mu_{ij}v_j\)\\
    \(F(v_1) F(v_2), \dots, F(v_n)\\
    \begin{pmatrix}
    \lb_1 & \mu_{12}
          & \lb_2 &       & X
          &       & \ddots
    0     &       &       & \lb_n
    \end{pmatrix}
    \)
  \end{proof}
\end{satz}

\begin{korrolar}
  Jeder Endomorphismus eines endlichdimensionalen Vektorraumes \"uber einen
  algebraisch abgeschlossenen K\"orper (z.B. \(\C\)) ist trigonalisierbar.
\end{korrolar}

\begin{lemma}
  Sei V endlichdimensional \'uber ienen K\"orper K.\\
  \(v \in V \setminus \{0\} \exists r \in \N : F^r(v) = sum_{i=0}^{r-1}a_i F^i(v)\) \(a_i\) ist eindeutig bestimmt.
  Insb. ist \(U = Span(v, F(v)m \dots, F^{r-1}(v))\) ist F inviariant, hat Basis \(v, F(V), \dots, F^{r-1}(v))\)\\
  \(F \uparrow U\) hat Darstellungsmatrix
  \(
  \begin{pmatrix}
  0     ... a_0
  1 0   ... \vdots
    1 0     a_n
  \end{pmatrix}
  \)
  \(\chi_{F \uparrow U}(T) = T^r - a_{r-1} T^{r-1} - \dots - a_0\).
  
  \begin{proof}
    \(n = \dim(V)\).
    \(\{v, F(v), \dots, F^n(v)\}\) sind lin.abh. Sei r die kleinste nat\'Urlich Zahl,
    so dass \(\{v, F(v), \dots, F^r(v)\}\) lin. abh sind. Dann sind
    \(\{v, F(v), \dots, F^{r-1}(v)\}\) linear unabh\"angig.
    \(\overset{\text{Ausstauschprinzip von r}}{\Rightarrow} F^r(v) = \sum_{i=0}^{r-1} a_i F^i(v)\)
    wobei \(a_i\) eindeutig bestimmt sind.\\
    \(U = \Span(v_1, \dots, F^{r-1}(v))\) ist F-invariant. Sei \(u \in U, u = \sum_{0}^{r-1} \mu_iF^i(v)\\
    F(u) = \sum_{i=0}^{r-1} \mu_i F^{i+1}(v)\\
    = \sum_{i=0}^{r-2} \mu_iF({i+1}(v) + \mu_{i-1}F^r(v)\)
    Wobei \(F^r(v) = \sum_{i=0}^{r-1} a_iF^i(v)\).\\
    Beide teile der Summe \(\sum_{i=0}^{r-2} \mu_iF({i+1}(v) + \mu_{i-1}F^r(v)\) liegen eindeutig in U, also liegt auch die Summe in U.\\
    \(\{v_1, \dots, F^{r-1}(v)\}\) ist eine Basis von U.\\
    \(F(v_1), \dots F(F^{3-1}(v))\\
    \begin{pmatrix}
      0 & 0 & \dots & a_0
      1 & 0 &       & \vdots
      0 & 1 &       & a_{r-1}
    \end{pmatrix}\)\\
    \(\chi_{F|U}(T) = det(T Id_r - A)
    = det(    \begin{pmatrix}
      T & 0 & \dots & -a_0
      -1 & T &       & -\vdots
      0 & -1 &       & -a_{r-1}
    \end{pmatrix}\)\\
    Laplacsche Entwicklung nach der letzten Spalte:
    \(= (-1)^{r+1}(-a_0) det(\ve{-1 & T\\0 -1 & \ddots} + (-1)^{r+2}(-a_1)\det(\ve{& T\\0 -1 & \ddots}) + \dots + (-1)^{2r} (T - a_r) \det(\ve{T\\-1 & T\\  & -1 T})\)
    \(= (-1)^r a_0 (-1)^r + (-1)^{r+1} a_1T (-1)^{r+1} + \dots + (-1)^{2r}(T - a_1)T^{r-1}
    = - a_0 - a_1 T - \dots - a_{r-1}T^{r-1} + T^r\)
  \end{proof}
\end{lemma}

Notation:\\
\(P(T) \in K[T], P(T) + \sum_{i=0}^m a_iT^i\), \(F : V \to V\) Endomorphismus.\\
\(P(f) : V \to V, v \mapsto \sum_{i=0}^m a_iF^i(v)\)\\
Mit dieser Notation haben wir, dass im vorigen Lemma
\(\chi_{F|_u}(F) = F^r(v) - \sum_{i=0}^{r-1} a_i F^i(v) = 0\)

\begin{satz}[Caloy - Hamilton]
  \(F : v \to V\) Homomorphismus\\
  \(\chi_F(F)\) ist der 0 Endomorphismus auf V.
  \begin{proof}
    z.Z. ist dass \(\forall v \in V : \chi_F(F)(v) = 0\)\\
    \(v = 0 \Rightarrow \chi_F(F)(v) = 0\)\\
    Ansosnten \(v \neq 0\):\\
    \(\exists r : U =  = \Span(v, F(v), \dots, F^{r-1}(v))\) ist F-invariant.\\
    U - F-invariant \(\Rightarrow \chi_f = \xi_{F|_U} \cdot \xi_{F'} = \xi_{F'} \cdot \xi_{F|_U}\), \(F' : v/U \to V/U, \bar{w} \mapsto \bar{F(w)}\).\\
    Aufgabe:\\
    \(R(T) = P \cdot Q\\ % als Polynome
    R(F) = P \circ (Q(F))\) % als Abbildungen\\
    \(\chi_F(F)(v) = \xi_{F'} \circ (\xi_{F|_U}(v))\), wobei \(\xi_{F|_U}(v) = 0\), also \(\xi_F{F}(v) = 0\).
  \end{proof}
\end{satz}

\begin{korrolar}
  \(A \in M_{n \times n}(K)\\
  \chi_A(T) = T^n + \sum_{i=0}^{n-1}b_i T^i \Rightarrow A^n + \sum_{i=0}^{n-1} b_i \cdot A^i = 0\)
\end{korrolar}

\begin{satz}
  V endlichdimensionaler Vr, \(F V \to V\) Endormophismus, Dann existiert genau ein normiertes
  Plynom kleinsten Grades,  \(m_f\), derart, dass \(\forall P \in K[T] : m_F |_P \Leftrightarrow P(\tau) = 0\)
  Insbesondere gilt \(m_F(F) = 0\). Das  Polynom \(m_F\) hei\ss{}t das Minimalpolynom von F. 
\end{satz}

% missing stuff

\begin{satz}
  \(F : V \to V\) Endormorphismus, Dann existiert genau ein normiertes Polynom
  derart, dass \(\forall P \in K[T]\)  \(m_F|_P \Leftrightarrow P(F) = 0\).
  Das Polynom \(m_F(T)\) hei\ss{}t das Minimalpolynom von F.
  \begin{proof}
    Sei \(\mathcal{F} = \{P[T] \ text{normmiert} | P(F) = 0 \text{als Endomorphis.}\).\\
    Caley - Hamilton:
    \(Xi_F(T) \in \mathcal{F} \neq \emptyset\) Sei \(m-F(T) \in \mathcal{F}\)
    Polynom kleinsten Grades.\\
    Zu zeigen: \(\forall P \in K[T] | m_F|_P |leftrightarrow P(F) = 0\)\\
    \(\Rightarrow\)\\
    \(m_F |_P \Rightarrow \exists Q \in K[T] : P = Q \cdot m_F\)\\
    \(P(F) = Q(F) \circ m_F(F), m_F(F) = 0\\
     P(F) = 0\)\\
    \(\Leftarrow\)\\
    Sei \(P \in K[T], P(F) = 0\). Division mit Rest
    \(\Rightarrow \exists q, r \in K[T] : P = Q m_F + r, \Grad(r) < \Grad(m_F)\)
    \(0 = P(F) = Q(F) \circ m_F(F) + r(F) = r(F) = 0\)\\
    \(\Rightarrow r(F) = 0\) als Endomorphismus, \(\Rightarrow r = 0\).
    (sonst \(\frac{1}{a_{\Grad(T)}} \cdot r(T) \in \mathcal{F}\)\\
    Eindeutigkeit:\\
    Angenommen \((m_F'\) w\"urde auch die Bedingungen erf\"ullen\\
    \(m_F'|_P \Leftrightarrow P(F) = 0 \forall P \in K[T]\)\\
    \(\Leftrightarrow M_F|_{m_F'} \land m_F'|_{m_F}\)\\
    \(m_F = Q m_F' \land m_F' = H m_F\)\\
    zu Zeigen \(Q = H = 1\). Sowohl \(m_F, m_F'\) beide normiert, \(\Rightarrow
    Q, H\) sind normiert.\\
    \(m_F = Q \cdot m_F' = Q H m_F \land K[T] \) Integrit\"atsbereich\\
    \(\Rightarrow 1 = Q H\)\\
    \(\Grad(G H) = \Grad(G) \Grad(H) = \Grad(1) = 0 \land G, H \text{ normiert}\\
    \Rightarrow G = H = 1 \Rightarrow m_F = m_F'\).
  \end{proof}
\end{satz}

Frage\\
\(A \in \ve{0 & 1 \\ 0 & 0}
m|_A = \\
\chi_A(T) = T^2
m_A|_{T^2} \Rightarrow m_A = \begin{cases} T \\ T^2 \end{cases}
m_A(A) = 0 \Rightarrow m_A \neq T \Rightarrow m_A = T^2
\)

\begin{lemma}
  gegeben \(F : V \to V\), V endlich dimensional, F Endormorphismus, dann haben
  \(\chi_F\) und \(m_F\) dieselben Nullstellen in K.
  \begin{proof}
    \(m_F|_{\chi_F} \Rightarrow \xi_F = Q m_F \Rightarrow \forall \lb \in K\),
    falls \(m_F(\lb) = 0 \Rightarrow \chi_F(\lb) = 0\)\\
    Sei \(\lb \in K " \chi_F(\lb) = 0 \Rightarrow \lb \) ist ein Eigenwert von F
    \(\Rightarrow \exists v \in V \setminus \{0\} : F(v) = \lb v\). Sei \(m_F(T)
    = T^d + \sum_{i=0}^{d-1} c_i T^i\). \(0 = m_F(F)(v) = (F^d + \sum c_iF^I)(v)
    = F^d(v) + \sum c_i F^i(v)
    = \lb^d v + \sum c_i \lb^i v
    = (\lb^d + \sum c_i \lb^i) v
    = m_F(\lb) v
    \Rightarrow m_F(\lb) = 0\), da \(v \neq 0\).
  \end{proof}
\end{lemma}

\begin{satz}
  \(F : V \to V\), \(V\) endlichdimensional, F endomorphismus ist
  diagonalisierbar gw. \(m|_F\) in lauter paarweise veschiedene Linearfaktoren
  zerf\"allt.
  \begin{proof}\hfill\break
    \(\Rightarrow\) sei F diagonalisierbar. Dann gilt
    \(\chi_F(T) = \pi_{i=1}^k (T\lb_i)^{d_i}\), mit \(\lb_i\) Eigenwerte von F.
    Ausserdem ist \(V = \oplus_i V(\lb_i)\). Sei \(v \in V\) bel. Dann gilt 
    \(V = \sum_{i=1}^k V_i\) wobei \(v_i \in V(\lb_I)\).\\
    Setze \(P(T) = \pi_{i=1}^k (T -\lb_i)\). z.Z. ist \(P(T) = m_F(T)\).
    \(\\
    P(F)(v) = \pi_{i=1}^k(F-\lb_i)(v)\\
    = \pi_{i=1}^k(F-\lb_i)(\sum_{j=1}^k v_j)\\
    = \sum_{j=1}^k (\pi_{i=0}^k(F-\lb_i)(v_j))\\
    = \sum_{j=1}^k (F-\lb_1 Id) \circ \dots \circ (F-\lb_{j-1} Id) \circ (F-\lb_{j+1} Id) \circ \dots \circ (F -\lb_k Id) \circ (F -\lb_j Id)\\
    = 0
    \), da \(v_j \in V(\lb_j)\).\\
    v war beliebei \(\Rightarrow P(F) = 0 \in \End(V)\).\\
    Also \(m_F |_P \Rightarrow P(T) = Q(T) \cdot m_F(T)\)\\
    Aber \(m_F(T) = \pi_{i=1}^k(T-\lb_i)^{S_i}\), \(S_i \ge 1\) und 
    \(P(T) = \pi_{i=1}^k (T-\lb_i) = Q(T) \pi_{i=1}^k(T-\lb_i)^{S_i}\)\\
    \(\Rightarrow S_i=1\) f\"ur \(i=1, \dots, k\), sonst w\"are der Grad der
    rechten Seite gr\"o\ss{} als der Grad der linken.\\
    \(\Rightarrow m_F(T) = \pi_{i=1}^k(T-\lb_i)\).\\
    \(\Leftarrow\)\\
    Sei \(m_F(T) = (T-\lb_1) \circ \dots \circ (T-\lb_k)\), \(i \neq j \Rightarrow \lb_i \neq \lb_j\)\\
    z.Z. \(F\) ist diagonalisierbar. Es gen\"ugt zu Zeigen, dass
    \(V = \oplus_i V(\lb_i)\). Beweis durch Induktion:\\
    Sonderf\"alle:
    \begin{enumerate}
      \item{
        Sei \(\dim(V) = 1 \Rightarrow m_F(T) = (T-\lb_1) \Rightarrow \lb_1\)
        ist Eigenwert von F. Sei v der Eigenvektor zu \(\lb_1\) dann folgt
        \(V = \Span(v) = V(\lb_1)\)
      }
      \item {
        Sei \(\dim(V) = n \ge 2\) aber \(k=1\).
        \(\Rightarrow m_F(T) = (T - \lb_1)\). Nach Def von \(m_F\) gilt
        \(m_F(F) = 0 \Rightarrow F - \lb_1 = 0 \in \End(V)\). F\"ur alle
        \(v \in V\) gilt \((F - \lb_1)(v) = 0 \Rightarrow F(v) = \lb_1v
        \Rightarrow V = V(\lb_1)
        \)
      }
    \end{enumerate}
    Induktionsanahme: Der Satz gilt f\"ur den Fall \(\dim(V) < n, n \in \N\).
    Also sei \(\dim(V) = n \ge 2, k \ge 2\).\\
    Zun\"achst zeigen wir:
    Behauptung \((1)\): \(V = \Ker(F-\lb_1 Id) \oplus \Ima(F - \lb_1 Id)\).
    Idee: Zeige dass \(\Ker(F-\lb_1 Id) = V(\lb_1), \Ima(F -\lb_1Id) = V(\lb_1)\oplus \dots \oplus V(\lb_K)\)\\
    Bew.:\\
    Sei \(R(T) = \Pi_{i=2}^k(T-\lb_i) \Rightarrow m_F(T) = R(T)(T- \lb_1)\).\\
    Division mit Rest: \(\exists Q(T), r \in K\), so dass:\\
    \(R(T) = Q(T)(T - \lb_1)+r\)\\
    \(0 \neq R(\lb_1) = 0 + r \Rightarrow r \neq 0 \in K\)\\
    Sei \(v \in V\) bel, \(R(F)(v) = Q(F)(F-\lb_1)(v)  r \cdot v\)\\
    \(\Rightarrow v = \underset{\in \Ker(F-\lb_1)}{\frac{1}{r}R(F)} + \underset{\in \Ima(F-\lb_1)}{(F - \lb_1) \cdot (-\frac{1}{r}Q(F)(V))}\)\\
    \((F-\lb_1)(R(F)(v) = m_F(F) v = 0 \Rightarrow \frac{1}{r}R(F) \in \Ker(F-\lb_1)\)\\
    \(\Rightarrow V = \Ker(F-\lb_1) + \Ima(F-\lb_1)\).\\
    Zu zeigen ist nun noch, dass die Summe direkt ist. \(\Leftrightarrow \Ker(F-\lb_1)\cap\Ima(F-\lb_1) = \{0\}\).\\
    Sei \(v \in \Ker(F-\lb_1) \cap \Ima(F-\lb_1)
    \Rightarrow v = (F-\lb_1)(w), w \in V\\
    R(f)(v) = Q(F) (F-\lb Id)(v) + r \cdot v\\
    m_F(F) = (R(F) \cdot (F-\lb_1))(w) = \underset{=0}{Q(f)(f\lb_1Id)v} + rv
    \Rightarrow rv = 0, r \neq 0 \Rightarrow v = 0
    \Rightarrow V = \Ker(F-\lb_1 Id) \oplus (\Ima(F - \lb_1 Id)
    \)\\
    Setze \(W = \Ima(F - \lb_1)\) (W ist UVR  von V mit \(\dim(W) < \dim(V)\).\\
    Behauptung (2):  W ist F-invariant:\\
    Set \(v \in W\) beliebeig \(\Rightarrow v = (F - \lb_1)(w), w \in V
    \Rightarrow F(v) = F \circ (F - \lb_1)(w) = (F -\lb_1) \circ \underset{\in V}{(F(w))}
    \in W\)\\
    Setze \(F' = F|_W \in End(W)\).\\
    Beh. (3): \(m_{F'}(T) = \Pi_{i=2}^k(T - \lb_i)\) (= \(R(T)\)).\\
    Sei \(v \in W\) bel. \(\Rightarrow v = (F - \lb_1)(w), w\in V\\
    \Rightarrow R(F')(v) = \underset{m_F(F) = 0}{(R(F)\circ (F - \lb_1))}(w)\\
    \Rightarrow R(F') = 0 \in \End(W) \\
    \Rightarrow m_{F'}|_R\\
    \Rightarrow R(T) = H(T) \cdot m_{F'}(T)
    \)\\
    Es gilt auch \(0 = \underset{=0}{m_{F'}(F')} \circ (F-\lb_1)(v)\\
    \Rightarrow m_{F'}\circ (F-\lb_1) = 0 \in \End(V)\\
    \Rightarrow m_{F}(T)|_{m_{F'}(T) \cdot (T-\lb_1)}
    \Rightarrow m_{F'(T)}(T-\lb_1 = Q(T) \cdot (m_F(T))\\
     = Q(T) R(T)(T-\lb_1)\\
     \Rightarrow m_{F'}(T) = Q(T) \cdot R(T) = Q(T) H(T) m_{F'}(T)\\
     \Rightarrow Q(T) = H(T) = 1
    \), ansonsten w\"are der Grad links und rechts verschieden\\
    \(
      R(T) = m_{F'}(T)
    \)\\
    
    \(\overset{\text{IA}}{\Rightarrow} W = \oplus_{i=2}^kW(\lb_i)\),
    mit \(W(\lb_i)\) Eigenraum von \(F'\) zu \(\lb_i\).\\
    \(\Rightarrow V = V(\lb_1) \oplus_{i=2}^k W(\lb_i)\)\\
    z.Z. \(W(\lb_i) = V(\lb_i)\). Dann gilt \(V = \oplus_{i=1}^k(V(\lb_i))\).\\
    Es gilt \(
    W(\lb_i) = \{w \in W : F'(W) = \lb_I w\}\\
    \subseteq \{v \in V : F(v) \ \lb_i v\} = V(\lb_i)\)\\
    Sei \(i > 1\), \(v \in V(\lb_i)\). \(\Rightarrow F(v) = \lb_i v\).
    Setze \(w = \frac{1}{\lb_i - \l _1} v \in V(\lb_i)\).
    \(\Rightarrow F(w) - \lb_1w = v \Rightarrow v \in Im(F-\lb_1) = W\).\\
    \(\Rightarrow v \in \oplus_{j=2}^k(W(\lb_j) \Rightarrow v \in W(\lb_i)\)
    , da \(v  = \sum \al_j w_j\) f\"ur \(w_j \in W(\lb_j), \al_j \in K\)
    Aber die Summe der \(V(\lb_J)\) ist direkt und \(v \in V(\lb_i)\), damit
    kann nur \(\al_i\) nicht 0 sein, damit ist \(v = \al_i w_i \Rightarrow
    w \in W(\lb_i)\)
    
  \end{proof}
\end{satz}

\begin{bemerkung}
  \(A \in Mat_{n \times n}(K) : A^2 = Id_n \Rightarrow A\) ist diagonalisierbar.\\
  \(A^2 = Id_n \Rightarrow (T^ - 1)(A) = 0 \Rightarrow m_F|_{T^2 - 1}
  m_A = \begin{cases}T^2 - 1 = (T - 1)(T + 1) \\ T - 1 \\ T + 1\end{cases}
  \)
\end{bemerkung}

\section*{Jordansche Normalform}

\begin{lemma}
  \(F : V \to V\) Endomorphismus, \(U \subset V\), dann ist U F-invariant gdw.
  \(U (F - \lb Id\)-invariant ist \(\forall \lb \in K\).
  \begin{proof}
    \"Ubungsblatt 4
  \end{proof}
\end{lemma}

\textbf{Beispiel}\\
\(F : V \to V, F \neq 0\) Endomorphismus, derart, dass \(F^m = 0\) als
Endomorphismus und \(m > 0\) minimal (F ist nilpotent). \(F^{m-1} \neq 0
\Rightarrow v \in V : F^{m-1}(v) \neq 0\). \(\{v, F(v), \dots, F^{m-1}(v)\}\)
ist linear unabh\"angig. Erg\"anze sie zu einer Basis B von V.
\(B = \{v, F(v), \dots, F^{m-1}(v), \dotsm v_n\).\\
\(F^{m-1}(v), \dots F(v), v, \dots v_n\)\\
\(
  \begin{pmatrix}
    0      & 1      & 0      & \\
    \vdots & 0      & \ddots & *\\
    0      & \vdots & 0      & \\
    0      & \dots  &        & *
  \end{pmatrix}
\)

\begin{definition}[Hauptraum]
  Sei \(\lb \in K\) ein Eigenwert von \(F : V \to V\), V endlichdimens..\\
  \(V(\lb) = \Ker(F - \lb)\) Eigenraum von F bzgl \(\lb\).\\
  \(\Ker(F - \lb) \subset \ker(F - \lb)^2 \subset \dots\)\\
  \(\cup_{n \in \N} \ker(F - \lb)^n\) ist der Hauptraum von F bzgl. \(\lb\).
  \begin{bemerkung}
    \(V_\lb\) ist ein Unterraum und \(V_\lb\) ist F-invariant,
    da \(V_\lb\) \((F - \lb)\)-invariant ist.
  \end{bemerkung}
  \begin{bemerkung}
    Falls \(\Ker(F - \lb) = V(\lb) = 0\) ist
    \(\cup_{n \in \N}\Ker(F - \lb)^n = 0\).
  \end{bemerkung}
\end{definition}

\begin{lemma}
  Seien \(\lb_1 \dots \lb_k\) verschiedene Elemente aus K.
  \(V_{\lb_i} \cap \sum_{j=1 j\neq i}^k V_{\lb_j} = \{0\}\).
  \begin{proof}
    \(V_{\lb_j} = (F - \lb_j)\) invariant \(\Rightarrow V_{\lb_j}\) ist
    F-invariant \(\Rightarrow (F-\lb_i\) invariant.\\
    Wir wollen zuerst zeigen, dass  \(F - \lb_i \uparrow V_{\lb_j}\) ein
    Automorphismus ist falls \(i \neq j\).\\
    \(\Rightarrow\) es gen\"ugt zu zeigen, dass \(F - \lb_i\) injektiv auf
    \(V_{\lb_j}\) ist.\\
    Sei \(w \in V_{\lb_j} \setminus \{0\} \Rightarrow m \in \N\) kleinstm\"oglich,
    \(F - \lb_J)^m(w) = 0 \Rightarrow (F - \lb_j)^{m - 1}(w) \neq 0
    \Rightarrow ((F - \lb_j)^{m-1}((\lb_i - \lb_j)(w)) \neq 0\)\\
    Sei \(0 \neq (F - \lb)j)^{m-1}(F(w) - \lb_j w - (F(w) - \lb_i w))
    \Rightarrow (F - \lb_i)^m(w) + (F - \lb_j)^{m-1} \circ (F - \lb_i)(w)
    \Rightarrow (F- \lb_i)(w) \neq 0\).\\
    Insebsondere is jede Potez \(F - \lb_i^k\) ein Automorphismus von
    \(V_{\lb_j}\). \(V_{\lb_i} \cap \sum_{j=1, j\neq i}^k V_{\lb_j} = \{0\}\)
    \(v \in V_{\lb_i} \cap \sum_{j \neq i} V_{\lb_j}\\
    v = \sum_{j \neq i} v_j \in V_{\lb_j}
    \Rightarrow exist sm_j\) kleinstes \(F( - \lb_J)^{m_j}(v_j) = 0\)\\
    \((F - \lb_i)^{m_i} \circ \dots \circ (F - \lb_{i - 1}^{m}{i-1} \circ
    (F - \lb_{i-1}^{m_{i+1}} \dots (F - \lb_{k})^{m_k}\) ist eine Automorphismus
    von \(V_{\lb_i}\). Sie dieser automorph. H. \(H(v) = \sum_{j \neq i}(H(v_j)
    = \sum_{j \neq i} (F - \lb1)^{m_1} \circ \dots \circ (F - \lb_j)^{m_j}(v_j), F - \lb_j)^{m_j}(v_j) = 0
    \Rightarrow H(v) = 0 \Rightarrow v = 0\).
    \end{proof}
\end{lemma}

\begin{lemma}
  Sei \(\lb \in K\). \(\dim(V_\lb) = ord_\lb(\chi_F)\). Ferner hat \(F \uparrow V_\lb\)
  hat Matrixdarstellung der der Form \(
  \begin{pmatrix}
    \lb &        & X\\
        & \ddots & \\
    0   &        & \lb
  \end{pmatrix}
  \)\\
  \begin{proof}
    Sei \(k = \ord_\lb(\chi_F(T))\).
    \(\chi_F(T) = (T-\lb)^k \cdot G(T)\) wobei \(G(\lb) \neq 0\).\\
    Behauptung: es gibt einen F-invarianten Unterraum U von V der Dimension k
    so dass \(F \uparrow U\) hat Matrixdarstellung 
    \(
      \begin{pmatrix}
        \lb &        & X\\
            & \ddots & \\
        0   &        & \lb
      \end{pmatrix}
  \)\\
  Falls \(n=0 \Rightarrow\) klar. OBdA. ist \(k \ge 1\). Wir beweisen die
  Begauptung mit Induktion auf \(n = \dim(V)\).\\
  \(n = 1\)\\
  \(
    \Rightarrow \exists v \in V \setminus \{0\} : F(v) = \lb v\\
    \Rightarrow V = \Span(v) = U
  \)\\
  \(n \ge 2\)\\
  \(
    \Rightarrow \text{lambda is ein Eigenwert}\\
    \Rightarrow \exists v \in V \setminus \{0\} F(v) = \lb v, U_0 = \Span(v) \Rightarrow \dim(U) = 1 \land \text{F-invariant}
    \Rightarrow \hat{F} : V/U_0 \to V/U_0, \bar{W} \mapsto \bar{F(w)},
                (T-\lb)(T-\lb)^{k-1} \cdot Q(T) = \chi_F(T) = \underset{=(T-\lb)}{\chi_{F\uparrow U}(T)} \cdot \chi_{\hat{F}}(T)\\
    \Rightarrow \chi_{\hat{F}}(T) = (T - \lb)^{k-1} \cdot Q(T), Q(\lb) \neq 0\\
  \)
  Nach Ia. existiert ein \(\hat{F}\)-invarianter Unterraum \(\hat{U} \subset V/U_0\) der Dimension
  n-1, so dass die Matrixdartstellung von \(\hat{F}\) der Form 
  \(
      \begin{pmatrix}
        \lb &        & X\\
            & \ddots & \\
        0   &        & \lb
      \end{pmatrix}
  \)
  ist. Sei \(\bar{v_2}, \dots, \bar{v_k}\) eine Basis von \(\hat{U}\) so dass 
  die Darstellungsmatrix von \(\hat{F}\) bzgl. \(\{v_2, \dots, \bar{v}_k\}\)\\
  \(
      \begin{pmatrix}
        \lb &        & X\\
            & \ddots & \\
        0   &        & \lb
      \end{pmatrix}
  \)\\
  Die Vektoren \(\{v_1, v_2, \dots, v_k\}\) sind linear unabh\"angig.
  \(U = \Span(v_1, v_2, \dots, v_k)\) hat Dimension k. U ist F-invariant,
  \(
    F(\sum_{i=1}^k \mu_I v_i)
    = \sum \mu_i F(v_i)
    = \mu_i \lb v_i + \sum_{i=2}^k \mu_i \underset{=\sum_{j\le i} a_{ji}v_i}{F(v_i)}
    \in U
  \)\\
  Zu zeigen ist nun dass \(U = V_\lb\).
  \(
    U \subset V_\lb:
    \chi_{F \uparrow U} = (T - \lb)^k
    \overset{\text{Caley-Hamilton}}{\Rightarrow} (F - \lb)^k = 0 \text{ auf U}\\
    \Rightarrow U \subseteq \Ker(F - \lb)^k\\
    \)
    Falls \(U \subsetneq V_\lb\)\\
    \(\Rightarrow \dim(V_\lb) \ge 1\).\\
    \(
      (T-\lb)^K \cdot G(T) = \chi_F(T) = \chi_{F\uparrow U}(T) \cdot \chi_{\hat{F}}(T)
      \Rightarrow \chi_{hat{F}}(T) = G(T)
    \) aber \(G(\lb) \neq 0\).\\
    \(\lb \) ist kein Eigenwert von \(\hat{F}\).
    \(\Rightarrow\) der Hauptraum von \(\hat{F}: V/u \to V/u\) ist trivial.\\
    Sei \(w \in V_\lb\)\\
    \(\Rightarrow \exists s \in N : (F - \lb)^s(w) = 0 \Rightarrow (F - \lb)^s(\bar(w)) = 0\\
    \Rightarrow \bar{w} = 0 \Rightarrow w \in U\).
  \end{proof}
\end{lemma}

\begin{definition}
  Ein Endormorphismus \(F : V \to V\) hei\ss{}t bilpotent, falls es eine nat.
  Zahl m gibt, so dass \(F \circ \dots \circ F = F^m = 0\) auf V ist.
\end{definition}

\begin{lemma}
  Sei \(F ; V \to V\). Folgende Ausagen sind \"aquivalent:
  \begin{enumerate}
    \item{F ist nilpotent}
    \item{\(\forall c \in V \ Exists m_v \in \N F^{m_v}(v) = 0\)}
    \item{Es existiert eine Basis von V, so dass F Darstellungsmatrix
    \(
      \begin{pmatrix}
        0 & & x\\
          & \ddots &\\
          0 & & 0
      \end{pmatrix}
    \)
    }
    \item{\(\chi_F(T) = T^n\)}
  \end{enumerate}
  \begin{proof}
  \begin{itemize}
    \item[\(1\Rightarrow 2\)]{
      trivial.
    }
    \item[\(2 \Rightarrow 3\)] {
      Induktion auf \(n = \dim(V)\). Sei \(v \in V \setminus\{0\}\\
      \Rightarrow\exists m_V \in \N\) kleinztest \(F^{m_v}(v) = 0\\
      \Rightarrow m_v \neq 0\\
      \Rightarrow f_{m_v-1}(v) =neq 0\\
      \Rightarrow V = \Span(F^{m-1}(v))\\
      \)
      Ferner\\
      \(F(F^{m_v-1}(v)) = 0\)\\
      \(\Rightarrow\) Darstellungsmatrix bzg. \(\{F^{m_v-1}(v)\}\) ist \((0)\).\\
      \(n \ge 2\)\\
      Sei \(v_1 \in V \setminus \{0\}\) so dass \(F(v_1) = 0\)\\
      \(\Rightarrow U = \Span(v_1)\) ist F-invariant.\\
      \(\Rightarrow \hat{f} = \underset{\dim = n-1}{V/U} \to V/U\)\\
      ach I.A. existiert eine Basis \(\{\bar{v}_2, \dots, \bar{v}_n\}\), so dass
      die Darstellungsmatrix von \(\hat{F}\) die Form\\
      \(
            \begin{pmatrix}
        0 & & x\\
          & \ddots &\\
          0 & & 0
      \end{pmatrix}
      \) hat.\\
      Die Familie \(\{v_1, v_2, \dots, v_n\}\) ist eine Familie von V und F hat
      Darstellungsmatrix:
      \(
      \begin{pmatrix}
        0 & & x\\
        & \ddots &\\
        0 & & 0
      \end{pmatrix}
      \)
    }
    \item[\(3 \Rightarrow 4\)] {
      \(\chi_F(T) = det(T \cdot Id_B - (\ve{0&&X \\ & \ddots\\ 0&& 0})
      = \det(\ve{T&&X \\ & \ddots\\ 0&& T} = T^n\)
    }
    \item[\(4 \Rightarrow 1\)] {
      Caley-Hamilton: \(F^n = \chi_F(F)= 0\)
    }
  \end{itemize}
  \end{proof}
\end{lemma}

\begin{satz}[Jordan-Charvallery]
  Sei \(F : V \to V\) Endomorphismus, V endl. dim. Fakks \(\chi_F(T)\) in
  Linearfaktoren zerf\"allt dann ist \(V = \oplus_{i}^k V_{\lb_i}\) mit \(\lb_i\)
  verschiedene Eigenwerte. F hat dann eine Blockmatrix als Darstellungsmatrix:\\
  \(
  \begin{pmatrix}
  A_1\\
      & A_2\\
      &       &\ddots\\
      &&&A_k\\
  \end{pmatrix}
  \)
  \(A_i = \begin{pmatrix}
    \lb_i&&X\\
    &\ddots&\\
    0&&\lb_i\\
  \end{pmatrix}\)
  
  Insbseondere ist \(F = G + H\) mit G ist diagonalisierbar und H ist nilpotent,
  und \(G \cdot H = H \cdot G\).
  \begin{proof}
    \(\chi_F(T) = \pi_{i=1}^k (T-\lb_i)^{d_i}\), \(\lb_i\) verschieden.\\
    \(\dim(V) = \Grad(\chi_F(T)) = n = \sum){i=1}^k _underset{=\dim(V_{\lb_i}}{d_i}\).\\
    \(\dim(\sum_{i=1}^kV_{\lb_I}) = \sum d_i = b = \dim(V)\) da die Hauptr\"aume transversal sind.\\
    \(\Rightarrow V = \oplus_{i=1}^k V_{\lb_i}\)\\
    V besitzt eine Bais B, welche aus der Vereinigung der Basen der \(v_{\lb_i}\)
    besteht. F wird durch \(F \uparrow V_{\lb_1}, \dots, F\uparrow V_{\lb_K}\)
    bestimmt. Die einzelnen \(F \uparrow V_{\lb_I}\) habe Matrixdarstellungen
    \(A_I = \begin{pmatrix}
      \lb_i&&X\\
      &\ddots\\
      0&&\lb_I\\
    \end{pmatrix}\)
    Die Darstellungsmatrix von F ist dann \\
    \(
      \begin{pmatrix}
      A_1\\
          & A_2\\
          &       &\ddots\\
          &&&A_k\\
      \end{pmatrix}
    \).\\
    Sei \(G : V \to V\) \(G \uparrow V_{\lb_I} = \) Multiplkation mit \(\lb_i\).
    G hat diagonale Matrixdarstellung zgl. der Basis B :
    \(
    \begin{pmatrix}
      \lb_1\\
      &\ddots\\
      &&\lb_1\\
      &&&\ddots\\
      &&&&\lb_k\\
      &&&&&\ddots\\
      &&&&&&\lb_k
    \end{pmatrix}\\
    \)
    Setze \(H = F - G\). Die Darstellungsmatrix von H bzgl. B ist\\
    \(
    \begin{pmatrix}
      0&&X\\
      &\ddots\\
      0&&0\\
    \end{pmatrix} \Rightarrow
    \) H ist nilpotent.\\
    z. Zeigen \(G H = H G\)\\
    Beachte, dass jedes \(V_{\lb_i}\) G-invariant ist. Damit ist \(V_{\lb_i}\)
    H-invariant.\\
    Es gen\"ugt zu zeigen, dass \(G H = H G\) auf \(V_{\lb_i}\). Sei
    \(w \in V_{\lb_i}\): \(H(G(W)) = H(\lb_i w) = \lb_i H(w) = G(H(w))\).
  \end{proof}
\end{satz}

% Missing beginning of lecture on the 29.05.2018
% Jordanblock

\begin{definition}
  \(F : V \to V\), V endlichdimensional. Eine Basis \(\{v_1, \dots, v_n\}\) ist 
  F-adaptiert falls \(F(v_1) = 0, F(v_j) = \begin{cases}
  0\\
  v_{j-1}
  \end{cases}\)
  \textbf{Bemerkung:}\\
  Falls V eine F-adaptierte Basis hat ist \(F : V \to V\) nilpotent.
  \begin{proof}
    Sei \(\{v_1, \dots, v_n\}\) F-adapteoert. Es gen\"ugt zu zeigen dass
    \(F^n = 0\).\\
    \(\sum_j^n \mu_jv_j = F^n(v) = \sum_{j=1}^n \lb_j F^n(v_j)\)
  \end{proof}
\end{definition}

Notation \(N_m = \begin{pmatrix}
0&&X\\
&\ddots\\
0&&0\\
\end{pmatrix}
\in Mat_{n \times n}\\
N_1 = (0)\\
N_m^m = 0
\)

\begin{definition}
  \(F : V \to V\) nilpotent. Es existiert \(m \in \N\)
  \(\Ker(F) \subset \dots \subset \Ker(F^m) = \Ker(F^{m+1}) = \dots \subset V\).
  m hei\ss{}t der Index von F. \(V = \Ker(F^m) \oplus F^m(V)\)
\end{definition}

\begin{satz}
  Sei \(F ; V \to V\) undd B eine F-adaptierte Basis von V. Dann hat F
  Matrixdarstellung der Form:\\
  \(
  \begin{pmatrix}
    N_{k_1}\\
    &\ddots\\
    &&N_{k_n}\\
  \end{pmatrix}\\
  \sum_{j=1}^r k_j = n\\
  Index(F) = max\{k_j\}_{1 \le j \le n}
  \)
  \begin{proof}
    Sei \(i_1 < \dots < i_r\) eine Aufz\"ahlung der Menge von Indices so dass 
    \(\{j | F(v_j) = 0\}\)\\
    \(v_1, v_2, \dots, v_{i_1}, v_{i_1 + 1}, \dots\)\\
    \(
    \begin{pmatrix}
    0&1\\
     &0&1\\
     & &\ddots&1\\
     &&&0&0\\
     &&&&0&1\\
     &&&&&0 & 1\\
     &&&&&&\ddots\\
    \end{pmatrix}
    \)
  \end{proof}
\end{satz}

\begin{satz}
  Sei \(F : V \to V\) nilpotent mit Index K, \(\dim(V) = n < \infty\).
  Gegeben ein Unterraum U von V derart, dass \(U \cap \Ker(F^{k-1}) = \{0\}\).
  Dann l\"asst sich jede Basis von U zu einer F-adaptierten Basis von V
  erg\"anzen.\\
  \begin{bemerkung}
    \(\Ker(F) \subsetneq Ker(F^2) \subsetneq \Ker(F^{k-1}) \subsetneq \Ker(F^k) =
    \dots = V\).\\
    \(\Ker(F^{k-1}\) hat ein Komplement U in \(\Ker(F^k)\)
  \end{bemerkung}
  
  \begin{proof}
    Induktion auf k:\\
    \(k = 1\)\\
    \(F = 0\) als Endomorphismus. \(\Rightarrow\) Jede Basis ist F-adaptiert.\\
    \(k \ge 2\):\\
    Sei \(V' = \Ker(F^{k-1}) \subsetneq V\). Sei weiterhin \(\{u_1, \dots, u_s\}\)
    eine Basis von U. \(U \cap V' = \{0\}\). \(U + V' = U \oplus U \oplus V'\)
    hat eine Basis: \(\{u_1, \dots, u_s, \hat v_1, \dots, \hat v_m\}\).
    Erg\"anze diese Basis zu einer Basis von V:\\
    \(\{u_1, \dots, u_s, \dots u_r, v_1, \dots, v_m\}\).\\
    Sei \(W = \Span(u, \dots, u_r), U \subset W, W \cap V' = \{0\}\).\\
    \(F(W) \subset \Ker(F^{k-1}) = V'\), da \(F^k = 0\). \(V'\) ist F-invariant
    und \(F \uparrow V'\) hat Index \(k-1\).\\
    \textbf{Beh.}:  \(\{0\} = F(W) \cap \Ker(F^{k-2})\)\\
    Sei \(u \in F(w) \cap \Ker(F^{k-2}) \Rightarrow F^{k-2}(u) = 0\). es existiert
    \(w \in W : u = F(W)\)\\
    \(0 = F^{k-2}(u) = F^{k-1}(w)\\
    \Rightarrow w \in \Ker(F^{k-1} = V'\\
    \Rightarrow w = 0\\
    \Rightarrow u = F(w) = 0\)\\
    \textbf{Beh.}: \(\{F(u_1), \dots, F(u_r)\}\) sind linear unabh\"angig. also
    eine Basis von \(F(W)\).\\
    \(\sum_{i=1}^r \lb_i F(u_i) = 0
    = F(\sum_{i=1}^k \lb_i F(u_i))\\
    \sum_{i=1}^k \lb_i u_i \in \Ker(F) \subset V'\\
    \overset{W \cap V' = \{0\}}{\Rightarrow} \sum_{i=1}^k \lb_i u_i = 0\\
    \Rightarrow \lb_1 = \dots = \lb_r = 0
    \)\\
    Aus unserer Induktionsanahme: \(F(W)\) und \(F |_{V'}\) \(\Rightarrow\)
    Es existiert eine adaptierte Basis \(\{v_1', \dots, v_m'\}\) welche
    \(\{F(u_1), \dots, F(u_r)\}\) erg\"anzt.\\
    \(F(u_j) = v_{i_j}'\) wobei \(i_1 < \dots < i_r\) (ansonsten sortiere
    \(u_j\) um.\\
    \(v_1 = v_1'\)\\
    \(v_2 = v_2'\)\\
    \(\vdots\\
    v_{i_1} = v_{i_1}'\\
    v_{i_1+1} = u_1\\
    v_{i_1+2} = v_{i_1 + 1}\\
    v_{i_1+} = v_{i_1 + 1}\\
    \)\\
    (F\"uge an den stellen \(i_j\) den vektor \(u_j\) ein).\\
    \(\rightarrow\) Basis von V.\\
    Es gen\"ugt nun zu zeigen, dass diese Basis F-adaptiert ist.\\
    z.B. \(F(v_{i_1+1}) = v_{i_1}' = F(u_1)\\
    \Rightarrow u_1 - v_i \in \Ker(F) \land u_1 - v_i \in V' \Rightarrow \) Widerspruch. 
  \end{proof}
\end{satz}

\textbf{Beispiel}\\
\(A =
\begin{pmatrix}
1  & 1 & 0 & 1\\
0  & 2 & 0 & 0\\
-1 & 1 & 2 & 1\\
-1 & 1 & 0 & 3
\end{pmatrix}\\
\chi_A(T) = det(\begin{pmatrix}
T - 1  & 1 & 0 & 1\\
0  & T - 2 & 0 & 0\\
-1 & 1 & T - 2 & 1\\
-1 & 1 & 0 & T - 3
\end{pmatrix}) = (T-2) det(\ve{T-1&0&-1\\1&T-2&-1\\1&0&T-3})
= (T-2)^4 \Rightarrow\) 2 ist der einzige Eigenwert\\
\(A - 2Id\) ist nilpotent.\\
\(A - 2Id = \begin{pmatrix}
-1  & 1 & 0 & 1\\
0  & 0 & 0 & 0\\
-1 & 1 & 0 & 1\\
-1 & 1 & 0 & 1
\end{pmatrix}\)\\
\(\Ker(A - 2Id) = \{(x_1, x_2, x_3, x_4) | -x_1 + x_2 + x_4 = 0\}\)\\
\(
\Ker(A - 2Id) \subsetneq \Ker((A - 2Id)^2), \dim(\Ker(A - 2Id)) = 4\)\\
Die Jordansche Normalform ist dann:
\(
\begin{pmatrix}
2 & 1\\
  & 2\\
  &   & 2\\
  &   &   & 2\\
  &   &   &   & 2
\end{pmatrix}
\)
Da es nur einen Vektor in \(\Ker((A - 2Id)^2)\) gibt der nicht in \(\Ker((A - 2Id))\) ist.

\begin{korrolar}
  Jeder Nilpontente Endormorphsimus besitzt eine F-adaptierte Basis.
\end{korrolar}

\begin{korrolar}
  Jede niplotente Endormorphismus l\"a\ss{}t sich bez\"uglich einer geeigneten
  Basis durch eine Blockmatrix darstellen.
  \(
  \begin{pmatrix}
    0&1\\
    &0&1\\
    & &\ddots&1\\
    &&&0&0\\
    &&&&0&1\\
    &&&&&0 & 1\\
    &&&&&&\ddots\\
  \end{pmatrix}\)
\end{korrolar}

\begin{korrolar}[Jordansche Normalform]
  Jeder Endomorphismus eines endlichdimensionalen Vektorraumes dessen
  charakteristisches Polynom in Linearfaktoren zerf\"allt l\"asst sich
  bez\"uglich einer geeigneten Basis durch eine Blockmatrix folgender Form
  darstellen.\\
  \(
  \left (
  \begin{array}{cccccccccccc}
    \lb_1&1\\
    &\lb_1&1\\
    & &\ddots&1\\
    &&&\lb_1&0\\
    &&&&\lb_2&1\\
    &&&&&\lb_2& 1\\
    &&&&&&\ddots\\
    &&&&&&&\lb_k & 1\\
    &&&&&&&&\ddots & 1\\
    &&&&&&&&&\lb_k & 1
  \end{array}
  \right )
  \)
\end{korrolar}

\chapter{Dualit\"at}

\begin{definition}
Sei V ein K-Vektorraum. Der Dualraum \(V*\) ist die Kollektion aller linearen
Abbildungen von \(V \to K\)
\begin{bemerkung}
\(V*\) ist ein K-Vektorraum.
\(F + G : V \to K, v \mapsto  F(V) + G(V)\) ist linear,\\
\(\lb \in K L \lb F : V \to K, V \mapsto \lb F(v)\) ist auch linear
\end{bemerkung}
\end{definition}

\begin{definition}
Sei V endlichdimensional und w\"ahle iene Basis \(B \ \{b_1, \dots, b_n\}\) von
V. Die duale Basis \(B* = \{b_1*, \dots, b_n*\) ist eine lineare Abbildung
derart, dass \(b*(b_j) = 1_{ij} = \begin{cases}
1 & i = j\\
0 & \text{ansosnten}
\end{cases}\)\\
Insbesondere \(b_i*(V) = b_i* (\sum_{j=1}^n \lb_j b_j) = \lb_i\)
\begin{bemerkung}
  Falls V endlichdimensional ist, dann ist \(B*\) eine Basis von \(V*\) und somit
  \(V \simeq V*\).
  \begin{proof}
    \(b_1*, \dots, b_n*\) sind linear unabh\"angig.\\
    1.)\\
    \(
      \sum \lb_i b_i* = 0\\
      \sum \lb_i b_i*(b_j) = 0, 1 \le i \le n
      \Rightarrow \lb_i = 0
    \)
    2.)\\
    \(
      \Span(b_1*, \dots, b_n*) = V*
    \)
    Sei \(F : V \to K\) beliebig. \(F(b_i) = \lb_i \in K\)\\
    \(
    F - \sum \lb_i b_i \overset{!}{=} 0\\
    (F - \sum \lb_i b_i*)(b_j) = 0 = F(b_j) - \sum \lb_i b_i*(b_j)\)
    Insbesondere ist \(V \to V*, b_i \mapsto b_i*\) ein Isomorphismus.\\
    Allerdings: Der Isomorphismus \(V \simeq V*\) h\"angt von der Wahl der Basis
    B ab, ist also nicht kanonisch.
  \end{proof}
\end{bemerkung}
\end{definition}

\begin{lemma}[kanonischer Monomorhismus]
 \(V \to (V^*)^*, v \mapsto \varphi_V : V^* \to K, F \mapsto F(V)\)\\
 \begin{proof}
    \(\varphi_V\) ist wohldefiniert.\\
    \(\varphi_V(F + G) = (F + G)(v) = (F(V) + G(V)\)\\
    zu zeigen: \(\varphi_V\) ist injektiv. (\"Ubungsaufgabe).
 \end{proof}
\end{lemma}

\begin{korrolar}
  Falls V endlichdimensional ist sind \(V \simeq (V^*)^*\) kanonisch isomorph.\\
  \begin{proof}
    \(\dim(V) = \dim(V^*) = \dim(V^*)^*
    \Rightarrow \varphi : V \to (V^*)^*\) ist surjektiv also auch ein
    Isomorphismus.
  \end{proof}
\end{korrolar}

\begin{lemma}
  Sei V endlichdimensional. W\"ahle Basen \(B = \{b_1, \dots, b_n\}\) und
  \(B' = \{b_1', \dots, b_n'\}\) von V. Seien \(B^*\) und \(B'^*\) die
  entsprechenden dualen Basen in \(V^*\). Wenn A die Transformationsmatrix
  von B nach B' ist, dann ist die Transformationsmatrixvon \(B^*\) nach
  \((B')^*\) \(A^*){^-1}\)\\
  \begin{proof}
    \(\ve{b_1' \\ \vdots \\ b_n'} = A\ve{b_1 \\ \vdots \\ b_n}\)\\
    Sei X die Transformationsmatrix von \(B^*\) nach \((B')^*\).\\
    \(ve{b_1 \\ \vdots \\ b_n} \ve{b_1^*& \dots& b_n^*} = Id_n\)\\
    \(\ve{b_1' \\ \vdots \\ b_n'} \ve{(b_1^*)' & \dots & (b_n^*)'} = Id_n\)\\
    \(Id_n = \ve{(b_1^*)' \\ \vdots \\ (b_n^*)'} \ve{b_1' & \dots & b_n'}
    = X \ve{(b_1^*)' \\ \vdots \\ (b_n^*)'} (A \cdot \ve{b_1 \\ \vdots \\ b_n})^T
    = X \ve{(b_1^*)' \\ \vdots \\ (b_n^*)'} \ve{b_1 & \dots & b_n} A^T
    = X  \Id_n A^T
    \Rightarrow X = (A^*)^{-1}
    \)
  \end{proof}
\end{lemma}

\begin{definition}
  Sei \(F : V \to W\) eine lineare Abbildung. Definiere die duale Abbildung
  \(F^* : W^* \to V^* : \psi \mapsto \psi \circ F\).
  \begin{bemerkung}
    \(F*\) ist linear.\\
    \(
      F*(\psi_1 + \psi_2) = (\psi_1) + \psi_2) \circ F
      = \psi_1 \circ F + \psi_2 \circ F
    \)\\
    \(U \overset{G}{\to} V \overset{F}{\to} W\\
    W^* \overset{F^*}{\to} V^* \overset{G^*}{\to}{U^*}\\
    (F \circ G)^* = G^* \circ F^*\)
    \begin{proof}
      \(\psi \in W^*\\
      (F\circ G)^*(\psi) = \psi \circ F \circ G = F^*(\psi) \circ G = G^*(F^*(\psi))\)
    \end{proof}
  \end{bemerkung}
  Eigenschaften:
  \begin{enumerate}
    \item{\(Id_V^* = Id_{V^*}\)}
    \item{\((F + G)^* = F^* + G^*\)}
    \item{\((F \circ G)^* = f^* \circ F^*)\)}
    \item{\(\mu F)^* = \mu F^*\)}
  \end{enumerate}
\end{definition}

\begin{lemma}
  Falls \(F^* = 0\) dann ist \(F = 0\). Falls V und W endlich sind,
  \(G : W^* \to V^*\) lineare Abbildung, dann gibt es \(F : V \to W\) so dass
  \(F^* = G\).
  \begin{proof}
    \(F^ = 0\) Sei \(v \in V\) beliebig. Zu zeigen: \(F(v) = 0\)\\
    Definiere:\\
    \(W^* \to K : \psi \to \psi(F(v))\)\\
    Erinnerung: \(W \leftrightarrow (W^*)^* : w \mapsto \psi_w : W^* \to K : \psi \mapsto \psi(w)\)\\
    \(\varphi_{F(v)}(\psi) = \psi(F(v)) = F^*(\psi)(v) = 0\)\\
    Abre \(\varphi\) ist ein Monomorphismus\\
    \(\Rightarrow F(v) = 0 \Rightarrow F = 0\).
  \end{proof}
\end{lemma}

\begin{definition}
  Die Operation \(* : Hom(V ,W) \to Hom(W^*, V^*)\)\\
  Falls V, W endlichdim. sind.\\
  \(\dim(Hom(V, W)) = \dim(V) \cdot \dim(W)\)\\
  \(\dim(Hom(W^*, V^*)) = \dim(V^*) \cdot \dim(W^*)\)
  Also ist die Operation \(*\) injektiv \(\Rightarrow\) surjektiv.
\end{definition}

\begin{bemerkung}
  \(F : V \to W\), \(\dim(V) = n, \dim(w) = m\) hat die Darstellungsmatrix A
  bzgl. der Basis \(B = \{b_1, \dots, b_n\}\) von V und
  \(B' = \{b_1', \dots, b_n'\}\) von W. Seien \(B^*\) und \((B')^*\) die
  entsprechenden dualen Basen von \(V^*\) und \(W^*\).\\
  Dann hat \(F^*\) die Darstellungsmatrix \(A^T\) bezgl \((B')^*\) und \(B^*\).
  \begin{proof}\hfill\break
  \(F \ve{x_1 \\ \vdots \\ x_n} = A \ve{x_1 \\ \vdots \\ x_n} = A x\\
  F^* \ve{\psi_1 \\ \vdots \\ \psi_m} \ve {x_1 & \dots & x_n} = 
  \ve{\psi_1 & \dots & \psi_n} \cdot F \ve{x_1 \\ \vdots \\ x_n} = A x
  \)\\
  Die Darstellungsmatrix von \(F^*\) ist
  \(F^* \ve{\psi_1 \\ \vdots \\ \psi_m} = \ve{\psi_1 & \dots & \psi_m} \cdot A\\
  F^* \ve{\psi_1 \\ \vdots \\ \psi_m} = (A^T \cdot \ve{\psi_1 \\ \vdots \\ \psi_m})^T
  \)
  \(\Rightarrow\) die Darsstellungsmatrix von \(F^*\) ist \(A^T\).
  \end{proof}\\
  Alternativer Beweis:\\
  \begin{proof}\hfill\break
    \(F^*(c_k^*) = \sum_l \lb_{l_k} b_l^*\\
    F(b_i) = \sum_{j=1}^n a_{ji} c_j\\
    F^*(C_k^*)(b_i) = (\sum\lb_{l_k}b_l^*)(b_i) = \lb_{i_k}\\
    = c_k^*(F(b_i)) = c_k^*(\sum a_{ji} c_j) = a_{ki}\\
    \Rightarrow A^T
    \) ist die Darstellungsmatrix on \(F^*\).
  \end{proof}
\end{bemerkung}

\begin{korrolar}
  \(\det(F^*) = \det(F)\).
\end{korrolar}

\begin{lemma}
  \(V = \oplus_{i=1}^n V_i \Rightarrow V^* \simeq \oplus_{i=1}^n V_i^*\)\\
  Der Beweis ist trivial: \(V \to K\) ist eindeutig bestimmt durch \(F |_{V_1},
  \dots, F|_{V_n}\).
\end{lemma}

\begin{lemma}
  \(F : V \to W\) linear.
  \begin{enumerate}
    \item{F injektiv \(\Leftrightarrow\) \(F^*\) surjektiv}
    \item{F surjektiv \(\Leftrightarrow F^*\) inj }
    \item{F Isom \(\Leftrightarrow F^*\) Isom.}
  \end{enumerate}
  \begin{proof}
    \begin{enumerate}
      \item {
        \(F : V \to W\) injektiv. z.Zeigen \(F^* : w^* \to v^*\) surjektiv:
        Es existiert ein \(\Theta : W \to k :
        F^*(\Theta) = \psi \in V^*, \forall v \in V \Theta(F(V)) = \psi(v)\)\\
        \(V \simeq Im(F) \subset W\). Sei Z ein Komplement von \(\Ima(F)\) in W.
        \(W = \Ima(F) \oplus Z\). \(\forall w \in W : w = w' + \hat w, w_1 = F(v),
        \hat w \in Z\).\\
        Definiere \(\Theta : W \to K : w = F(v) + \hat w \mapsto \psi(v)\).
        \(\Theta\) ist wohldefiniert. Zu zeigen ist nun noch, dass \(\Theta\)
        linear ist:\\
        \(\Theta(w_1 + w_2) = \Theta(F(v_1) + F(v_2) + \hat w_1 + \hat w_2)
        = \psi(v_1 + v_2) = \psi(v_1) + \psi(v_2) = \Theta(w_1) + \Theta(w_2)\)\\
        \(\Leftarrow\)\\
        \(F^* : W^* \to V^*\) surjektiv. Zu zeigen \(F : V \to W\) ist unjektiv.
        \(v \in V : F(v) = 0 \Rightarrow v = 0\).\\
        Falls \(v \neq 0 \Rightarrow \exists \psi : V \to K : \psi(v) \neq 0\)\\
        \(\Rightarrow \exists \Theta : W \to K : F^*(\Theta) = \psi \Rightarrow
        \underset{=0}{\Theta(F(v))} = \underset{\neq 0}{\psi(v)}\)
      }
      \item {
        \(\Rightarrow\)\\
        \(F : V \to W\) surjektiv. Zu zeigen: \(F^* : W^* \leftrightarrow V^*\)
        injektiv.\\
        Sei \(\Theta \in W^* : F^*(\Theta) = 0\) als lineare abbildung.
        \(F^*(\Theta) = \Theta \circ F : V \ to K\). Zu zeigen:
        \(\forall w\in W : \Theta(w) = 0\).\\
        \(w \in W : \exists v \in V : F(v) = w\). F surjektiv
        \(\Theta(w) = \Theta(F(V)) = F(\Theta)(v)\).\\
        \(\Leftarrow\)
        \(F^* : V^* \to W^*\) ist injektiv. z.Z. \(F\) ist surjektiv.\\
        Sei Z ein Komplement von \(\Ima(F)\) in W : \(W = \Ima(F) \oplus Z\).
        Zu zeigen ist nun, dass \(Z = \{0\}\). Sonst sei B eine Basis von Z.
        \(G : W \to K\) linear derart, dass \(G |_{\Ima(F)} = 0\) und
        \(\forall b \in B : G(b) = 1\). \(\Rightarrow G \in W^* \Rightarrow
        F^*(G) = G \circ F : V \to K\). Sei
        \(v \in V : G \circ F(v) = G(F(v)) = 0\). \(F^*(G)\) ist die
        triviale Abbildung \(\Rightarrow G = 0 \Rightarrow B = \emptyset
        \Rightarrow Z = \{0\}\)
      }
      
    \end{enumerate}
  \end{proof}
\end{lemma}

\chapter{Duale Paarungen}
\begin{definition}[Bilinearit\"at]
  Seien V, W K-Vektorraeume. Eine Abbildung \(\varphi : V \times W \to K\) ist
  bilineare wenn \(\varphi\) linear in jeder Koordinate ist.
  \begin{enumerate}
    \item{
      \(\varphi(v, \lb_1 w_1 + \lb_2 w_2)
      = \lb_1 \varphi(v, w_1) + \lb_2 \varphi(v, w_2)\)
    }
    \item{
      \(\varphi(\lb_1 w_1 + \lb_2 w_2, v)
      = \lb_1 \varphi(w_1, v) + \lb_2 \varphi(w_2, v)\)
    }
  \end{enumerate}
\end{definition}

\begin{bemerkung}
  Falls V, W endlichdimensional mit Basis \(B = \{b_1, \dots, b_n\}\) von V,
  \(C = \{c_1, \dots, c_m\}\) von W, dann ist \(\varphi\) eindeutig bestimmt
  durch die \((n \times m\) Matrix A = \(\varphi(b_i, c_j)\).\\
  \(\varphi(v, w) = \varphi(\sum_{i\le n}\lb_i b_i, \sum_{j \le m}\mu_j c_j)
  = \sum_{i \le n} \lb_i \varphi(b_i, \sum_{j \le m}\mu_j c_j)
  = \sum_{i \le n} \lb_i \sum_{j \le n} \mu_i \varphi(b_i, c_j)
  = (\lb_1, \dots, \lb_n) A \ve{\mu_1 \\ \vdots \\ \mu_m}
  \)
\end{bemerkung}

\begin{bemerkung}
  Falls wir Basen \(B'\) von V und \(C'\) von W gew\"ahlt h\"atten, dann ist 
  die Darstellungsmatrix von \(\varphi\) : \(M(B, B')^T \cdot A \cdot M(C, C')\)
  \(\ve{\lb_1 \\ \vdots \\ \lb_n} = M(B, B') \ve{\lb_1' \\ \vdots \\ \lb_n'}\)\\
  \(\ve{\mu_1 \\ \vdots \\ \mu_n} = M(C, C') \ve{\mu_1' \\ \vdots \\ \mu_n'}\)\\
  \((\lb_1, \dots, \lb_n) = (\lb_1', \dots, \lb_n') M(B, B')^T\)
\end{bemerkung}

\begin{korrolar}
  Der Rang h\"angt nicht von der Auswahl der Basen ab. Somit ist der Rang von
  \(\varphi\) (\(\rg(\varphi)\)) wohldefiniert.
\end{korrolar}

\begin{definition}[Duales Paar]
  Ein Tupel \((V, W, \varphi)\) ist ein duales Paar, falls:
  \begin{enumerate}
    \item{\(\dim(V) < \infty, \dim(W) < \infty\)}
    \item{\(\dim(V) = \dim(W)\)}
    \item{\(\varphi\) ist bilinear.}
    \item{\(\rg(\varphi) = \dim(v)\)}
  \end{enumerate}
\end{definition}

\begin{bemerkung}
  Falls \(\varphi : V \times W \to K\) bilinear ist, dann ist \(\varphi' : W \times V
  \to K : (w, v) \mapsto \varphi(v, w)\) auch bilinear. Wenn \((V, W, \varphi)\)
  ein duales Paar ist, dann ist auch \((W, V, \varphi')\) auch ein duales Paar.
\end{bemerkung}

\begin{lemma}
  F\"ur gegebene V und W gilt:\\
  \(\{\varphi : V \times W \to K\} \overset{\Phi}{\leftrightarrow} \{F : V \to W^*\}\)\\
  \(\varphi : V \times W \to K \overset{\Phi}{\mapsto} F_{\varphi} : V \to W^* : v \mapsto F_{\varphi}:
  W \to K : w \mapsto \varphi(v, w)\)\\
  \(\varphi_F : V \times W \to K \overset{\Phi^{-1}}\leftarrow F : V \to W^* :
  (v, w) \mapsto F(v)(w)\)\\
  Ferner gilt: \((v, w, \varphi)\) ist ein duales Paar \(\Leftrightarrow F_{\varphi}
  : V \to W^*\) Isomorph.\\
  Beobachtung:\\
  \(\Phi^{-1} \cdot \Phi(\varphi)(v, w)
  = \Phi^{-1}(F_{\varphi}(v) : w \mapsto \varphi(v, w)) = \varphi(v, w)\)
  \begin{proof}
    \(\Phi\) ist wohldefiniert:\\
    \begin{enumerate}
      \item{
       \(F_{\varphi}(v) \in W^*\), weil \(\varphi(v, -)\) lineare in der zweiten
        Koordinate ist.
      }
      \item{
        \(F_{\varphi}\) ist linear, weil \(\varphi\) in der ersten Koordinate 
       linear ist.
      }
    \end{enumerate}
    \(\Phi^{-1} \cdot \Phi = Id_X, \Phi \cdot \Phi^{-1} = Id_Y\) mit X linke,
    Y rechte Menge von \(\Phi\) (in der Definition oben).\\
    Falls \(V, W\) endlichdimensional sind, mit Basen \(\{b_1, \dots, b_n\},
    \{c_1, \dots, c_m\}\). \(A_{\varphi}\) sei die Darstellungmatrix
    bez\"ulgich dieser Basen \(A_{\varphi} = (a_{ij}) = (\varphi(b_i, c_j))\)\\
    Sei \(\{c_1^*, \dots, c_m^*\}\) die duale Basis zu C in \(W^*\).
    \(c_i^*(c_j) = \begin{cases}1 & i = j\\ 0\end{cases}\).\\
    Die Darstellungsmatrix von \(F_{\varphi}\) bez\"uglich \(B, C^*\)
    \((\lb_{ij}) \in M_{m \times n}(K), F_{\varphi}(b_k) = \sum \lb){lk}c_l^* \)
    \((\sum \lb_{lk}c_l^*)(c_r) = \sum \lb_{lk}c_l^*(c_r) = \lb_{rk}
    = F_{\varphi}(b_k)(c_r) = \varphi(b_k, c_r)\\
    \begin{pmatrix}
      \lb_{11} & \lb_{12} & \dots\\
      \lb_{21} & \ddots\\
      \vdots
    \end{pmatrix} = A_{\varphi}^T\\
    \rg(A_{\varphi}^T) = \rg(A)
    \)\\
    \((V, W, \varphi)\) ist eine duales Paar
    \(\Leftrightarrow \dim(V) = \dim(W) = \dim(W^*) = \rg(A_{\varphi})
    = \rg(A_{\varphi}^T\\
    \Leftrightarrow F_{\varphi}\) ist ein Isomorphismus.
  \end{proof}
\end{lemma}

\begin{korrolar}
  Seien V, W endlichdimensional, \(\varphi : V \times W \to K\) eine
  Bilinearform. Dann ist \((V, W, \varphi)\) ein duales Paar genau dann, wenn
  \(\varphi\) nicht ausgeartet ist.\\
  d.h.
  \(\forall v, \in V \varphi(v, w) = 0 \forall w, \in W \Rightarrow v = 0 \land
  \forall w, \in W \varphi(v, w) = 0 \forall v, \in V \Rightarrow w = 0\)
  \begin{proof}
    \(\Rightarrow\)\\
    \begin{enumerate}
      \item{
          Sei \(V \in V\) fest \(: \varphi(v, w) = 0 \forall w \in W
        \Rightarrow F_\varphi(V)(w) = 0 \Rightarrow (V, W, \varphi)\) ist ein
        duales Paar \(\Rightarrow v = 0\)
      }
    \item{
      \((V, W, \varphi)\) duales Paar \(\Rightarrow (W, V, \varphi ')\) auch
        dual \(\Rightarrow F_{\varphi'} : W \to V^*\) Isomorphismus
        \(\Rightarrow w \in W " \varphi(v, w) = 0 \forall v \in V \Rightarrow
        F_{\varphi}(w) = 0 \Rightarrow w = 0\)
      }
    \end{enumerate}
    \(\Leftarrow\)\\
    Es gen\"ugt zu zeigen, dass \(F_{\varphi} : V \to W^*\) ein Isomorphismus
    ist.\\
    a.) \(\Rightarrow F_{\varphi}\)  ist ein Monomorphismus (injektiv).\\
    Insbesondere ist \(\dim(V) \le \dim(W^*) = \dim(W)\). Es gen\"ugt zu zeigen,
    dass \(\dim(V) = \dim(W^*)\), da \(F_{\varphi}\) injektiv ist.\\
    \((W, V, \varphi'\)  ist eine Bilinearform.\\
    Aus b.)  folgt \(\Rightarrow F_{\varphi'}\) injektiv
  \end{proof}
\end{korrolar}

\begin{beispiel}
  \((R^n, <, >)\)
  \(<x_1, \dots, x_n, y_1, \dots, y_n> = \sum x_i y_i\)
\end{beispiel}

\begin{korrolar}
  Sei \(V, W, \varphi)\) ein duales Paar. F\"ur jede Basis
  \(\{c_1, \dots, c_n\}\) aus W gibt es eine Basis \(\{b_1, \dots, b_n\}\) aus
  V, welche dual  zu \(\{c_1, \dots, c_n\}\) ist:
  \(\varphi(b_i, c_j) = S_{ij} = \begin{cases} 1 & i=j\\0 \end{cases}\)
  \begin{proof}
    F\"ur \(\{c_1, \dots, c_n\}\) aus W eine Basis. Sei
    \(\{c_1^*, \dots, c_n^*\}\) die duale Basis aus \(W^*\).\\
    \(F_{\varphi}: V \to W^*\) Iso. Sei \(b_i = F^{-1}_{\varphi}(c_i^*)\).
    \(\{b_1, \dots, b_n\}\) ist dann eine Basis von V.\\
    \(\varphi(b_i, c_j) = \underset{=c)i^*}{F_{\varphi}(b_i)}(c_j)\)
  \end{proof}
\end{korrolar}

\begin{definition}
  Sei \((V, W, \varphi)\) ein duales Paar. Gegeben \(U \subset V\) UVR
  % U ^ umgedrehtes T
  von V. Definiere \(
  U^{\bot} = \{w \in W : \forall u \in U : \varphi(u, w) = 0 \}
  \). \(U^\bot\) ist ein UVR von W.
\end{definition}

\begin{beispiel}
  \((0)^\bot = V, V^\bot = \{0\}\)
\end{beispiel}

\begin{bemerkung}
  \(U \subset V,
  (U^\bot)^\bot = \{v \in V : \forall w \in U^\bot \varphi(v, w) = 0\} = U\)
  \begin{proof}
    \(U \subset (U^\bot)^\bot\) trivial.\\
    Sei \(v \not\in U\). z.Z. \(v \not\in (U^\bot)^\bot\).
    es existiert \(G : V \to K : G|_U = 0, G(v) = 1\). \(G \in V^* \simeq W\)
    (da \(F_{\varphi}\) ein isomorphismus ist. \(\Rightarrow \exists w \in W
    : F_{\varphi}(w) = G\). D.h. \(\forall z \in V : G(z) = \varphi'(z, w)\)
    \(u \in U : 0 = G(u) = \varphi(u, w) \Rightarrow w \in U^\bot\}\) aber
    \(G(v) = 1 = \varphi(v, w) \Rightarrow v \not\in (U^\bot)^\bot\)
  \end{proof}
\end{bemerkung}

\begin{lemma}
  Sei \((V, W, \varphi)\) ein duales Paar und \(U \subset V\) UVR. Dann ist
  \((V /U, U^\bot, \bar\varphi)\) ein duales Paar, wobei
  \(\bar\varphi (v+U, w) = \varphi(v, w)\). Insbesondere gilt
  \(\dim(V) + \sim(U) + \dim(U^\bot)\)
  \begin{proof}
    \"Ubungsaufgabe.
  \end{proof}
\end{lemma}

% missing part of lecture 14.06.2018
\begin{definition}[adjungierter Endomorphismus]
  Sei \(V, W, \varphi)\) ein duales paar und \(G : W \to W\) ein Endomorphismus.
  Der adjungierte Endormorphismus \(G^T : V \to V\) und definiert als
  \(G^T = F^{-1}_{\varphi} \cdot G^* \cdot F_\varphi\)
\end{definition}
% end missing part

\chapter{Euklidische R\"aume}

\begin{definition}[symmetrische Bilinearform]
  Eine Bilinearform \(\varphi : V \times V \to K\) ist symmetrisch, falls
  \(\varphi = \varphi'
  \Leftrightarrow \forall u, v, \in V : \varphi(u, v) = \varphi(u, v)\)
\end{definition}

\begin{bemerkung}
  Seien B, C Basen von V und A die Darstellungsmatrix von \(\varphi\) bzgl.
  B und C. Dann gilt:\\
  \(\varphi\) ist symmetrisch \(\Leftrightarrow A = A^T\)
  \begin{proof}\hfill\break
    \(\Rightarrow\)\\
    Siehe \"Ubungsblattt\\
    \(\Leftarrow\)\\
    \(
    \varphi(u, v)
    = u^T \cdot A \cdot v
    = (A^T \cdot u)^T \cdot v
    = v^T \cdot ((A^T \cdot u)^T)^T
    = v^T \cdot A^T \cdot u
    = v^T \cdot A \cdot u
    = \varphi(v, u)
    \)
  \end{proof}
\end{bemerkung}

\begin{bemerkung}
  Wenn \(\varphi\) symmetrisch ist ist der Begriff der orthogonalit\"at
  wohldefiniert und vor allem symmetrisch.\\
  \(
  u \bot v \Leftrightarrow \varphi(u, v) = 0\\
  v \bot u \Leftrightarrow \varphi(v, u) = 0
  \)
\end{bemerkung}

\begin{beispiel}
  \(\varphi(x, y) = \sum_{i=1}^n x_i y_i\) ist symmetrisch.\\
  \(\varphi(x, x) = \sum_{i=1}^n x_i^2 \ge 0\)
\end{beispiel}

\begin{definition}[quadratische Form]
  Sei \(\varphi : V \times V \to K\) eine symmetrische Bilinearform.
  Die zugeh\"orige quadratische Form ist:
  \(q : V \to K, v \mapsto \varphi(v, v)\)
\end{definition}

\begin{bemerkung}
  \(\dim(V) = n\) Sei \(B = \{b_1, \dots, b_n\}\) eine Basis.\\
  \(\varphi\) hat die Darstellungsmatrix A bzgl. B.\\
  \(Q(v) = \varphi(v, v) = \varphi(\sum \lb_i b_i, \sum \lb_i b_i)
  = \ve{\lb_1, \dots, \lb_n} A \ve{\lb_1 \\ \vdots \\ \lb_n}
  = \sum a_{ij} \lb_i \lb_j
  \)
\end{bemerkung}

\begin{korrolar}
  Fallse \(char(K) \neq 2\):\\
  Jede symmetrische Lienarform is durch ihre quadratische Form eindeutig
  bestimmt.\\
  \(
  q(u + v) = \varphi(u+v, u+v) = \varphi(u, u) + \varphi(u, v) + \varphi(v, u)
  + \varphi(v, v) = \varphi(u, u) + \varphi(v, v)  + 2 \varphi(u, v)
  = q(u) + q(v) + 2 \varphi(u, v)\\
  q(u - v) = \dots = q(u) + q(v) - 2 \varphi(u, v)
  \)\\
  Insbesondere: \(\varphi(u, v) = \frac{q(u+v) - q(u - v)}{4}\)
\end{korrolar}

\begin{definition}[Bilinearform difinit]
  Sei \(V\) ein endlichdimensionaler \(\R\)-Vektorraum und
  \(\varphi : V \times V \to \R\) eine symmetrische  Bilinearform.\\
  \begin{enumerate}
    \item {
      Wir sagen, dass \(\varphi\) positiv semidifinit ist, falls
      \(\varphi(u, u) \ge 0 \forall u \in V\).\\
    }
    \item {
      Falls \(\varphi(u, u) \le 0 \forall u \in V\) ist \(\varphi\) negativ
      semidefinit\\
    }
    \item {
      \(\varphi\) ist positiv definit, falls \(\varphi\) pos. semidifinit ist und
      \(\varphi(u, u) > 0 \forall v \in V \setminus \{0\}\)\\
    }
    \item {
      \(\varphi\) ist negativ definit, falls \(\varphi\) neg. semidefinit ist und
      \(\varphi(u, u) < 0 \forall v \in V \setminus \{0\}\)\\
    }
    \item {
      Ansonsten ist \(\varphi\) indefinit.
    }
  \end{enumerate}
\end{definition}

\begin{beispiel}
  \begin{enumerate}
    \item {
      Standard Skalarprodukt auf \(\R^n\) ist positiv definit
    }
    \item {
      \(\varphi : \R^2 \times \R^2 \to \R,
      ((x_1, y_1), (x_2, y_2)) \mapsto x_1 y_2\)
      Ist positiv semidefinit aber nicht pos. definit (\(\varphi((0, 1), (0, 1))
       = 0\)).
    }
    \item {
      \(\R^2 \times \R^2 \to \R,
      ((x_1, y_1), (x_2, y_2)) \to x_1 y_1 - x_2 y_2\) ist indefinit.
    }
  \end{enumerate}
\end{beispiel}

\begin{bemerkung}
  \(\varphi\) ist pos. (semi) definit \(\Leftrightarrow\) \(-\varphi\)
  ist neg. (semi) definit.
\end{bemerkung}

\begin{definition}[Skalarprodukt]
  Ein Skalarprodukt auf einem endliche nVektorraum V ist eine positiv definite
  symmetrische Bilinearform.
  \(\varphi(u, v) \to <u, v>\)
\end{definition}

\begin{definition}[Euklidischer Raum]
  \(V, <, >)\) ist ein euklidischer Raum, wenn V ein endlicher \(\R\)-Vr ist
  und \(<, >\) ein Skalarprodukt.
\end{definition}

\begin{definition}[Norm, normierter Vektorraum]
  Eine Norm auf einem \mR-Vektorraum V ist eine Abbildung
  \(||\cdot|| : V \to \R\), so dass:
  \begin{enumerate}
    \item {
        \(||v|| >\ge 0\\
        ||v|| = 0 \Leftrightarrow v = 0\)
    }
    \item {
      \(||\lb v|| = |\lb| \cdot ||v||\)
    }
    \item {
      Dreiecksungleichung:\\
      \(|| u + v|| \le ||u|| + ||v||\)
    }
  \end{enumerate}
  \((V, ||\cdot||)\) ist ein normierter Vektorraum.
\end{definition}

\begin{beispiel}
  \begin{enumerate}
    \item {
      \(||x|| = \sqrt{x_1^2 + \dots + x_n^2}\), die euklidische Norm. 
    }
    \item {
      \(||x||_{\infty_1} = \sum |x_i|\)
    }
    \item {
      \(||x||_{\infty_2} = max(|x_i)\)
    }
  \end{enumerate}
\end{beispiel}

\begin{definition}
  Sei \((V, <, >)\) ein euklidischer Raum und definiere
  \(||\cdot|| : V 'to \R, v\ mapsto \sqrt(<v, v>)\). Die Abbildung ist
  wohldefiniert und induziert sie eine Norm auf V.
\end{definition}

\begin{lemma}[Cauchy-Schwarz Ungleichung]
  \(\forall v, w \in V: |<v, w>| \le ||v|| \cdot ||w||\)
  \begin{proof}
    Falls \(w = 0\) ist die Aussage trivial. ObdA. ist \(w \neq 0\), also
    \(||w|| > 0\).\\
    Sei \(\lb \in R\) beliebig.\\
    \(
    0 \le <v - \lb w, v - \lb w> = ||v||^2 + \lb^2 ||w||^2 - 2 \lb <v, w>
    \)\\
    Insbesondere \(2\lb <v, w> \le \lb^2 ||w||^2 + ||v||^2\).\\
    Falls \(\lb = \frac{<v, w>}{||w||^2} \in \R\):\\
    \(2 \frac{<v, w>^2}{||w||^2} \le ||v||^2 + \frac{<v, w>^2}{||w||^2}\\
    \Rightarrow \frac{<v, w>^2}{||w||^2} \ge ||v||^2
    \Rightarrow <v, w>^2 < ||v||^2 \cdot ||w||^2\\
    \Rightarrow |<v, w> \le ||v|| \cdot ||w||^2
    \)
  \end{proof}
\end{lemma}

\begin{korrolar}
  \((V, <, >)\) ist ein Euklidischer Raum:\\
  \(||\cdot|| : V \to \R, v \to \sqrt{<v, v>}\) ist eine Norm.
  \begin{proof}
  \begin{enumerate}
    \item {
      Klar.
    }
    \item {
      \(||\lb v|| = \sqrt{<\lb v, \lb v} = \sqrt{\lb^2 <v, v>}
      = |\lb| \cdot ||v|\)
    }
    \item {
      Es gen\"ugt zu zeigen, dass \(||u+v||^2 \le (||u|| + ||v||)^2\)\\
      \(
      ||u+v||^2 = <u+v, u + v>
      = <u,u> + <v,v>+ 2 <u,v>
      \le ||u||^2 + ||v||^2 + 2 ||u|| ||v||
      = (||u|| + ||v||)^2
      \)
    }
  \end{enumerate}
  \end{proof}
\end{korrolar}

\begin{definition}
  Sei \((V, <, >)\) ein euklidischer Raum.\\
  \(-1 \le \underset{=\cos(\Theta)}{\frac{<v, u>}{||v|| ||u||}} < 1\)\\
  Mit \(\Theta \in (0, \pi]\) ist eindeutig bestimmt. \(\Theta\) ist der Winkel
  zwischen u und v.\\
  \begin{bemerkung}
    \(u \bot v
    \Leftrightarrow <u v> = 0
    \Leftrightarrow\) der Winkel \(\frac{\pi}{2}\) ist.
  \end{bemerkung}
\end{definition}

\begin{satz}[Satz von Pythagoras]
  Sei \((V, <, >)\) ein euklidischer Raum. Dann gilt:\\
  \(v \bot w \Leftrightarrow ||v+w||^2 = ||v||^2 + ||w||^2\)\\
  \begin{proof}
    \(||v + w||^2 = <v + w, v + w>
    = <v, v> + <w, w> + 2 <v, w>
    = ||v||^2 + ||W||^2 + 2<v, w> = ||v||^2 + ||w||^2
    \Leftrightarrow <v, w> = 0 \Leftrightarrow v \bot w\)
  \end{proof}
\end{satz}

\begin{definition}[Orthogonales System]
  Sei \(\varphi : V \times V \to K\) eine symmetrische Bilinearform.
  Ein orthogonales System bez\"uglich \(\varphi\) ist eine Kollektion von
  Verktoren M, so dass:\\
  \(
  0 \not\in M, u, v \in M \land u \neq v : \varphi(u, v) = 0
  \).\\
  Ein Orthonormales System M ist eine Kollektion von Vektoren, so dass:\\
  \(
    \varphi(u, v) = \begin{cases}0 & u \neq v \\ 1 & u = v\end{cases}
  \)
\end{definition}

\begin{bemerkung}
  Dementsprechend definieren wir Orthogonalbasis und Orthonomalbasis (ONB).
\end{bemerkung}

\begin{beispiel}
  Standardbasis \(\{e_1, \dots, e_n\}\) in \((R^n, <, >)\) mit dem
  Standarskalarprodukt.
\end{beispiel}

\begin{satz}
  Sei \(char(K) \neq 2\). Jede symmetrische Bilinearform \(\varphi\) auf einem
  endlichdimensionalen K-VR V l\"asst sich bei einer geeigneten Basisauswahl
  durch eine Diagonalmatrix darstellen.\\
  Ferner ist \(\varphi\) nicht ausgartet \(\Leftrightarrow\) kein Eigenwert der
  Matrix ist null.
  \begin{proof}
    Es gen\"ugt zu zeigen, dass \((V, \varphi)\) eine Basis
    \(\{b_1, \dots, b_n\}\) besitzt, welche aus paarweise orthogonalen Vektoren
    besteht.\\
    Dann ist die Darstellungsmatrix von \(\varphi\) bzgl.
    \(\{b_1, \dots, b_n\}\):\\
    \(
    \begin{pmatrix}
      \varphi(b_1, b_1) & \dots & 0
      \vdots & \ddots & \vdots
      0 & \dots & \varphi(b_n, b_n)
    \end{pmatrix}
    \)
    Sei \(q : V \to k : v \mapsto \varphi(v, v)\) die zugehoerige quadratische
    Form.\\
    Falls \(q(v) = 0 \forall v \in V
    \Rightarrow \varphi(u, v) = 0 \forall u, v \in V\).
    Dann besteht jede Basis von V aus paarweise orthogonalesn Bektoren.\\
    Sonst existiert \(b_1 \in V : q(b_1) \neq 0\).\\
    \(F : V \to K, v \mapsto \varphi(v, b_1), F \neq 0, \Ima(F) = K\) als K-Vr.
    \(\Rightarrow \dim(\Ker(F)) = n-1\).\\
    \(\Ker(F) = \{v \in V ; \varphi(v, b_1) = 0\} = \{v, \in V ; v \bot b_1\}
    = Span(b_1)^\bot\).\\
    Nach Induktion auf der Dimension von V existiert eine Basis von
    \(b_2, \dots, b_n\) von \(\Ker(F)\) welche aus paarweise orthogonalen
    Vektoren besteht.\\
    Die Basis \(\{b_1, \dots, b_n\}\) ist eine Orthogonalbasis von V.\\
    \begin{bemerkung}
      Die Eigenwerte der Matrix h\"angen nicht von der Basis ab.\\
      \(\Rightarrow\)
      Eigenwerte sind \(\varphi(b_1, b_1), \dots, \varphi(b_n, b_n)\)\\
      \(\Rightarrow\)\\
      Angenommen, dass \(\mu_j = \varphi(b_j, b_j) = 0\) w\"are.\\
      \(\varphi(b_j, b_i) = \begin{cases} 0 & i = j \\ 0 & i \neq j\end{cases}
      \) \(\varphi(b_j, -) = V \to K\) ist die triviale Abbildung und 
      \(b_j \neq 0\).\\
      \(\Leftarrow\)\\
      Sei \(v \in V \setminus \{0\}\) beliebig. Zu zeigen:
      \(\varphi(v, -) : V \to K\) ist nicht trivial.\\
      \(v = \sum_{i=1}^n b_i \Rightarrow \exists i : \lb_i \neq 0\)\\
      \(\varphi(v, b_i) = \sum_{j=1}^n \lb_j \varphi(b_j, b_i)
      = \lb \varphi(b_i, b_j) \neq 0\)
    \end{bemerkung}
  \end{proof}
\end{satz}

\begin{korrolar}
  Sei \(char(K) \neq 2\). Falls in K jedes Element ein Quadrat ist, dann
  l\"asst sich jede symmetrische Bilinearform  durch eine Matrix der Form
  \(\begin{pmatrix}
    1\\
    &\ddots\\
    &&1\\
    &&&0\\
    &&&&\ddots\\
    &&&&&0
  \end{pmatrix}\) darstellen.
  \begin{proof}
    Es existiert eine Orthogonalbasis \(\{b_1, \dots, b_n\}\) f\"ur
    \(\varphi : V \times V \to K\)\\
    \(c_i = \begin{cases}
    \frac{b_i}{\sqrt{\varphi(b_i, b_i)}} & \varphi(b_i, b_i) \neq 0\\
    b_i & \text{ansonsten}
    \end{cases}
    \)\\
    OBdA. k\"onnen wir annehmen, dass \(\{c_1, \dots, c_n\}\) ist so geordnet,
    dass:\\
    \(\varphi(c_i, c_i) = 1, i \le k\\
    \varphi(c_j, c_j) = 0,  j > k\)\\
    Die Darstellungsmatrix von \(\varphi\) bzgl \(\{c_1, \dots, c_n\}\) ist
    dann:\\
    \(
    \begin{pmatrix}
    1\\
    &\ddots\\
    &&1\\
    &&&0\\
    &&&&\ddots\\
    &&&&&0
    \end{pmatrix}
    \)
  \end{proof}
\end{korrolar}

\begin{satz}[Satz von Sylvester]
  Jede symmetrische Bilinearform \(\varphi\) auf einem endlichdimensionalem
  \(\R\)-Vektorraum l\"asst sich bei geeigneter Basisauswahl durch eine Matrix
  der Form:\\
  \(\begin{pmatrix}
  1\\
  &\ddots\\
  &&1\\
  &&&-1\\
  &&&&\ddots\\
  &&&&&-1\\
  &&&&&&0\\
  &&&&&&&\ddots\\
  &&&&&&&&0
  \end{pmatrix}
  \) datstellen, wobei die matrix \(p\) 1er, \(q\) -1 er und r 0er hat.\\
  Ferner h\"angen die Zahen p, q und r nur von \(\varphi\) ab.
  \begin{proof}
    Sei \(\{b_1, \dots, b_n\}\) eine Orthogonalbasis f\'ur \(\varphi\).\\
    \(
    c_i = \begin{cases}
      \frac{b_i}{\sqrt{\varphi(b_i, b_i)}} & \varphi(b_i, b_i) > 0\\
      \frac{b_i}{\sqrt{-\varphi(b_i, b_i)}} & \varphi(b_i, b_i) < 0\\
      0 & \varphi(b_i, b_i) = 0
    \end{cases}
    \)\\
    Nach Umordnung der \(c_i\)'s ist die Darstellungsmatrix, dann  
    \(\begin{pmatrix}
      1\\
      &\ddots\\
      &&1\\
      &&&-1\\
      &&&&\ddots\\
      &&&&&-1\\
      &&&&&&0\\
      &&&&&&&\ddots\\
      &&&&&&&&0
    \end{pmatrix}
    \)\\
    \begin{bemerkung}
      \(\varphi |_{\Span(c_1, \dots, c_p) \times \Span(c_1, \dots, c_p)}\) ist
      positiv definit.
      \begin{proof}
        \(\varphi(\sum_{i=1}^p \lb_i c_i, \sum_{i=1}^p \lb_j c_j)
        = \sum_{i, j} \lb_i, \lb_j \varphi(c_i, c_j) = \sum_{i = j} \lb_i^2\)
      \end{proof}
    \end{bemerkung}
    Sei \(U \subset V\) der gr\"osste Unterraum von V derart, dass
    \(\varphi|_{U \times U}\) positiv definit ist.\\
    \(\Span(c_1, \dots, c_p) \subset U\).\\
    z.Zeigen: \(p = \dim(U)\)\\
    Ansonsten \(U \cap \Span(c_{p+1}, \dots, c_n) \neq 0\).\\
    Sei \(0 \neq v \in U \cap \Span(c_{p+1}, \dots, c_n)\).\\
    \(c = \sum_{i=p+1}^n \lb_i, c_i\)\\
    \(0 \le \varphi(v, v) = \sum_{i=p+1}^n \lb^2 \varphi(c_i, c_i) \le 0
    \Rightarrow v = 0\)
    \(\Rightarrow p = \dim(U)\)\\
    \(q, r\) bestimmen:\\
    \(\rg(\varphi) = p + q\), also ist q eindeutig bestimmt. r ist dann
    \(n - \rg(\varphi)\)
    \end{proof}
\end{satz}

\begin{definition}[Signatur]
  \(Signatur(\varphi) = p - q\)
\end{definition}

% etwas anderers...
\begin{beispiel}
  \(
  \varphi = \R^2 \times \R^2 \to \R : ((x, y), (x_2, y_2)) \mapsto 2 x_1x_2 - y_1y_2\\
  \)
  \(\varphi\) ist positiv definit auf \(\Span((1, 0))\):\\
  \(\varphi((\lb, 0), (\lb, 0)) = 2 \lb^2 \ge 0\)\\
  \(\varphi\) ist positiv definit auf \(\Span((1, 1))\):\\
  \(\varphi((\lb, \lb), (\lb, \lb)) = \lb^2 \ge 0\)\\
  \(\varphi\) ist nicht positiv definit auf \(\Span(1, 0) + \Span((1, 1)) = \R^2\):\\
  \(\varphi((0, 1), (0, 1)) = -1\)\\
  Damit gibt es keinen gr\"o\ss{}ten Unterraum, so wie im Beweis zum Satz von
  Sylvester angenommen.
\end{beispiel}

\begin{proof} Korrektur zum Satz on Sylvester\\
 Wir wollen p Eindeutig bestimmen. \(\varphi\) ist positiv definit auf
 \(\Span(c_1, \dots, c_p)\).\\
 Sei \(h = \max\{\dim(U) | U \subset V : \varphi|_{U \times U} \text{ ist positiv definit}\}\)
 Sei \(U \subset V\) ein UVR der Dimension h, so dass \(\varphi|_{U \times U}\)
 positiv definit ist. Wir zeigen, dass \(U \cap \Span(c_{p+1}, \dots, c_n) = \{0\}\)
 \(\Rightarrow h+n-p = \dim(U) + n - p = \dim(\Span(c_{p+1}, \dots, c_n)) \le n\)\\
 \(\Rightarrow h \le p\)
\end{proof}

\chapter{Unit\"are R\"aume}

\begin{definition}[Unit\"arer Raum]
  Ein unit\"arer Raum V ist ein \(\C\)-Vektorraum zusammen mit einem komplexen
  Skalarprodukt \(<, > : V \times V \to \C\) mit folgenden Eigenschaften:\\
  \begin{enumerate}
    \item {
      \(<v, w> = \bar{<w, v>}\) wobei \(\overline{a+bi} = a - bi\)
    }
    \item {
      \(<v + v', w> = <v, w> + <v', w>\)
    }
    \item {
      \(<\lb v, w> = \lb <v, w>\)
    }
    \item {
      \(<v, v> \in \R\) und ferner \(<v, v> > 0\) f\"ur \(v \neq 0\)
    }
  \end{enumerate}
\end{definition}

\begin{beispiel}
\(<x, y> = \sum_{i=1}^n x_i \bar{y_i}\)
\end{beispiel}

\begin{bemerkung}
  Die Abbildung \(<, >\) ist nicht bilinear sondern hermitsch sesquilinear.\\
  \(<v, \lb w + \mu w'> = \bar\lb <v, w> + \bar\mu<v, w'>\)
  \begin{proof}
    \(<v, \lb w + \mu w'>
    = \overline{<\lb w + \mu w', v>}
    = \bar\lb \overline{<w, v>} + \bar\mu \overline{<w', v>}
    = \bar\lb<v, w> + \bar\mu<v, w'>
    \)
  \end{proof}
\end{bemerkung}

\begin{bemerkung}
  F\"ur einen unit\"aren Raum V ist \(||\cdot|| : V \to \R, v \mapsto \sqrt{<v, v>}\)\\
  \(
    ||v|| > 0\\
    ||\lb \cdot v|| = |lb| ||v||\\
    ||v + w|| \le ||v|| + ||w||
  \)
\end{bemerkung}

\begin{bemerkung}
  Sei \((V, <, >)\) unit\"arer Raum:\\
  \(v \bot w \Leftrightarrow <v, w> = 0\)\\
  Orthogonalit\"at, orthogonales System und orthonomales System sowie 
  orthogonal- und orthonomalbasen werden in Unit\"are R\"aumenn analog
  zu Euklidischen R\"aumen definiert.
\end{bemerkung}

\begin{definition}[Orthonormalbasis]
  Sei \(V, <, >\) ein euklidischer oder unit\"arer Raum. Eine Orthonomalbasis
  \(V\) ist eine Basis \(B = \{b_i\}\), so dass
  \(i \neq j \Rightarrow b_i \bot b_j, ||b_i|| = 1\).\\
\end{definition}

\begin{bemerkung}
  Jedes orthogonale System ist linear unabh\"angig.
  \begin{proof}
    \(\sum \lb_i b_i = 0\\
    0
    = <\sum\lb_i b_i, b_j>
    = sum \lb_i <b_i, b_j>
    = \lb_j <b_j, b_j>
    \Rightarrow \lb_j = 0
    \)
  \end{proof}
\end{bemerkung}

\begin{lemma}
  Sei \((V, <, >)\) ein euklidischer oder unit\"arer Raum und
  \(\{b_1, \dots, b_n\}\) eine ONB (orthonormalbasis). Dann gilt:\\
  \begin{enumerate}
    \item {
      \(\forall v \in V v = \sum_{i=1}^n <v, b_i> b_i\)
    }
    \item {
      \(v = \sum_{i=0}^n \lb_i b_i, w = \sum_{i=0}^n \mu_i b_i\\
      \Rightarrow <v, w> = \sum \lb_i \bar{\mu_i}\)
    }
    \item {
      Sei \(F : V \to V\) ein Endomorphismus. Dann ist die Darstellungsmatrix
      A von F bez\"uglich \(\{b_1, \dots, b_n\}\) gegeben durch
      \(a_{ij} = <F(b_j), b_i>\)
    }
  \end{enumerate}
  
  \begin{proof}
    \begin{enumerate}
      \item {
        \(v = \sum_{i=1}^n \lb_i b_i\)\\
        \(<v, b_j> = \sum_{i=1}^n \lb_i <b_i, b_j> = \lb_j\)
      }
      \item {
        \(<\sum \lb_i b_i, \sum \mu_j b_j>
        = \sum_{i, j} \lb_i \bar{\mu_j} <b_i, b_j>
        = \sum \lb_i \bar{\mu_i}
        \)
      }
      \item {
        \(
          A = (a_{ij})\\
          F(b_1), \dots, F(b_n)\\
        \)
        \(a_{ij}\) ist die Koordinate von \(Fb_j)\) bzgl \(b_i\). Aus 1. folgt:
        \(a_{ij} = <F(B_j), b_i>\)
      }
    \end{enumerate}
  \end{proof}
\end{lemma} 

\begin{satz}[Gram-Schmidtsches Orthonormaliserungsverfahren]
  Sei V ein euklidischer oder uni\"arer Vektorraum. Gegeben
  \(\{v_1, \dots, v_n\}\) lin. unabh. Dann gibt es ein orthonormalsystem
  \(\{e_1, \dots, e_n\}\), dass
  \(\Span(v_1, \dots, v_n) = \Span(e_1, \dots, e_n)\).\\
  Insbesondere falls \(V\) endlichdimensional ist besitzt V eine ONB.
  \begin{proof}
    Zwei Schritte: zuerst aus \(v_1, \dots, v_n\)
    eine orthogonalbasis konstruieren, dann diese Vektoren normalisieren.\\
    \(e_1, \dots, e_n\) werden rekursiv definiert.\\
    \(e_1' = v_1\), \(e_1 = \frac{e_1'}{||e_1'||}\)\\
    Angenommen \(e_1, \dots, e_k\) wurden konstruiert, so dass
    \(i \neq j \Rightarrow e_i \bot e_j, ||e_i|| = 1\) und
    \(\Span(e_1, \dots, e_k) = \Span(v_1, \dots, v_k)\)\\
    \(e_{k+1}' = v_{k+1} - \sum_{i=1}^k <v_{k+1}, e_i> e_i\)\\
    \(e_{k+1}' \neq 0\) Sonst ist \(v_{k+1} \in \Span(e_1, \dots, e_k) = \Span(v_1, \dots, v_k)\)\\
    \(e_{k+1} = \frac{e_{k+1'}}{||e_{k+1}'||}\)
    z.Zeigen:\\
    \(e_{k+1} \bot e_j : <e_{k+1}. e_j> = \frac{1}{||e_{k+1}'}' <v_{k+1} - \sum_{i=1}^k <v_{k+1}, e_i> e_i, e_j>
    = \frac{1}{||e_{k+1}'}' <v_{k+1}, e_i> - \sum_{i=1}^k <v_{k+1}, e_i><e_i, e_j>)
    = \frac{1}{||e_{k+1}'}' <v_{k+1}, e_i> - <v_{k+1}, e_j>
    = 0
    \)
    \(\Span(e_1, \dots, e_{k+1}) = \Span(v_1, \dots, v_{k+1}\\
    e_{k-1} \in \Span(v_{k-1}, e_1, \dots, e_k) = \Span(v_{k+1}, v_1, \dots, v_k)\\
    v_{k+1} = ||e_{k+1}'|| e_{k+1} + \sum_{j=1}^k<v_{k-1}, e_j>e_j \in \Span(e_1, \dots, e_{k+1})
    \)
  \end{proof}
\end{satz}

\begin{korrolar}
  \((V, <, >)\) euklidisch oder unit\"ar endlich dimensional und \(D \subset V\)
  ein Orthonormales system dann \(\exists\) ONB \(B\), so dass \(D \subset B\).
  \begin{proof}
    Sei \(D= \{v_1, \dots, v_k\}\) und erg\"anze zu einer Basis von
    \(\{v_1, \dots, v_n\}\) von V. Konstriuiere eine ONB.\\
    z.Zeigen: \(i \le k \Rightarrow e_i = v_i\).\\
    \(e_1 = \frac{v_1}{||v_1||} = v_1\)\\
    Annahme \(i \le j \Rightarrow e_i = v_j\)\\
    \(e_{i+1} = \frac{e_{i+1}'}{||e_{i+1}'}, e_{i+1}' = v_{i+1} - \sum_{j=1}^i <v_{i+1}, e_j> v_j = v_{i+1}\)
  \end{proof}
\end{korrolar}

\newpage
\renewcommand{\listtheoremname}{Satz- und Definitionsverzeichnis}
% \listoftheorems[ignoreall, show={definition}, show={satz}, show={lemma}, show={definitionn}, show={korrolar}, show={altdefinition}]
\addcontentsline{toc}{chapter}{Satz- und Definitionsverzeichnis}
\newpage
\printindex
\end{document}
